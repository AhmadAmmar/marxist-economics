% content.tex
% Main body of the Marxian Political Economy pamphlet

%=========================================================
\section{Why this pamphlet? From meme to method}
\label{sec:intro-meme-method}
%=========================================================

On social media, you might see a graph about ``the tendency of the rate of
profit to fall'' or a meme about capitalism ``needing infinite growth on a
finite planet''. The references are often to Karl~Marx, but the underlying
method and mathematics are rarely explained carefully.

This pamphlet is a long-form attempt to do that work.

A major reason careful method matters today is the culture of techno-utopianism.
We are repeatedly told that some new technology---a platform, a blockchain, a ``smart'' city,
a generative model---will fix politics, fix inequality, or fix ecology. Under capitalism,
these claims typically slide past the central question: \emph{what changes in property relations,
and therefore in the organisation of production and reproduction}? A tool can be technically impressive
and still function primarily as a way to create new markets, intensify labour, privatise public goods,
or inflate speculative valuations.

In Marxian terms, this hype often functions as the cultural face of fictitious capital: it helps capitalise expectations of future monopoly positions and rent streams into present valuations, long before any socially useful diffusion has occurred. This pamphlet therefore treats technology neither as a saviour nor as a demon. The Marxian approach is
more concrete: ask who owns the means of production (including digital infrastructure), what labour processes
the tool reorganises, how it shifts socially necessary labour time and the pace of work, and how it creates
new opportunities for surplus extraction, monopoly rent, and control. That is the only way to separate
technical possibility from capitalist deployment.

With that in mind, it is aimed at two overlapping readers:


\begin{itemize}[leftmargin=1.5em]
  \item the \textbf{university student}---often taught neoclassical economics
  and game theory while hearing that Marx is ``relevant but outdated'', and

  \item the \textbf{modern urban worker}---in a call centre, logistics
  warehouse, software firm, school, hospital, or gig platform---who feels
  that their life is organised around work, debt, and stress, yet is told
  that ``the economy'' is an abstract thing they have no say in.
\end{itemize}

Marxian political economy offers:

\begin{enumerate}[leftmargin=1.5em]
  \item a \textbf{method}---dialectical and historical materialism---for
  understanding how societies organise production and reproduction;

  \item a \textbf{set of concepts and equations}---value, surplus value,
  constant and variable capital, the rate of profit, the organic composition
  of capital---that link everyday exploitation to macro dynamics; and

  \item a \textbf{perspective on social change}: why capitalism generates
  recurrent crises and how collective action can point beyond it.
\end{enumerate}

The aim is not to turn you into a specialist, but to give you enough
conceptual and mathematical grip that you can read Marx, contemporary Marxist
work, and even mainstream economic reports without feeling like the equations
and jargon are a priestly language.

Throughout, we will move back and forth between:

\begin{itemize}[leftmargin=1.5em]
  \item \emph{basic definitions} (commodities, value, labour-power),
  \item \emph{formal relations} (simple equations and tables), and
  \item \emph{concrete examples} (smartphone supply chains, garment workers in
  Bangladesh, real-estate speculation in Pakistan, logistics warehouses,
  platform work, AI automation).
\end{itemize}

Whenever we write $c$, $v$, $s$, $r$, or $e$, we are using
a standard Marxian shorthand. A summary of these symbols is given in Appendix~\ref{sec:notation}, and they are developed
in more detail in Sections~\ref{sec:labour-surplus}--\ref{sec:rate-profit}.


%=========================================================
\section{Method: dialectical and historical materialism}
\label{sec:method}
%=========================================================

Marx did not start from ethics (\emph{capitalism is bad because it is
unfair}) but from a method of analysing how societies reproduce themselves
over time. In his words, capitalism is a ``definite mode of production''
with its own laws of motion.\cite{Marx1867,MarxGrundrisse}

\subsection{Materialism}

Materialism starts from the simple fact that humans must labour on nature to
survive: grow food, build shelter, care for children, maintain bodies, repair
machines. The way this labour is organised---who controls the means of
production, who performs which tasks, who receives what share of the output---
gives rise to classes and class struggle.

The categories of Marxian economics are therefore not timeless. They are
historical forms of social relations. Wage-labour, profit, interest, rent,
and the world market are specific to capitalism, just as tribute, serf
labour, and feudal dues were specific to feudalism.

\subsection{Dialectics}

Dialectics means thinking in terms of \textbf{contradictions} and
\textbf{totalities}, not isolated variables.

\begin{itemize}[leftmargin=1.5em]
  \item A commodity is simultaneously useful (\emph{use-value}) and an item
  for sale (\emph{exchange-value}).
  \item Labour is simultaneously concrete activity (teaching, sewing,
  coding) and an abstract expenditure of human energy that creates value.
  \item Capitalist development raises productivity but also tends to reduce
  the share of living labour in production, undermining the very source of
  surplus value.
\end{itemize}

The dialectical move is to see how these opposites presuppose and transform
each other, rather than choosing one side. For instance, capitalism cannot
live without wage-labour; yet it constantly tries to replace workers with
machines and algorithms. It cannot live without states and borders; yet it
also needs global supply chains and mobile capital.

\subsection{Historicity}

Historical materialism insists that our categories are historically specific.
The laws of motion of capitalism emerge from the separation of producers from
the means of production, the generalisation of commodity exchange, and the
subordination of production to profit.

The task is therefore to reconstruct, conceptually and empirically, how this
mode of production works: from the commodity up to crises, imperialism, and
the possibility of transition.\cite{EngelsAntiDuhring,Shaikh2016}

\paragraph{A note on levels of abstraction.}
The ordering of this pamphlet follows a Marxist movement from the abstract to the
concrete, and from the simplest form of appearance to the more complex totality it
presupposes. We begin with the commodity and the value-form, because capitalism first
confronts us as a world of commodities, prices, and exchange relations. We then move
to labour-power as a peculiar commodity, because only there do we find the hinge that
makes profit compatible with apparently equal exchange. From that pivot we develop the
production of surplus value (necessary and surplus labour time) and the rate of
exploitation, before turning to how competition redistributes surplus value and, through
the formation of prices of production, generates systematic divergences between values
and market prices (see Section~\ref{sec:prices-production}). On that basis we can treat credit and fictitious capital as claims
on future surplus value that can expand well beyond the limits set by production and
realisation, and therefore become a central transmission belt of crisis. We then extend
the analysis to world-scale determinations---imperialism and unequal exchange---and to
the reproduction of labour-power (social reproduction) as a material condition of
accumulation, not a separate moral or cultural ``add-on.'' Finally, we return to
contemporary restructurings of the labour process and capitalist control (including
platform power and AI): changes in technology and organisation that can transform how
exploitation is administered and intensified, without abolishing the underlying
relations that make value and accumulation possible.


%=========================================================
\section{Commodities, value, use-value, and exchange-value}
\label{sec:commodities}
%=========================================================

Marx opens \emph{Capital} with the line that the wealth of societies where the
capitalist mode of production rules appears as an ``immense collection of
commodities''.\cite{Marx1867} A \textbf{commodity} is, at minimum:

\begin{enumerate}[leftmargin=1.5em]
  \item a useful thing (\textbf{use-value}); and
  \item produced for exchange, not direct use (\textbf{exchange-value}).
\end{enumerate}

A handmade meal cooked by you at home for yourself is not a commodity; the
same meal produced in a commercial kitchen and sold through a delivery app
is.

\subsection{Use-value}

Use-value is about the concrete usefulness of a thing: food nourishes, a
phone enables communication, a medicine treats a disease. Use-values are
qualitative and historically varied. They exist in all societies.

\subsection{Exchange-value}

Exchange-value is the quantitative relation in which one commodity exchanges
for another on the market: one coat for twenty metres of linen, one hour of a
developer's labour for a certain wage, one laptop for a certain sum of
money. In capitalist societies, this relation is typically expressed in money.

Money and markets can exist in many kinds of society; merchants traded, coins circulated,
and taxes or tribute were sometimes paid in money long before capitalism. In many pre-capitalist
settings exchange was episodic, bounded by custom, tribute, or status, and money did not function
as a universal mediator of everyday life. Large parts of life were organised through direct
obligation and production for use (rent in kind, customary dues, corv\'ee labour, household and
village production), with commodity exchange often limited or secondary to these relations.
Capitalism generalises commodity exchange so that money becomes the \emph{universal equivalent}:
most goods and services are routinely bought and sold, and survival itself depends on access to money.
That is why under capitalism the typical question is not only \emph{what} is produced, but
\emph{what can be sold}, and for how much---including the selling of labour-power for wages.

When exchange-value is expressed in money, it appears as a \textbf{price}; later we return
to why prices do not always coincide with underlying values.

What interests Marx is: \emph{what regulates these exchange ratios over the
long run}? Why does a high-end smartphone sell for many times the price of a
sack of rice, even though rice is more essential to life?

\subsection{Labour and value}

Marx's answer is the \LTV. Very roughly, the value of a commodity grows in
proportion to the amount of socially necessary labour time it takes to make it.

\begin{equation}
  \Vi{i} = \MELT\,\LSN{i},
\end{equation}

where:
\begin{itemize}
  \item $\Vi{i}$ is the value (in money terms) of \emph{one unit} of commodity $i$;
  \item $\LSN{i}$ is the \emph{socially necessary labour time} required
        to produce one unit of commodity $i$: that is, the labour time needed
        under \emph{average} conditions of production, with average skill and
        intensity, using the prevailing technology; and
\item $\MELT$ is the \textbf{monetary expression of labour time} (MELT): a proportionality
      constant that converts labour-time into money magnitudes, i.e.\ the amount of money
      that corresponds to one hour of socially necessary labour time in a given country
      and period (money per hour of \snlt).
\end{itemize}

We formalise \snlt{} as a social average across producers in Section~\ref{sec:abstract-concrete} (Eq.~\eqref{eq:snlt}).


A further clarification is needed. The labour-time that regulates value is \emph{social}
labour reduced to a common unit of \textbf{simple labour}. Concrete labours differ in
skill, training, and intensity; Marx treats \emph{complex} (skilled) labour as
\emph{multiplied simple labour}---that is, one hour of skilled labour counts as several
hours of average labour. Capitalism performs this reduction in practice behind our backs,
through training costs (often borne by households and the state), wage hierarchies, and
competitive norms that define what counts as ``average'' labour in each branch.

If a certain smartphone model typically takes $4$ hours of direct and indirect labour
time across the global supply chain (direct assembly labour plus the labour embodied in
parts, transport, and logistics), and a basic T-shirt takes $0.5$ hours, then the ratio
of their values (and thus their exchange-values in the value-theoretic sense) will tend,
other things equal, to be about
\[
  \frac{\Vi{\text{phone}}}{\Vi{\text{shirt}}}
  = \frac{\MELT\,\LSN{\text{phone}}}{\MELT\,\LSN{\text{shirt}}}
  \approx \frac{4}{0.5} = 8,
\]
so the phone’s exchange-value tends to be about eight times the shirt’s (an $8{:}1$ ratio).

Market prices fluctuate around these values due to supply, demand, speculation, brand
power, and state policy, but in Marx's framework the underlying regulator of prices over
the long run is labour time \emph{through competition and the formation of prices of production} (see Section~\ref{sec:prices-production}).
In other words, value does not appear on the surface as a direct ``price tag'': it asserts itself
through competitive pressures, productivity differences, and the redistribution of surplus value
across capitals.

\subsection{Nature, wealth, and the source of value}
\label{sec:nature-value}

A common confusion (especially in climate politics) is to say: ``If nature is essential,
then nature must be the source of value.'' Marx’s position is more precise.

\begin{itemize}[leftmargin=1.5em]
  \item \textbf{Nature is indispensable to production}: it provides materials, energy flows,
  ecosystems, and conditions of life. Without nature there is no labour process and no material wealth.
  \item \textbf{Nature is a source of use-values (material wealth)}: some use-values exist with little or
  no human labour (sunlight, air, wild fish), while most are produced by labour working on nature.
  \item \textbf{Value} (in the Marxian sense) is a \emph{social form} specific to generalised commodity
  production: it is labour-time that society is compelled to treat as commensurable through exchange,
  regulated by \textbf{socially necessary labour time} and expressed in money.
\end{itemize}

So Marx does not deny nature. He is distinguishing categories. Nature helps determine \emph{how much}
labour is required (by shaping productivity conditions and material inputs), but under capitalism it is
labour-time that appears \emph{as value} in the dominant social accounting of wealth.

This distinction matters because capitalism can \emph{increase} physical throughput and ecological harm
while \emph{reducing} the value embodied per unit (if productivity rises and \snlt{} falls). In other words:
more stuff, faster, with higher energy and extraction, can coincide with falling labour-time per unit. That
is one reason ecological crisis and profitability pressures can move together rather than separately.

When natural conditions become \emph{monopolised} (land, location, minerals), payments for access appear as
\emph{rent}. Rent is not value created by nature; it is a claim on surplus value created elsewhere, enforced
by property rights over natural conditions.\cite{Marx1894}


\subsection{A modern example: the smartphone}

Consider a mid-range smartphone assembled in southern China for a large
brand.\cite{DedrickKraemerLinden2010,Smith2016}

\begin{itemize}[leftmargin=1.5em]
  \item The phone retails in London for, say, \pounds800.
  \item The total wage bill for assembly workers per phone might be on the
  order of \pounds10--\pounds15.
  \item The rest of the price covers constant capital (components, machinery
  depreciation, logistics), marketing, design, and various profits and rents.
\end{itemize}

The exchange-value of the phone vastly exceeds the direct wages paid in the
factory. Part of the explanation is that value from many segments of the
global supply chain is realised in the final sale; part is that monopolistic
firms capture \emph{surplus value} generated across the chain. To understand
how, we need to go deeper into labour, surplus, and capital.



%=========================================================
\section{Abstract and concrete labour; socially necessary labour time}
\label{sec:abstract-concrete}
%=========================================================

Every act of labour is concrete: a nurse tending patients, a teacher
preparing a lesson, a coder debugging, a farmer transplanting rice seedlings.
Concrete labour produces specific use-values.

At the same time, under capitalism, labour also counts as an \emph{abstract}
expenditure of human energy that can be compared across jobs: eight hours of
call-centre work and eight hours of garment stitching are treated as equal
quantities of homogeneous labour time on the market.

The reduction of concrete labour to abstract labour happens socially, behind
the backs of individuals, through the discipline of competition and the
wage relation. If a firm uses more labour time per unit than the social
average, it loses out.

We can formalise the idea of socially necessary labour time (SNLT) for a given
commodity $i$ as an output-weighted average over firms:

\begin{equation}
  \LSN{i} \approx
  \frac{\sum_{j=1}^{n} q_{ij}\,L_{ij}}{\sum_{j=1}^{n} q_{ij}},
  \label{eq:snlt}
\end{equation}

where $L_{ij}$ is the direct and indirect labour time \emph{per unit} of commodity $i$
at firm $j$, measured at normal intensity and average skill for the period, and
$q_{ij}$ is firm $j$'s \emph{socially realised} output of commodity $i$ in the period
(i.e.\ output that is validated in exchange). Competition tends to penalise high-$L_{ij}$
producers (those using more labour time per unit than the social average) through
profit squeezes and loss of market share, shifting production toward lower-$L_{ij}$
techniques and pulling the socially necessary time toward what is required under
prevailing, normal conditions. The precise social process is more complex, but the key
point is: value depends on the \emph{average} conditions of production, not any individual
firm's whims.


If a new technology halves the labour time required while becoming the
industry norm, the value of the commodity falls—even if the selling price
only gradually adjusts.

%=========================================================
\section{Labour-power, wages, surplus labour, and surplus value}
\label{sec:labour-surplus}
%=========================================================

A central conceptual breakthrough in \emph{Capital} is the distinction between
\textbf{labour} and \textbf{labour-power}.\cite{Marx1867}

\begin{itemize}[leftmargin=1.5em]
  \item \textbf{Labour-power} is the worker's capacity to work, treated as a
  commodity. Its value is the SNLT embodied in the worker's means of
  subsistence: food, housing, clothing, education, healthcare, social
  support.
  \item \textbf{Labour} is the actual activity performed during the working
  day.
\end{itemize}

The capitalist buys labour-power for a wage, but receives labour itself. The
secret of profit lies in the difference between the value produced by labour
and the value of labour-power.

\subsection{What the worker sells: labour-power, not labour}
\label{sec:labourpower-not-labour}

In everyday speech we say workers ``sell their labour.'' Marx insists this is
misleading. What workers sell is \emph{labour-power}: the capacity to work for a
given time under the employer’s control.

That distinction is not a semantic trick; it is the logical key. If workers sold
\emph{labour itself} as a commodity, then its price (the wage) would have to equal
the value labour produces. Profit would be impossible except by cheating.

Under capitalism, the wage can be an \emph{equivalent} exchange even while exploitation
occurs, because the exchange is:
\[
  \text{wage} \longleftrightarrow \text{labour-power for } T \text{ hours},
\]
and the \emph{use-value} of labour-power is precisely that it can create new value
during those $T$ hours. The capitalist pays for the reproduction cost of that capacity
(the value of labour-power), but purchases control over its use in production for the
whole working day.

So the contract is for \emph{labour-power}, not labour itself. This distinction is the
hinge of the pamphlet: it lets us explain exploitation without assuming cheating, and it
lets us define $v$ and $s$ in a way that maps directly onto the working day.

In Marx’s shorthand, $v$ (variable capital) is the value advanced as wages to buy
labour-power. When labour-power is put to work for the full $T$ hours, it creates new
value greater than $v$. The portion of new value that merely replaces the wage is
\emph{necessary labour} (corresponding to $v$), while the remaining portion is
\emph{surplus value} $s$, the money-form of unpaid surplus labour time.

\medskip

We can now state this in the simplest temporal form: split the working day into the portion that
reproduces the wage and the portion that produces surplus.

\medskip

\subsection{Necessary and surplus labour time}

Assume the worker is paid the full value of labour-power, $v$ (no cheating). In
production, living labour adds new value; call the new value created in the day
$(v+s)$. Since the capitalist owns the product, the worker receives $v$ while the
remainder $s$ is appropriated as surplus-value. The puzzle is resolved: equal
exchange in the wage contract is compatible with exploitation because the use-value
of labour-power is value-creating labour.

Let the working day be $T$ hours. Suppose the value of the worker’s labour-power
corresponds to $\Tn$ hours of labour---the \emph{necessary} labour time. The
remaining
\begin{equation}
  \Ts = T - \Tn,
  \label{eq:Ts}
\end{equation}
is \emph{surplus labour time}, during which the worker produces value for the
capitalist without equivalent wages.

The new value created in the day is proportional to $T$, but the worker is only
paid for $\Tn$. The surplus value is proportional to $\Ts$:
\[
  v \propto \Tn,
  \qquad
  s \propto \Ts.
\]

We can therefore capture the rate of surplus value, or \emph{rate of
exploitation}, as
\begin{equation}
  e = \frac{s}{v} = \frac{\Ts}{\Tn},
  \label{eq:exploitation-time}
\end{equation}
because the proportionality factors cancel when we take the ratio.

Here, $v$ is the value of variable capital---the wages advanced---and $s$ is the
surplus value extracted.

\subsection{Why surplus necessarily appears once labour-power is commodified}
\label{sec:why-surplus}

Now we can state the logic in one chain.

Let the money-value created per hour of socially necessary labour be $\MELT$ (money/hour).
If the working day lasts $T$ hours, then the new value created in the day is:
\[
  \text{new value} = \MELT \, T.
\]
If the value of labour-power corresponds to $\Tn$ hours of socially necessary labour,
then the wage (variable capital) corresponds to:
\[
  v = \MELT \, \Tn.
\]
But the capitalist has purchased control over labour-power for the \emph{whole} day $T$,
so the surplus value is the difference:
\begin{align*}
  s
  &= \MELT \, T \;-\; \MELT \, \Tn \\
  &= \MELT \, (T - \Tn) \\
  &= \MELT \, \Ts.
\end{align*}
So $s$ is not a moral hypothesis; it is the quantitative expression of surplus labour
time, mediated through money.

This is why Marx insists on the labour / labour-power distinction: without it, you
cannot explain profit \emph{as a normal outcome of equivalent exchange}.

\subsection{Is this ``theft''? Contract, coercion, and property relations}
\label{sec:theft-or-relation}

It is tempting to describe surplus value as ``wage theft.'' Sometimes employers
\emph{do} literally steal wages (illegal non-payment). But Marx’s concept of exploitation
is deeper than that.

None of this denies \emph{illegal} wage theft (non-payment, underpayment, forced overtime without pay, etc.). Those are real and widespread. The point here is analytical: even if every boss paid exactly what the law and contract stipulate, exploitation would still occur so long as labour-power is bought as a commodity and the product is owned by capital. The core problem is not a defective contract; it is the property relation that makes the contract necessary in the first place.

In the normal case, exploitation does not require the capitalist to cheat the worker
at the point of exchange.

The wage contract can be formally voluntary and ``fair'' in the sense that the worker is paid the value of labour-power.
The point, as shown in Section~\ref{sec:labourpower-not-labour}, is that exploitation is compatible with equivalent exchange
because capital purchases control over labour-power for the whole working day, not the value the worker will create.


The coercion is \emph{structural}: workers are separated from the means of production
and must sell labour-power to live. Because capital owns the workplace, machines, land,
and inputs, it also owns the product and the surplus. That is why exploitation is a
property relation reproduced daily through ``free'' contracts.

So the Marxian claim is:
\begin{itemize}[leftmargin=1.5em]
  \item exploitation is not primarily a matter of bad individuals stealing;
  \item it is a normal outcome of capitalist production once labour-power is a commodity;
  \item therefore it cannot be abolished by moral appeals or contract reform alone,
        while the wage relation and private ownership of the main means of production remain.
\end{itemize}

This is exactly why the horizon of struggle is not only better terms inside the wage
relation, but ultimately the abolition of the wage relation itself.


\subsection{Example: a coffee-shop worker}

Consider a barista in a large coffee chain in a global North city.

\begin{example}[A coffee-shop worker]
Suppose:

\begin{itemize}[leftmargin=1.5em]
  \item The shop sells drinks and snacks worth \$800 during an 8-hour shift.
  \item Of this, \$300 covers the value of used-up constant capital (beans, milk,
        cups, energy, machine depreciation) --- this is $\constcap$.
  \item The barista's wage bill for the shift is \$100 --- this is $\varcap$.
\end{itemize}

We want to find:

\begin{itemize}[leftmargin=1.5em]
  \item the new value created by living labour during the shift ($\varcap + \surplus$),
  \item the surplus value $\surplus$,
  \item the rate of exploitation $\exploitrate$, and
  \item how many hours of the 8-hour day are necessary labour time and how many
        are surplus labour time.
\end{itemize}

\subsubsection*{Step 1: Separate the value created by labour from the value of used-up machinery and inputs}

The total value of the day's output is \$800.  Of this, \$300 is just the value of
used-up constant capital $\constcap$ being transferred to the finished products.
The remainder must therefore be new value created by living labour:
\begin{align*}
  \text{Total value of output} &= \$800,\\
  \text{Value of used-up constant capital } (\constcap) &= \$300,\\[0.5em]
  \Rightarrow\quad
  \text{New value created by labour } (\varcap + \surplus)
    &= \text{Total value} - \text{Constant capital}\\
    &= \$800 - \$300\\
    &= \$500.
\end{align*}
So the labour of the barista has added \$500 of new value during the shift:
\[
  \varcap + \surplus = \$500.
\]

\subsubsection*{Step 2: Split this new value into wages and surplus-value}

Out of this \$500 of new value, the capitalist pays the barista \$100 as wages.
This \$100 is the value of variable capital $\varcap$:
\[
  \varcap = \$100.
\]
The rest of the new value is surplus-value $\surplus$:
\begin{align*}
  \surplus
    &= (\varcap + \surplus) - \varcap\\
    &= \$500 - \$100\\
    &= \$400.
\end{align*}
So
\[
  \surplus = \$400.
\]

\subsubsection*{Step 3: Compute the rate of exploitation}

The \emph{rate of exploitation} (or \emph{rate of surplus-value}) is defined as
the ratio of surplus-value to variable capital:
\[
  \exploitrate = \frac{\surplus}{\varcap}.
\]
Plugging in our numbers:
\begin{align*}
  \exploitrate
    &= \frac{\surplus}{\varcap}\\[0.3em]
    &= \frac{\$400}{\$100}\\[0.3em]
    &= 4.
\end{align*}
As a percentage:
\[
  \exploitrate = 4 \times 100\% = 400\%.
\]
This means that, measured in value terms, the barista produces four times as
much surplus for the owner as they receive back in wages.

\subsubsection*{Step 4: Translate the value relations into hours of the working day}

The barista works an 8-hour shift.  We know that, in value terms,
\[
  \varcap : \surplus = 100 : 400 = 1 : 4.
\]
So, of the \emph{new} value (\$500) created in the day, one-fifth belongs to
necessary labour (reproducing the wage) and four-fifths is surplus labour.

Necessary labour time is therefore one-fifth of the working day:
\begin{align*}
  \text{Necessary labour time}
    &= \frac{\varcap}{\varcap + \surplus} \times 8\ \text{hours}\\[0.3em]
    &= \frac{100}{500} \times 8\\[0.3em]
    &= 0.2 \times 8\\[0.3em]
    &= 1.6\ \text{hours}.
\end{align*}
The remaining part of the working day is surplus labour time:
\begin{align*}
  \text{Surplus labour time}
    &= 8\ \text{hours} - 1.6\ \text{hours}\\[0.3em]
    &= 6.4\ \text{hours}.
\end{align*}

So in this simple coffee-shop example:

\begin{itemize}[leftmargin=1.5em]
  \item In the first $1.6$ hours of the shift, the barista produces value
        equivalent to their own wage (necessary labour time).
  \item In the remaining $6.4$ hours, they produce surplus-value for the owner
        (surplus labour time).
  \item Over the whole day, the rate of exploitation is $\exploitrate = 400\%$:
        for every \$1 paid in wages, \$4 of surplus-value is extracted.
\end{itemize}

This does \emph{not} mean that the branch manager personally pockets
\$400. The surplus value is divided, through complex mechanisms, into
different forms of revenue: profit of enterprise, interest on loans,
rent to the landlord, franchise fees to the global brand, and taxes to
the state. Marx’s key point is that all of these streams ultimately
originate in unpaid surplus labour time like that of this barista.

\end{example}

\subsection{Forms of wages: time-wages and piece-wages}
\label{sec:wage-forms}

Marx treats wages not only as a magnitude ($v$) but also as a \emph{form} that can
conceal exploitation.\cite{Marx1867} Two classic forms are time-wages and piece-wages.

\subsubsection*{Time-wages}

Under \textbf{time-wages}, the worker is paid per unit of time (hour/day/week/month).
Let the money wage per hour be $w_h$ and the paid hours in the period be $T$.
Then the \emph{money wage bill} is
\[
  W_w = w_h \, T.
\]

Time-wages make it appear as if the worker is paid for \emph{all} labour performed in
those hours. But in Marxian terms, the wage is the price of \emph{labour-power}, not the
price of \emph{labour}. What is purchased is the worker's capacity to labour for $T$
hours under capital's control. The wage corresponds (in value terms) to $v$, the value of
labour-power, i.e.\ the socially necessary labour time required to reproduce the worker.

To connect money-wages to labour-time, introduce the monetary expression of labour time
(MELT), $\MELT$ (money per hour of socially necessary labour time). Then the value of
labour-power corresponds to a quantity of necessary labour time $\Tn$, and (schematically)
\[
  v = \MELT \, \Tn,
  \qquad
  W_w \approx v.
\]

The approximation sign here flags that money-wages and market prices do not, in general,
equal values. Section~\ref{sec:prices-production} explains how competition and the tendency
toward an average rate of profit generate prices of production that systematically diverge
from values; market power, state policy, and world-market (exchange-rate and terms-of-trade)
conditions can produce further, conjunctural deviations. The key point is conceptual: even
if $W_w$ equals the value of labour-power (no cheating), exploitation can still occur because
the use-value of labour-power is value-creating labour for a longer working day than $\Tn$.


\subsubsection*{Piece-wages}

Under \textbf{piece-wages}, the worker is paid per unit output. Let output per hour be
$q$ units/hour and the piece rate be $w_p$ money/unit. Then hourly pay is:
\[
  w_h = w_p \, q
  \qquad\Rightarrow\qquad
  w_p = \frac{w_h}{q}.
\]
Marx’s key point is that piece-wages are not a different substance of payment: they are
\emph{time-wages in disguise}. The piece-rate is derived from an assumed \emph{normal} output
per unit time (a socially enforced ``standard'' pace of work), and the worker is then pushed
to meet or exceed that norm. Piece-wages therefore do not abolish time-wages; they are typically
a \emph{modified} time-wage in which the wage is tied more directly to labour intensity and output.
In practice, this form often sharpens managerial control: it individualises pay, intensifies labour,
and can shift risks of stoppages, defects, and speed-up onto workers, while leaving the basic relation
unchanged---labour-power is still bought for a period, and surplus labour can still be extracted within
(or by extending) the working day.


\subsubsection*{Why this matters}

Piece-wages tend to:
\begin{itemize}[leftmargin=1.5em]
  \item intensify labour (workers self-discipline to increase $q$);
  \item shift monitoring costs from management to workers (``the numbers'' police you);
  \item increase competition among workers and weaken collective pacing and breaks.
\end{itemize}

But the underlying relation remains: the worker produces new value during the whole
working day, while the wage corresponds to the value of labour-power. Wage-forms alter
how exploitation is \emph{experienced}, enforced, and fought over; they do not abolish
the exploitation relation itself.

\subsection{Example: smartphone assembly}

\paragraph{Note on values and prices.}
For simplicity in these numerical examples, treat observed money prices and costs as
proportional to underlying values. In reality, prices can systematically deviate from
values through competition and the redistribution of surplus value via prices of
production; see Section~\ref{sec:prices-production}. Market power, state policy, and
world-market conditions can generate further deviations around these regulating prices.


Return to the smartphone example. Suppose a batch of phones sells for
\$1.4 million. The costs are:

\begin{itemize}[leftmargin=1.5em]
  \item $\constcap = \$1{,}000{,}000$ in components, depreciation, logistics.
  \item $\varcap = \$200{,}000$ in direct wages at the assembly plant.
\end{itemize}

If the phones sell for \$1.4 million, the surplus value \emph{realised} on this batch is
\begin{equation}
  \surplus = \$1{,}400{,}000 - (\$1{,}000{,}000 + \$200{,}000) = \$200{,}000.
\end{equation}

The exploitation rate for the assembly labour (in this simplified example) is
\begin{equation}
  \exploitrate = \frac{\surplus}{\varcap}
  = \frac{\$200{,}000}{\$200{,}000} = 1 \quad\Rightarrow\quad \exploitrate = 100\%.
\end{equation}

Two cautions matter for Marxian precision.

First, this arithmetic treats money magnitudes as proportional to values for clarity.
In reality, competition, productivity differentials, and market power generate systematic
value--price divergences (see Section~\ref{sec:prices-production}); we flag this here but
abstract from those divergences in the worked numbers.

Second, surplus value is produced where \emph{productive labour} is set to work in production
processes that create commodities (or commodity-services) for sale. Much of what happens
``downstream'' (advertising, branding, some retail functions, finance) is better analysed as
the \emph{realisation} and \emph{redistribution} of surplus value rather than its direct
production. The brand and its shareholders can nonetheless capture surplus generated across
the global chain through control of intellectual property, market access, finance, logistics,
and pricing power (including transfer pricing and monopoly rents).\cite{Smith2016,Hickel2022}

%=========================================================
\section{Constant and variable capital; circuits of capital}
\label{sec:constant-variable}
%=========================================================

To move from exploitation at the point of production to the dynamics of the
system as a whole, we need to distinguish types of capital.\cite{Marx1867}

\subsection{Constant and variable capital}

\begin{itemize}[leftmargin=1.5em]
  \item \textbf{Constant capital} ($\constcap$) is the value of the means of
  production: machinery, buildings, raw materials, intermediate goods.
  Its value is \emph{transferred} to the product but does not itself create
  new value.
  \item \textbf{Variable capital} ($\varcap$) is the value advanced to buy
  labour-power (wages, broadly understood). It is called ``variable'' because
  it is the part of capital that can create more value than it costs.
\end{itemize}

If we consider one turnover of capital, the \textbf{value} of the total
commodity product is:
\begin{equation}
  W = \constcap + \varcap + \surplus.
  \label{eq:total-value}
\end{equation}

The \textbf{cost price} for the capitalist is:
\begin{equation}
  k = \constcap + \varcap,
\end{equation}
while the surplus value $\surplus$ is the excess of value over cost.

\paragraph{Why ``selling above market price'' cannot explain profit.}
A common intuition is that profit comes from \emph{selling dear}: buy for $100$, sell for $120$.
But as a general explanation of profit for the whole capitalist class, this cannot work.

\textbf{Example (1): pure price mark-up is redistribution, not creation.}
Suppose Merchant A buys a commodity for \$100 and sells it to Merchant B for \$120.
A gains \$20, but B has paid \$20 more than the prevailing market price. Across the two
merchants taken together, no new value is created \emph{by this exchange}; \$20 has simply been
transferred from B to A. If \emph{everyone} tried to ``sell above market price'' at the same time,
the ``market price'' would just re-set upward. The question would return in the same form:
where does the extra money come from \emph{in the aggregate}?

\medskip

\textbf{Example (2): profit without any mark-up.}
Now consider a capitalist who advances $\constcap=80$ for inputs and depreciation, and $\varcap=20$ in wages.
Living labour produces new value $\varcap+\surplus = 40$, so the commodity product has value
\[
  W = \constcap + \varcap + \surplus = 80 + 20 + 20 = 120.
\]
If the capitalist sells the output \emph{at its value} \$120 (no overcharging at all), the cost price is
$k=\constcap+\varcap=100$, so profit is
\[
  W-k = 120-100=20,
\]
which is exactly the surplus value $\surplus$.

\medskip

\noindent
Selling above (or below) value can redistribute surplus value between capitals (as shown in
Section~\ref{sec:prices-production}), especially under monopoly power. But the system-wide source of
profit is surplus value extracted from labour-power. Profit appears here even though the commodity is
sold at its value, because the source is the surplus value $s$ created in production.

\medskip
With this cleared up, we can return to the movement of capital itself: how money is advanced, turned into means of production and labour-power, and then returns expanded as $M'$. This is what Marx captures in the basic circuits of circulation.
\medskip


\subsection{Circuits of capital}
\label{subsec:circuits}

Marx contrasts two circuits:

\begin{align}
  \CMC &: C \rightarrow M \rightarrow C,
  \label{eq:cmc}\\
  \MCM &: M \rightarrow C \rightarrow M'.
  \label{eq:mcm}
\end{align}

\begin{itemize}[leftmargin=1.5em]
  \item In $C \rightarrow M \rightarrow C$, a small producer sells a commodity $C$ for money $M$ to buy another
  commodity $C$ (e.g.\ a farmer selling grain to buy tools and food). The goal is use-value.

  \item The same overall form also describes the \emph{worker's} circuit, with an important twist. The worker
  does not usually sell a finished commodity they own; they sell their \textbf{labour-power} (the capacity to
  work for a period) for a money wage $M$. That wage is then spent to buy commodities $C$ needed to live and to
  return to work tomorrow---food, transport, rent, childcare, medicine, and so on. In that sense the wage is not
  payment for \emph{all} the value the worker creates; it is the money form of the value of labour-power, i.e.\
  the cost of reproducing the worker as a worker. This is why questions of households, care, and unpaid work
  (social reproduction) are not ``outside'' political economy: they are part of the conditions of possibility of
  wage-labour itself.

    \item This is also why money is not a neutral ``medium'' hovering above society. In capitalism it
  becomes the everyday form of constraint: you do not access housing, food, transport, care, or time
  directly; you access them \emph{through wages}. Nature does not demand coins for land, water, or
  energy. The demand comes from property and enforcement: social structures that monopolise conditions
  of life and charge for entry. Money is therefore not merely a tool of exchange; it is the social form
  in which private control over social wealth confronts the majority as a daily requirement.


  \item In $M \rightarrow C \rightarrow M'$, a capitalist advances money $M$ to buy labour-power and means of
  production $C$, producing commodities that are sold for $M' = M + \Delta M$. The goal is the increment
  $\Delta M$, i.e.\ surplus value realised in money-form as profit.
\end{itemize}

Capital is therefore best understood not as a thing (a machine) but as a social relation and a circuit:
\textbf{value in motion that valorises itself through the exploitation of labour, and that necessarily appears
and is realised in monetary form as $M \rightarrow M'$.}


%=========================================================
\section{Rate of profit and organic composition of capital}
\label{sec:rate-profit}
%=========================================================

The \textbf{rate of profit} is the central ratio from the capitalist point
of view. It measures the surplus value obtained per unit of total capital
advanced:

\begin{equation}
  r = \frac{s}{c + v}.
  \label{eq:profit-rate-basic}
\end{equation}


From the definition of the exploitation rate in Eq.~\eqref{eq:exploitation-time},
\(e = s/v\), we have \(s = e v\), so we can rewrite the rate of profit as
\begin{equation}
  r = \frac{e v}{c + v}
    = \frac{e}{1 + c/v}
    = \frac{e}{1 + \organiccomp},
  \label{eq:profit-rate-occ}
\end{equation}

where \(\organiccomp = c/v\) is the organic composition of capital.



\subsection{Illustration: varying the organic composition}

Suppose, for simplicity, that the exploitation rate is fixed at
$\exploitrate = 100\%$, i.e.\ $\surplus = \varcap$, and set $\varcap=100$
as a numéraire. We examine what happens to $\profitrate$ as the ratio
$\constcap/\varcap$ rises.

\begin{center}
\begin{tabular}{@{}lrrr@{}}
\toprule
$c/v$ & $c$ & $s$ & $r = \dfrac{s}{c+v}$ \\
\midrule
$0.5$ & $50$  & $100$ & $\approx 66.7\%$ \\
$1.0$ & $100$ & $100$ & $50.0\%$        \\
$2.0$ & $200$ & $100$ & $\approx 33.3\%$ \\
$4.0$ & $400$ & $100$ & $20.0\%$        \\
$8.0$ & $800$ & $100$ & $\approx 11.1\%$ \\
\bottomrule
\end{tabular}
\end{center}

Even with a constant rate of exploitation $e = s/v = 100\%$, the rate of profit $r$ falls as capital becomes more “machine-heavy” relative to wages (as the ratio $c/v$ rises). This logical connection between a rising organic composition of capital $c/v$ and a falling profit rate $r$ is at the heart of Marx’s \emph{law of the tendency of the rate of profit to fall} (TRPF).\cite{Roberts2015,Shaikh2016}

\subsection{Competition, average profit, and prices of production}
\label{sec:prices-production}

So far we have treated $\surplus$ and $\profitrate$ at the level of production:
surplus value is produced where living labour is set to work, and the profit rate
is $\profitrate = \surplus/(\constcap+\varcap)$ (Eq.~\eqref{eq:profit-rate-basic}).
But on the surface of capitalist society, capitals do not simply pocket the surplus
they individually produce. They compete, they move, and they compare returns.
This movement produces a \emph{tendency} toward an \emph{average} rate of profit.

To formalise this, index industries (or branches of production) by $i$.
Let the capital advanced in industry $i$ be $(\constcap_i + \varcap_i)$.
Assume, for clarity, that the \emph{rate of exploitation} is uniform across industries,
so that in each industry
\[
  \surplus_i = \exploitrate\,\varcap_i.
\]
Then the \textbf{value} (in money terms) of the industry's output (at this level of abstraction)
can be written as
\begin{equation}
  w_i = \constcap_i + \varcap_i + \surplus_i
      = \constcap_i + (1+\exploitrate)\,\varcap_i.
  \label{eq:value-output-industry}
\end{equation}

Define the \textbf{cost price} (capital advanced that must be replaced) as
\begin{equation}
  k_i = \constcap_i + \varcap_i.
  \label{eq:cost-price}
\end{equation}

Total surplus value and total capital advanced are
\[
  \surplus = \sum_i \surplus_i,
  \qquad
  K = \sum_i (\constcap_i + \varcap_i) = \sum_i k_i.
\]
Competition tends to equalise profit rates, so the \textbf{average rate of profit} is
\begin{equation}
  \profitrate = \frac{\surplus}{K}.
  \label{eq:average-profit-rate}
\end{equation}

Under this tendency, the regulating price is not the direct value $w_i$ but the
\textbf{price of production}, which yields the average profit on the capital advanced:
\begin{equation}
  p_i = k_i(1+\profitrate) = (\constcap_i + \varcap_i)(1+\profitrate).
  \label{eq:price-of-production}
\end{equation}
The profit \emph{received} by industry $i$ is therefore
\[
  \pi_i = p_i - k_i = \profitrate\,k_i = \profitrate(\constcap_i+\varcap_i).
\]
Compare this with surplus value \emph{produced} in that industry, $\surplus_i=\exploitrate\varcap_i$.
The difference is
\[
  \pi_i - \surplus_i = \profitrate(\constcap_i+\varcap_i) - \exploitrate\varcap_i,
\]
which expresses the redistribution of surplus value across industries through competition.

It is useful to make the dependence on the \emph{organic composition} explicit.
Let $C=\sum_i \constcap_i$ and $V=\sum_i \varcap_i$. With $\surplus=\exploitrate V$ we have
\[
  \profitrate = \frac{\surplus}{C+V} = \frac{\exploitrate V}{C+V}
  = \frac{\exploitrate}{1 + C/V}.
\]
Then, using $\Omega = C/V$ as the \textbf{average} organic composition, one can show
\begin{equation}
  p_i - w_i
  = \exploitrate\,\varcap_i \,\frac{\big(\constcap_i/\varcap_i\big) - \Omega}{1+\Omega}.
  \label{eq:pop-deviation}
\end{equation}
So industries with \emph{above-average} $\constcap_i/\varcap_i$ tend to have $p_i>w_i$
(they receive more profit than the surplus they themselves produce), while those with
\emph{below-average} $\constcap_i/\varcap_i$ tend to have $p_i<w_i$.

\subsubsection*{A two-industry illustration (values vs.\ prices of production)}

\begin{example}[Two industries and the redistribution of surplus value]
Assume two industries, $A$ and $B$, with a common exploitation rate $\exploitrate = 100\%$,
so $\surplus_i = \varcap_i$. Let $\varcap_A=\varcap_B=100$, and let $A$ be lower-composition
and $B$ higher-composition:
\[
  \constcap_A=50,\qquad \constcap_B=150.
\]

\paragraph{Step 1: Surplus value produced and values.}
Since $\exploitrate=100\%$, we have
\[
  \surplus_A=\varcap_A=100,\qquad \surplus_B=\varcap_B=100.
\]
So values (Eq.~\eqref{eq:value-output-industry}) are:
\[
  w_A=\constcap_A+\varcap_A+\surplus_A=50+100+100=250,\qquad
  w_B=\constcap_B+\varcap_B+\surplus_B=150+100+100=350.
\]

\paragraph{Step 2: The average profit rate.}
Totals are $C=200$, $V=200$, $\surplus=200$, so the average profit rate is
\[
  \profitrate=\frac{\surplus}{C+V}=\frac{200}{400}=50\%.
\]

\paragraph{Step 3: Prices of production and profits received.}
Prices of production (Eq.~\eqref{eq:price-of-production}) are:
\[
  p_A=(\constcap_A+\varcap_A)(1+\profitrate)=(50+100)(1.5)=225,\qquad
  p_B=(\constcap_B+\varcap_B)(1+\profitrate)=(150+100)(1.5)=375.
\]
Profits received are $\pi_i = p_i-(\constcap_i+\varcap_i)=\profitrate(\constcap_i+\varcap_i)$, hence
\[
  \pi_A=0.5\times 150=75,\qquad \pi_B=0.5\times 250=125.
\]

\paragraph{Step 4: Who transfers surplus value, and who receives it?}
Industry $A$ sells \emph{below} its value ($225<250$) and receives less profit than the surplus it produces:
\[
  \pi_A-\surplus_A = 75-100 = -25 \quad\Rightarrow\quad \text{$A$ transfers 25.}
\]
Industry $B$ sells \emph{above} its value ($375>350$) and receives more profit than the surplus it produces:
\[
  \pi_B-\surplus_B = 125-100 = +25 \quad\Rightarrow\quad \text{$B$ receives 25.}
\]

\medskip
\begingroup
\setlength{\tabcolsep}{6pt}
\renewcommand{\arraystretch}{1.15}
\begin{center}
\begin{tabular}{@{}lrrrrrrr@{}}
\toprule
Industry & $c$ & $v$ & $s$ (produced) & $w=c+v+s$ & $k=c+v$ & $p=k(1+\profitrate)$ & $\pi=\profitrate k$ \\
\midrule
$A$ & 50  & 100 & 100 & 250 & 150 & 225 & 75  \\
$B$ & 150 & 100 & 100 & 350 & 250 & 375 & 125 \\
\midrule
Total & 200 & 200 & 200 & 600 & 400 & 600 & 200 \\
\bottomrule
\end{tabular}
\end{center}
\endgroup

\paragraph{Aggregate identities (the punchline).}
Under these simplifying assumptions,
\[
  \sum_i p_i = \sum_i w_i,
  \qquad
  \sum_i \pi_i = \sum_i \surplus_i.
\]
So prices of production redistribute surplus value; they do not create it. Market prices can
fluctuate around $p_i$ due to demand, supply, monopoly power, state policy, and world-market
conditions, but the competitive tendency toward average profit is the key mediating mechanism
through which labour-time asserts itself in the long run.\cite{Marx1894,Shaikh2016}
\end{example}

\subsection{Modern intuition}

Think of the evolution of car production. A hundred years ago, producing a
car required many hours of relatively low-mechanisation labour. Today, large
parts of the process are automated: robots weld, paint, and assemble; software
optimises flows. The ratio $\constcap/\varcap$ has risen massively.

At the level of an individual firm, this may boost profits for a while,
because the firm gets ahead of the social average and sells at prices that
embody more labour time than it actually expends. But as the new technology
diffuses, the industry-wide \snlt{} falls, values fall, and competitive
pressures reassert themselves. At a given or even rising exploitation rate,
the long-run tendency is for \profitrate{} to come under pressure.

%=========================================================
\section{The tendency of the rate of profit to fall and its countertendencies}
\label{sec:trpf}
%=========================================================

Marx formulates the TRPF as a \emph{tendency} of capitalist accumulation:\cite{Marx1894,Roberts2015}

Marx argues that, \emph{other things equal}, accumulation tends to raise the organic composition of capital:
more machinery, technology, and materials are set in motion per worker. Since only living labour produces
new value, a rising $c$ relative to $v$ tends, over long periods, to press down the rate of profit, unless
counter-tendencies (higher exploitation, cheapening of elements of constant capital, foreign trade, etc.)
offset it.

\begin{quote}
\emph{``[T]he gradual growth of constant capital in relation to variable capital must necessarily lead to a
gradual fall of the general rate of profit \dots''}\cite{Marx1894}
\end{quote}


The mathematics is compact in Eq.~\eqref{eq:profit-rate-occ}: if \OCC{} rises while $e$ is constant, then the rate of profit $r$ must fall. But Marx immediately adds that this tendency is modified by counteracting influences.

\subsection{Counteracting tendencies}

Some of the main countertendencies are:

\begin{enumerate}[leftmargin=1.5em]
  \item Raising the exploitation rate $\exploitrate$ by lengthening the
  working day, intensifying labour, or cutting the value of labour-power
  (e.g.\ through wage repression, union-busting, outsourcing, or attacks on
  social reproduction).\cite{HarveyCompanion}

  \item Cheapening elements of constant capital, so that $\constcap$ grows
  more slowly than productivity, for example through technological progress
  in the production of machines and raw materials.

  \item Depressing wages below the value of labour-power, especially in the
  global South, migrant labour regimes, and informal sectors.

  \item Expanding to new markets, new branches of production, and new forms
  of commodification (land, housing, health, education, data, platforms).

  \item Destroying capital in crises (bankruptcies, write-offs, wars), which
  reduces the denominator $\constcap + \varcap$ and can temporarily restore
  profitability for surviving capitals.
\end{enumerate}

These factors explain why capitalism can go through long booms despite the
underlying tendency. They do not abolish the law; they generate cycles and
waves.

\subsection{Empirical patterns}

Empirical work by Marxist economists suggests that, across many advanced
capitalist economies, profitability tends to fall over long periods, interrupted
by partial recoveries, and that major crises are often preceded by profit
squeezes.\cite{Roberts2015,Shaikh2016}

The details are debated, but the broader point is clear: capitalist firms
are structurally driven to raise \OCC{} via automation and capital intensity,
and this makes profitability increasingly fragile. When combined with high
debt levels and financial speculation, the outcome is recurrent crises.

%=========================================================
\section{Crises, cycles, and unemployment}
\label{sec:crises}
%=========================================================

Mainstream economics often treats crises as the result of ``shocks'': a bad
policy, a pandemic, an oil price spike. Marxian political economy instead
sees crises as \textbf{endogenous} to the system.

\subsection{Overaccumulation and overproduction}

If profits fall in the productive sectors, capitalists may cut investment.
Yet credit and fictitious capital---stocks, bonds, derivatives, property
titles---can keep expanding claims on future surplus value even when the
underlying surplus is under strain.\cite{HarveyNewImperialism}

At the same time, the drive to raise \exploitrate{} and cut costs means that
workers' wages grow slowly relative to productivity. The result is a chronic
tendency towards \textbf{overproduction}: more goods and services than can
be profitably sold at existing wages and power relations.

\subsection{The boom–bust rhythm}

A highly simplified stylised cycle looks like this:

\begin{enumerate}[leftmargin=1.5em]
  \item \textbf{Expansion}: Profits are high; firms invest; credit expands;
  employment rises.

  \item \textbf{Overaccumulation}: Competition and rising \OCC{} squeeze
  profitability; inventories build up; debt burdens grow.

  \item \textbf{Crisis}: Investment collapses; firms fail; banks curtail
  lending; unemployment rises; states respond with bailouts and austerity.

  \item \textbf{Devaluation and restructuring}: Capital is destroyed or
  devalued; wages and working conditions are attacked; new technologies and
  organisational forms are introduced; profitability is partially restored.

  \item \textbf{Renewed expansion}---until the next wave.
\end{enumerate}

The Great Depression, the post-war boom, the profitability crisis of the
1970s, the global crisis of 2007–08, and the uneven post-COVID conjuncture
can all be read through this lens.\cite{Roberts2015,HarveyCompanion}

%=========================================================
\section{Policy fixes, reformism, and the capitalist veto}
\label{sec:policy-fixes-veto}
%=========================================================

When crises intensify, politics usually floods with proposals that promise to stabilise capitalism without
touching ownership: raise the minimum wage, tax the rich, give everyone cash (UBI), print money (QE/MMT),
cancel debts, regulate monopolies, or build a ``green'' capitalism through carbon markets and climate finance.
Many of these measures can \emph{matter} for living conditions, sometimes a great deal. The Marxian question is
more specific: \emph{what kinds of contradictions can such measures actually resolve, and which ones do they
inevitably reproduce?}

To answer, we need one distinction.

\paragraph{Reforms versus reformism.}
A \textbf{reform} is a concrete gain won in struggle: higher wages, shorter hours, cheaper essentials, housing,
healthcare, democratic rights. Communists fight for such gains. \textbf{Reformism} is the belief that parliamentary
measures, taken as administrative policy inside unchanged property relations, can overcome capitalism's core laws
of motion: exploitation, profit-governed investment, overaccumulation, and crisis. The difference is strategic.
Reforms can improve life and build capacity. Reformism treats reforms as a substitute for confronting ownership
and class power.

\paragraph{The capitalist veto (the structural core).}
Under capitalism, the majority do not control production. Capital owns (or commands) workplaces, supply chains,
credit, land, and key infrastructures. Investment is not a neutral technical decision; it is governed by expected
profitability and risk. That means reforms that threaten profitability, discipline, or property are met not only
by debate in parliament but by a set of \emph{structural responses} available to capital: layoffs, closures,
informalisation, subcontracting, relocation, price hikes where market power allows, capital flight into dollars or
property, accounting tricks, lobbying, court action, and media and state sabotage. This is the ``capital strike''
dynamic: when private owners control investment, they can withhold it.

This is not a moral claim about individual bosses. It is a claim about a system in which production is privately
controlled and social reproduction is forced to pass through money and markets. With that clarity, we can assess
the most common policy fixes.

\begin{quote}
\small
\textbf{``But where would the money come from?'' Money versus provisioning.}%

A recurring objection to universal services, public housing, free transit, or healthcare is:
``Where would the money come from?'' Marxian political economy answers by separating
\emph{money as an accounting and command device} from \emph{the real conditions of provisioning}.
The material question is not whether society can print tokens, but whether society can organise
labour, land, energy, machinery, and logistics to produce and distribute use-values.

Under capitalism, money appears as the gatekeeper to food, homes, medicine, and education because
access is controlled by property, prices, and profitability. Nature does not demand payment for
land, water, or sunlight. The demand comes from social power: private ownership, state enforcement,
and corporate control, which turn conditions of life into commodities. Once essentials are treated as
social wealth---planned and provided as rights---``funding'' stops being a mystical problem and becomes
a practical one: how much capacity exists, what inputs and labour are required, what timelines are
feasible, and which priorities society sets under democratic control.
\end{quote}


%---------------------------------------------------------
\subsection{Minimum wage: why a good law meets a structural constraint}
\label{sec:policy-minimum-wage}
%---------------------------------------------------------

Consider a large legal increase in the minimum wage, for example a guaranteed living wage. On paper this is a gain.
But the law immediately collides with the relation developed earlier: labour-power is purchased as \emph{variable
capital} $v$, and surplus value $s$ is extracted in production. If the wage floor rises, and if output prices and
productivity are unchanged, then $v$ rises relative to $s$, and the rate of profit
\[
  r=\frac{s}{c+v}
\]
is pushed downward. Capital therefore tries to restore profitability through mechanisms it controls:

\begin{itemize}[leftmargin=1.5em]
  \item \textbf{Evasion and informalisation:} shifting workers off contracts, underreporting hours, using
  subcontractors, replacing formal jobs with informal or gig arrangements.
  \item \textbf{Layoffs and closures:} threatening job cuts or shutdowns, especially in low-margin sectors or
  where employers can easily switch locations or lines of business.
  \item \textbf{Speed-up and intensification:} raising the pace of work, cutting breaks, increasing output norms
  so that each paid hour yields more surplus in practice (a rise in the realised exploitation rate).
  \item \textbf{Price pass-through where possible:} raising prices in sectors with market power, turning part of
  the wage gain into higher living costs, especially when essentials are privately priced.
  \item \textbf{Recomposition of investment:} moving into speculative outlets (property, currency arbitrage,
  rent-seeking) if productive profitability is weak or contested.
\end{itemize}

None of this implies that minimum-wage struggles are pointless. It implies that a wage reform, by itself, does not
command production. It has to be paired with enforcement, organisation, and measures that limit capital's ability to
evade, relocate, or recapture the gain through prices and intensified labour. The stronger the labour movement and
the more socialised the provision of essentials (housing, transport, healthcare, energy), the more durable a wage gain
can become.

%---------------------------------------------------------
\subsection{Progressive taxation: redistribution meets capital mobility and state dependence}
\label{sec:policy-taxation}
%---------------------------------------------------------

Progressive taxation is another common solution: tax profits, wealth, and high incomes to fund social spending. From a
Marxian perspective, the first point is simple: taxes do not abolish exploitation; they redistribute claims on social
output \emph{after} production and appropriation have already occurred. That redistribution can still be socially vital.
But its limits are structural.

First, capital has multiple routes of resistance: avoidance and evasion, hiding assets, shifting profits, moving money
offshore, and reclassifying income. Where the tax state is weak, and where large sectors are informal, taxation tends to
fall back onto indirect taxes and wage earners while wealth escapes. Second, the capitalist state itself is structurally
pressured to maintain ``investment'' and ``confidence'': it needs employment, revenues, foreign exchange, and credit
access. Without controls on capital movement and strong enforcement capacity, progressive taxation becomes a terrain of
class struggle, not a purely administrative adjustment.

So the Marxian point is not ``taxation never helps.'' It is: taxation that seriously threatens capitalist income and
property tends to trigger a confrontation with capital's structural power. If that confrontation is not met with organised
power and controls over finance and investment, redistribution is diluted, reversed, or captured.

%---------------------------------------------------------
\subsection{UBI: cash without decommodification}
\label{sec:policy-ubi}

%---------------------------------------------------------

A universal basic income (UBI) proposes unconditional cash payments. In principle, it can reduce poverty and provide
a survival floor. The Marxian question is: does a cash grant decommodify life, or does it become another money stream
circulating inside the same commodity economy?

If essentials remain commodified and privately priced, a cash grant can be partially captured by landlords, private
utilities, and monopolised suppliers through higher rents and administered prices. In that case UBI risks functioning
as an indirect subsidy to low-wage capital and rentiers: wages can be held down while the state tops up survival in cash.
UBI also faces a funding constraint that immediately becomes political: it must be financed by taxing capital/wealth,
cutting other spending, borrowing (state debt), or monetary expansion. Each route collides with class power under
capitalism: capital resists taxation, borrowing shifts today's conflict into a claim on \emph{future} surplus, and money
creation in an import-dependent, price-sensitive economy can return as inflation or currency pressure.

The most robust version of ``basic income'' therefore tends to be not cash alone but \textbf{universal basic services}:
socialised housing, transport, energy, healthcare, education, and care. These are not simply moral add-ons; they reduce
capital's ability to recapture wage gains and cash transfers through prices, and they shift part of social reproduction
out of the market.

%---------------------------------------------------------
\subsection{Quantitative easing and ``printing money'': expanding claims versus producing value}
\label{sec:policy-qe}
%---------------------------------------------------------

Quantitative easing (QE) and related ``print money'' proposals are often framed as a technical solution: if the state can
create money, why not fund jobs, services, and investment directly? Marxian political economy begins by separating two
things: \emph{money as a claim} and \emph{value as socially necessary labour time} expressed in money form.

QE is best understood as a state-led intervention that supports financial balance sheets by purchasing assets and
creating liquidity. It can lower interest rates and stabilise banks. But it does not, by itself, set additional labour
to work in productive processes that generate surplus value. In Marxian terms it supports and revalues \emph{titles} to
future income, i.e.\ fictitious capital, without guaranteeing an expansion of surplus value actually produced. When
profitability is weak, QE therefore commonly shows up as asset-price inflation and the preservation of financial wealth,
rather than as a transformation of everyday life.

More generally, monetary expansion can relax constraints, but it cannot abolish the basic contradiction that the mass of
profits and interest-bearing claims must be validated by production and realisation. Where supply is constrained, imports
are decisive, or oligopolies dominate essentials, money injections can reappear as price inflation and renewed class
discipline. The question is not whether money can be created, but who controls production, pricing, and allocation.

%---------------------------------------------------------
\subsection{Job guarantees and Keynesian stimulus: demand management without investment control}
\label{sec:policy-job-guarantee}
%---------------------------------------------------------

A Job Guarantee (JG) or large fiscal stimulus is often posed as an alternative to UBI: instead of paying people to exist,
the state directly employs them and stabilises demand. Such measures can reduce unemployment and build useful infrastructure.
But they still confront capitalist investment control. If the private sector retains command over major production, credit,
and trade, then sustained full employment and rising wage power can provoke inflation politics, investment strikes, and
pressure for austerity and ``discipline.'' In other words, demand management can moderate the cycle, but it cannot remove
the profitability and overaccumulation pressures that repeatedly drive capitalist crises unless it is paired with democratic
control over investment and strategic sectors.

This is a recurring pattern: the state can absorb some unemployment and fund projects, but if capital remains free to
withhold investment, relocate, or raise prices, the system fights back against reforms that shift power to labour.

%---------------------------------------------------------
\subsection{Debt relief, jubilees, and bailouts: revaluing claims without transforming production}
\label{sec:policy-debt-relief}
%---------------------------------------------------------

Debt relief can be necessary. In a debt-driven economy, households, firms, and states can become locked into servicing
claims that are no longer compatible with living standards and productive capacity. From a Marxian standpoint, debt is a
structured claim on future surplus. Writing down debts therefore revalues claims on future surplus---it is a moment of
devaluation and restructuring. The class question is: \emph{whose claims are protected and whose lives are sacrificed?}

Bailouts that protect banks and bondholders while imposing austerity on workers preserve fictitious capital and restore the
discipline of repayment. Jubilees that relieve households can materially help, but without socialising finance and controlling
credit allocation, the debt machine tends to rebuild itself through new lending, asset bubbles, and renewed extraction.

Debt relief is therefore best understood not as a final solution but as a terrain of struggle over who bears crisis costs,
and whether finance remains a private power or is brought under democratic control.

%---------------------------------------------------------
\subsection{``Green capitalism'': carbon markets, offsets, ESG, and climate finance}
\label{sec:policy-green-capitalism}
%---------------------------------------------------------

The dominant ruling-class response to ecological crisis increasingly takes the form of ``green capitalism'': carbon pricing,
offset markets, ESG metrics, green bonds, and climate-finance instruments. Some of these measures can shift relative prices
and force limited disclosures. But they also tend to reproduce the deeper contradiction: capitalism treats ecological repair
as a new field for profitable claims, rents, and enclosures.

Carbon markets and offsets can convert the atmosphere and land-use into tradable assets, producing new revenue streams for
consultancies, financiers, and land intermediaries. ESG and climate finance can become reputational and financial screens
that reprice assets without guaranteeing absolute reductions in extraction and throughput. In Marxian terms, these strategies
often create new layers of fictitious capital and rent extraction on top of an accumulation process that still requires
expansion, competitiveness, and cost-cutting. The result can be a ``green'' veneer that leaves the compulsion to grow and to
externalise ecological costs intact.

A socialist approach begins from a different premise: ecological transition is a planning problem, not primarily a pricing
problem. It requires deliberate decisions about what to phase out, what to build, and how to allocate labour, energy, and
materials under democratic control. That, in turn, requires confronting ownership in energy, transport, housing, land, and
finance.

%---------------------------------------------------------
\subsection{Regulation-only fixes: antitrust and ``better governance''}
\label{sec:policy-regulation-only}
%---------------------------------------------------------

Finally, there are proposals that aim to civilise capitalism through regulation: antitrust, platform rules, transparency,
corporate governance reform. Such measures can curb specific abuses. But they do not abolish exploitation, competition, or
the profit imperative. Monopolies and oligopolies are not just policy accidents; they are recurrent outcomes of accumulation,
concentration, and control over chokepoints (land, finance, logistics, intellectual property). Regulation can change tactics.
It does not remove the underlying compulsion that drives the system toward crisis and toward new forms of domination.

%---------------------------------------------------------
\subsection{The general point: what these fixes can and cannot do}
\label{sec:policy-general-point}
%---------------------------------------------------------

The common thread is not that reforms are irrelevant. It is that reforms which act mainly on \emph{distribution},
\emph{circulation}, or \emph{regulation}, while leaving the ownership and control of production intact, cannot eliminate the
contradictions generated by accumulation itself. They can manage symptoms, redistribute some pain, and win real improvements.
But they also tend to trigger counter-moves by capital (price recapture, disinvestment, informalisation, financial speculation)
and by the state (austerity, ``confidence'' politics, repression) unless they are backed by organised power and structural
measures that neutralise the capitalist veto.

This is why crisis management so often feeds directly into financialisation (the next section): when productive profitability
is weak and deeper transformations are blocked, the system leans harder on credit, asset prices, and speculative claims.

%=========================================================
\subsection{Reforms as struggle: making gains durable by confronting power}
\label{sec:reforms-as-struggle}
%=========================================================

Communists therefore fight on two levels at once.

First, we fight for immediate reforms that materially improve life: higher wages, shorter hours, strong labour rights, and
universal provision of essentials. These struggles build organisation and confidence, and they expose the real lines of power.

Second, we insist that making such gains \emph{durable} requires confronting the capitalist veto. In practice, that points
toward measures such as:

\begin{itemize}[leftmargin=1.5em]
  \item \textbf{decommodifying essentials} through universal basic services (housing, healthcare, education, transport, energy,
  water, care), so wage gains and cash transfers are not immediately recaptured by rents and prices;
  \item \textbf{capital controls and financial regulation with teeth}, so investment strikes and capital flight are not the
  automatic response to popular reforms;
  \item \textbf{socialising key sectors} (especially finance, energy, transport, and strategic supply chains) so credit and
  investment can be directed by democratic priorities rather than private profit;
  \item \textbf{workers' organisation and workplace power} (unions, shopfloor committees, collective bargaining, the right to
  strike), without which reforms become paper promises;
  \item \textbf{ecological planning} that treats decarbonisation and adaptation as deliberate social projects, not as markets
  for offsets and speculative green assets.
\end{itemize}

In short: reforms matter, but reformism fails. Parliamentary measures can win real improvements, but the system's contradictions
are rooted in production and property relations. To move beyond recurring crisis, society must bring investment, production, and
the conditions of life under democratic control. That is the horizon that gives reform struggles their strategic direction.

%=========================================================
\section{Financialisation, real estate, and fictitious capital}
\label{sec:financialisation}
%=========================================================

When profitability in productive sectors is squeezed, capital tends to
search for easier routes to profit: finance, real estate, speculation on
currencies and commodities. This is often called \textbf{financialisation}.

\subsection{Fictitious capital}

\textbf{Fictitious capital} refers to \emph{tradable claims} on future income streams
ultimately grounded in surplus value: shares (claims on profits/dividends), bonds and loans
(claims on interest), mortgage-backed securities (claims on interest and principal), and
many derivatives (claims whose payoffs are linked to the prices of other claims).
It is ``fictitious'' not because it is unreal, but because it is a \emph{title} to future
income rather than a direct embodiment of value already produced and currently functioning
as capital in production.\cite{HarveyNewImperialism}

A share can be understood as a claim on a stream of expected future profits (or dividends);
a bond as a claim on a stream of interest payments; and many structured assets as claims on
bundled or re-sliced cash flows. A basic way to see this is capitalisation. If an asset is
expected to yield net receipts $R_t$ (dividends, interest, rents, or other contractual payoffs)
over time, and those receipts are discounted at rate $d$ (which may incorporate risk and liquidity),
then its market valuation can be written schematically as
\begin{equation}
  K_f = \sum_{t=1}^{\infty} \frac{R_t}{(1+d)^t}.
\end{equation}
We use $d$ (not $i$) for the discount rate to avoid confusing it with the commodity index $i$
used elsewhere. This is a stylised expression (it assumes a stable discounting convention and a
reasonably well-defined expected flow), but it captures the key point: the market value of the
claim can move sharply even when the underlying production of value changes little.

Here $K_f$ is not ``capital'' in the sense of means of production and labour-power set in motion;
it is the \emph{price of a claim} on future cash flows. In short: it is paper wealth whose valuation
is governed by conventions of expectation and discounting, even though the ultimate substance that can
validate those expectations is surplus value produced in the sphere of production.

The critical Marxian point is the macro-constraint: the expansion of fictitious capital can
run far ahead of the growth of surplus value actually produced and realised. Crises erupt when
the expectations embedded in $K_f$ become incompatible with the underlying capacity of production
and realisation to sustain those cash flows. Asset prices then fall sharply, destroying or
revaluing paper wealth, while leaving behind debts, unemployment, and social damage.

\subsection{Tech hype as a vehicle for speculative valuation}

A contemporary cultural form of fictitious capital is techno-utopian speculation: venture-funded firms and their investors often sell not present profitability but stories about future monopoly positions and future command over revenue streams. Under conditions of weak productive profitability, capital
hunts for assets whose \emph{valuation} can be pushed upward faster than any measured improvement
in everyday life. The point is not simply to sell a useful product; it is often to capitalise
an expectation of future revenue, lock in market power, and attract early entrants to a rising claim.

High-profile frauds are only the most visible edge of this logic. More commonly, we see cycles where
technologies are marketed as world-changing mainly because that narrative can justify venture capital,
inflate stock prices, or produce monopoly rents through patents, platforms, and gatekept access.
In this sense, techno-utopianism is not merely an idea; it is a \emph{financial mechanism} that helps
convert uncertain futures into tradable claims in the present.

None of this implies that technological development is irrelevant. It implies that under capitalism
the dominant selector is not social need but expected profitability and controllable revenue streams.
That is why comparatively “boring” public goods---mass transit, universal sanitation, public housing,
open-access health systems, resilient grids---are chronically underbuilt, while speculative fantasies
can attract extraordinary funding. The difference is not technical feasibility alone; it is the
difference between what meets needs and what can be owned, priced, and capitalised.

\subsection{Example: real estate and speculation in Pakistan}

In many parts of the global South, including Pakistan, the combination of
energy costs, infrastructural bottlenecks, political instability, and import
dependence has made productive investment risky. At the same time, property
speculation, urban land grabs, and financial arbitrage have offered
relatively easier gains for those with capital.\cite{StrugglePK_TDR}

This pattern is inseparable from belated, uneven development and the structure of the state. Where local
capital has historically remained narrow, import- and contract-dependent, and politically risk-averse, it has
often failed to build an industrial base capable of absorbing labour at scale and raising productivity in a
broad, socially transformative way. In that vacuum, the most coherent and force-concentrated institutions of
the capitalist state can become not only coercive arbiters but also major economic organisers: they sit atop
land allocations, procurement chains, regulatory power, and the ``security'' framing through which budgets and
property regimes are protected. The point is not that such institutions are alien to the state, but that they
are \emph{one of its central forms}---and, in crisis conditions, they can expand their economic footprint as
other fractions falter.

In that context, liberal hopes that rotating civilian parties will steadily produce a secular-democratic
modernisation run into a material barrier: the underlying property relations, the world-market constraint, and
the coercive apparatus that defends both. Without confronting those relations---especially land, finance,
strategic infrastructure, and the external discipline of debt and conditionality---electoral alternation tends
to reshuffle administrators while leaving the core drivers of speculation, informalisation, and repression
intact.\cite{StrugglePK_WorldPerspectives,PeetUnholyTrinity}


This leads to:

\begin{itemize}[leftmargin=1.5em]
  \item a misallocation of resources towards luxury housing, gated
  communities, and malls while basic manufacturing and public services are
  starved of investment;
  \item a reliance on imported goods and remittances, generating trade
  deficits and external debt; and
  \item rising land prices that push workers and peasants to the margins,
  deepening class and gendered burdens of social reproduction.
\end{itemize}

From a Marxian perspective, this is not an aberration but a variant of the
general tendency of capital to flee from the hard work of exploiting living
labour in production towards speculative claims on future surplus value.

%=========================================================
\section{Why social democracy was a conjuncture, not an alternative system}
\label{sec:socdem-conjuncture}
%=========================================================

Liberals often say: ``Communism is too extreme. Let's aim for social democracy.'' In everyday discussion,
``social democracy'' is treated as if it were a different economic system: capitalism with a human face.
From a Marxist standpoint, that is a category mistake. Social democracy, where it existed in a strong form,
was not a separate mode of production. It was a historically specific \emph{regime of managing capitalism}
under unusually favourable conditions for capital accumulation, and under unusually strong pressure from below.

This matters for strategy. If social democracy were simply a matter of political will and better policy,
then the task would be to elect kinder managers. But if it was a conjuncture inside capitalism—enabled by
post-war reconstruction, imperial advantage, high growth, and a particular balance of class forces—then the
question becomes: what happens when those conditions vanish? Marx's answer is not moral; it is structural:
capitalism can tolerate reforms only so long as they do not block the extraction and realisation of surplus value.

To make this concrete, we need a short historical map: how capitalism expanded globally, why wars and depressions
recur, why the post-war boom was exceptional, why neoliberalism followed, and why ``returning'' to that earlier
compromise is structurally blocked today.

% Hide the many subsections of this long historical section from the TOC
\setcounter{tocdepth}{1}

%---------------------------------------------------------
\subsection{Colonialism, the world market, and why capital must expand}
\label{subsec:colonialism-worldmarket}
%---------------------------------------------------------

Capitalism does not grow peacefully inside national borders. It generalises commodity production and then
compels expansion: more markets, more inputs, cheaper labour, and new outlets for accumulated capital.
The world market is not a later ``globalisation'' add-on. It is built into capitalist development from early on.

Colonialism was a decisive mechanism in this expansion. It forcibly reorganised land, labour, and trade to
secure raw materials, open captive markets, and impose money taxes that pushed people into wage labour and
commodity dependence. In Marxist terms, this was not simply robbery for its own sake; it was the violent
construction of capitalist social relations on a world scale: dispossession, coerced labour, export monocultures,
and the subordination of entire regions to metropolitan accumulation.

As capitalist competition intensifies, the stakes rise. Capitals do not compete only as firms; they lean on
states for protection, subsidies, access, and coercion. Conflicts over markets, routes, strategic resources, and
investment territories therefore become conflicts between states. That is the material basis of modern imperialism:
the fusion of concentrated capital with state power in the struggle over world-scale conditions of accumulation.

This does not mean that every war has one simple cause. It means that, under capitalism, rivalry over world-market
advantage and control over conditions of accumulation is not accidental. It is structural.

%---------------------------------------------------------
\subsection{War, depression, and the reorganisation of capitalism (1914--1945)}
\label{subsec:1914-1945}
%---------------------------------------------------------

The first half of the twentieth century shows the brutal logic of capitalist rivalry and crisis.

\paragraph{World War I (1914--1918).}
The First World War was an inter-imperialist conflict among major powers competing over colonies, markets,
routes, and geopolitical advantage. It was not a temporary lapse into irrationality; it was a catastrophic
expression of rivalry in a world where capitalist development had become uneven, combined, and militarised.

\paragraph{The interwar instability and the Great Depression.}
The 1920s and 1930s were shaped by unstable recoveries, debt burdens, and then collapse. The Great Depression
(ignited by the 1929 financial crash and deepened through the early 1930s) produced mass unemployment,
bank failures, and political crisis. States responded with protectionism, competitive devaluations,
and intensified nationalism. In Marxist terms, depression is not merely a shortage of demand; it is a crisis
of profitability, overaccumulation, and realisation in which too much capital has been accumulated relative to
profitably employable labour and markets at the prevailing conditions.

\paragraph{World War II (1939--1945).}
The Second World War was also rooted in imperial rivalry and the drive to reorganise world power. It destroyed
enormous masses of capital and human life. That destruction was barbaric—but it also reset conditions of
accumulation. One cannot understand the post-war boom without understanding that war and depression forcibly
restructured the system: debts were rewritten, industries were reorganised, and global leadership changed.

%---------------------------------------------------------
\subsection{The post-war boom and the social-democratic compromise (roughly 1945--early 1970s)}
\label{subsec:postwar-boom}
%---------------------------------------------------------

After 1945, much of Western Europe and Japan underwent reconstruction. The period from the late 1940s through
the early 1970s is often described as a ``Golden Age'' of capitalism: rapid growth, rising productivity, and,
in many countries, expanding welfare states and labour protections. It is this period that many people have in
mind when they say ``social democracy.''

From a Marxist standpoint, three points are decisive.

\paragraph{First: Keynesianism managed capitalism; it did not replace it.}
Keynesian policy—state spending, public investment, welfare provision, and demand management—did not abolish
wage labour, private ownership, or profit as the regulator of investment. It aimed to stabilise accumulation,
smooth the cycle, and contain social conflict. In other words, it was a mode of governance inside capitalism,
not an alternative mode of production.

\paragraph{Second: the compromise was forced from below.}
Welfare gains were not gifts. They were concessions wrested from capital under conditions of militant unions,
strong socialist and communist parties, mass strike waves, and, internationally, the existence of a rival bloc
that frightened Western ruling classes. The post-war order was built in part to prevent revolution and contain
communism. The welfare state was therefore both a product of class struggle and a strategy of class rule.

\paragraph{Third: the boom rested on exceptional material conditions.}
Reconstruction created vast demand for steel, housing, transport, and industry. Productivity gains were high in
many sectors. US hegemony and new monetary arrangements stabilised key trade and financial relations. Cheap energy
inputs supported growth. These conditions made it easier for capital to tolerate redistribution for a time,
because the mass of profits could still expand.

But the compromise was conditional. It worked only so long as the underlying engine—profitability in the productive
economy—could sustain it.

%---------------------------------------------------------
\subsection{Why the boom broke: profitability, stagflation, and the crisis of the 1970s}
\label{subsec:1970s-crisis}
%---------------------------------------------------------

The post-war boom did not end because society became less reasonable. It ended because capitalism's contradictions
reasserted themselves.

As accumulation proceeded, competitive pressures intensified, and overcapacity developed in key industries.
Profitability came under pressure. States could stimulate demand, but demand management cannot guarantee that
private capital will invest profitably. When profitability is weak, capital seeks other outlets: speculation,
real estate, financial claims, or relocation to cheaper labour regimes.

The 1970s were also marked by major oil shocks (notably 1973--74 and 1979), which raised costs across production
and transport. The resulting combination—sluggish growth with rising prices—became known as stagflation, a problem
that standard Keynesian tools were ill-suited to manage within capitalist constraints. From a Marxist perspective,
oil shocks were not the sole cause; they were a trigger and amplifier within a deeper profitability and accumulation
problem.

The political conclusion drawn by ruling classes was clear: the post-war compromise had to be dismantled to restore
profitability and discipline labour.

%---------------------------------------------------------
\subsection{Monetarism and neoliberalism: restoring class power}
\label{subsec:neoliberal-turn}
%---------------------------------------------------------

What followed is now called neoliberalism: privatisation, deregulation, attacks on unions, cuts to social spending,
and the reorientation of states toward ``market confidence.'' This was not merely an ideological fad. It was a class
strategy to restore profitability by reshaping the balance of power between labour and capital.\cite{HarveyNeoliberalism}

Monetarism—the shift toward using interest rates and tight money as the primary policy tool—played a key role.
High interest-rate regimes disciplined wages, raised unemployment, and strengthened creditors. In Marxist terms,
this was a reassertion of capital's power through the labour market and through finance: a deliberate raising of the
social cost of resistance.

Neoliberalism also internationalised discipline. Trade liberalisation and global supply chains increased the credible
threat of relocation. Financial liberalisation increased the power of capital flight. States were increasingly told to
treat social spending as a cost, not a right.

This is why, even where social-democratic parties remained in office, they increasingly administered neoliberal policy.
The system's constraints tightened, and reformist parties became managers of austerity.

%---------------------------------------------------------
\subsection{The global South as laboratory: debt, coups, and structural adjustment}
\label{subsec:global-south-laboratory}
%---------------------------------------------------------

Neoliberalism was not rolled out only through elections. It was imposed through coups, repression, and debt regimes,
especially in the global South.

Latin America provides clear examples of ``shock'' restructuring: labour rights crushed, public assets privatised,
and economies opened to foreign capital. Debt crises—especially from the late 1970s into the 1980s—became a lever.
States dependent on external finance were pushed into IMF-style programmes: austerity, currency devaluation,
cuts to subsidies, and privatisation.\cite{PeetUnholyTrinity} These policies were sold as ``stabilisation,'' but their class content was
consistent: shift crisis costs onto workers and the poor, open new fields for accumulation, and guarantee repayment
to creditors.

This history matters for countries like Pakistan because it clarifies that fiscal policy is not only a domestic choice.
External debt, import dependence, and creditor conditionality shape what states are ``allowed'' to do under capitalism.

%---------------------------------------------------------
\subsection{Belated capitalism, combined and uneven development, and the bourgeois-democratic impasse}
\label{subsec:combined-uneven}
%---------------------------------------------------------

A further clarification is needed for much of the global South: capitalism does not arrive everywhere as a
``clean'' national development sequence in which an independent bourgeoisie steadily builds an industrial
base, completes land reform, secularises the state, and then deepens democratic participation. It arrives
\emph{through a world market already dominated by monopolies, imperial states, and finance}, and it therefore
produces a pattern of \textbf{combined and uneven development}: advanced technologies and enclaves of modern
production sit alongside older property relations, coercive labour regimes, and vast zones of abandonment.\cite{TrotskyPermanentRevolution,RosenbergUCD}
The result is not ``underdevelopment'' as a cultural failure, but a structured outcome of integration into a
hierarchical world economy.\cite{LeninImperialism,PatnaikImperialism}

This matters because the classic tasks often associated with bourgeois-democratic revolutions---land reform,
a thorough secularisation of law and schooling, universal civil equality, democratic control over the state,
and the construction of an inclusive nation-state on a modern material basis---collide with the actual
interests and dependencies of local ruling classes in belated capitalist formations. A dependent bourgeoisie
typically sits at the intersection of landlordism, foreign capital, state contracts, and creditor discipline.
It therefore fears mass mobilisation from below more than it fears the survival of ``pre-modern'' coercions
and compromises. Even when liberal elections occur, decisive questions about investment, energy, trade,
security, and debt servicing remain constrained by property, the world market, and external conditionality.\cite{PeetUnholyTrinity,HarveyNeoliberalism}

This is why the democratic question cannot be reduced to whether elections are formally competitive. In many
settings, the deeper absence is \emph{popular decision-making power over the material organisation of life}:
what is built, where, for whom, with what labour conditions, and under what ecological constraints. When those
decisions remain governed by profitability, creditor confidence, and unelected coercive apparatuses, liberal
democracy becomes a narrow procedure for rotating managers of an order whose core decisions are insulated from
the majority.

\paragraph{A note on ``outliers''.}
There are partial exceptions where state-led industrialisation achieved significant upgrading (often under
specific geopolitical conditions, with strong state direction of credit, export strategy, and sometimes land
reform). But even these cases do not refute the general point: capitalist development remains structured by the
world market, and democratic control over investment is not guaranteed by growth. The question is not simply
``development'' but \emph{who controls it} and \emph{whose interests it serves}.

The strategic conclusion is not that elections never matter. It is that, under dependent and crisis-ridden
capitalism, the completion of democratic and social tasks cannot be entrusted to bourgeois parties whose
material reproduction depends on the very property relations and external constraints that block those tasks.
In that sense, what liberals treat as the horizon of ``secular, democratic progress'' becomes inseparable from
working-class struggle for power: decommodification of essentials, democratic control of investment, and
breaking the mechanisms of imperial and creditor discipline.\cite{PeetUnholyTrinity,HarveyNeoliberalism}


%---------------------------------------------------------
\subsection{Financialisation and bubble-led management: dot-com to housing}
\label{subsec:dotcom-2008}
%---------------------------------------------------------

As productive investment became more difficult or less profitable, capitalism leaned harder on finance, asset prices,
and credit expansion. This is the context for repeated bubble cycles.

A stylised pattern is:
\begin{itemize}[leftmargin=1.5em]
  \item weak or uneven productive profitability;
  \item expanding credit and rising asset prices to sustain demand and paper wealth;
  \item speculative booms (equities, then real estate, then complex securities);
  \item crashes that socialise losses through bailouts, austerity, and restructuring.
\end{itemize}

The dot-com boom of the late 1990s and its bust around 2000--2002 showed how speculative expectations could inflate
claims far beyond realised profitability. The much larger global crisis of 2007--08 then demonstrated the same logic
at a systemic scale: mortgage-credit expansion, securitisation, and bank leverage created vast fictitious claims that
collapsed when repayment and realisation failed.

From a Marxist standpoint, these crises are not simply regulatory accidents. They are expressions of a system trying
to offset profitability and realisation problems by expanding financial claims.

%---------------------------------------------------------
\subsection{After 2008: austerity, revolt, and the rise of the far right}
\label{subsec:post-2008}
%---------------------------------------------------------

After 2007--08, states intervened massively to stabilise banks and financial markets, while much of the cost was pushed
onto working classes through austerity, wage restraint, privatisation, and degraded public services. This was not
``balanced'' crisis management; it was class management.

The political consequences were explosive. Across different regions, we saw waves of revolt and experimentation:
Occupy-style movements, mass square occupations, anti-austerity uprisings, and attempts at electoral ruptures. The Arab
Spring expressed deep social contradictions under authoritarian and neoliberal regimes. In Europe, anti-austerity
projects rose and collided with the hard constraints of capital mobility, creditor power, and supranational governance.

Where the left failed to build durable organisation and a credible alternative, the right often filled the vacuum.
Far-right politics thrives by redirecting anger away from capital and toward scapegoats: migrants, minorities, women,
queer and trans people, religious targets, ``corrupt elites'' in the abstract. It offers belonging, revenge, and
authoritarian order while leaving property relations intact. Liberalism, which manages decline while preaching civility,
often cannot defeat this; it frequently feeds it by policing dissent and protecting wealth.

This is one reason the question of social democracy returns today as nostalgia: people feel the loss of security,
services, and meaning. But nostalgia is not a strategy.

%---------------------------------------------------------
\subsection{Why you cannot simply ``return'' to social democracy today}
\label{subsec:no-return-socdem}
%---------------------------------------------------------

Even if one admires the best welfare achievements of the post-war period, restoring that regime today is structurally
blocked under contemporary capitalism.

\begin{enumerate}[leftmargin=1.5em]
  \item \textbf{Capital mobility is greater.} Supply chains, finance, and ownership structures allow capital to move,
  threaten, and evade more easily. This strengthens the capitalist veto: disinvestment and capital flight become routine
  weapons against reforms.

  \item \textbf{The global balance of class forces has shifted.} Union density and workplace power are weaker in many
  places after decades of defeat, repression, and fragmentation. Without strong organisation from below, welfare gains
  are easier to roll back.

  \item \textbf{Debt and external constraints are tighter.} Many states operate under heavy debt burdens and creditor
  conditionality. In the global South, import dependence and foreign-exchange vulnerability amplify these constraints.
  This is why IMF-style programmes so often demand cuts to public spending and ``market reforms.''

  \item \textbf{The ecological constraint is now unavoidable.} The post-war boom assumed cheap energy and expanding
  throughput. Today, climate breakdown and ecological limits mean that a growth-at-all-costs welfare compromise is not
  a viable horizon.\cite{IPCCAR6WGIII} Transition requires planning, not just redistribution.

  \item \textbf{Financialisation changes the terrain.} Asset-price inflation, rent extraction, and private control of
  housing, energy, and infrastructure mean that cash-based reforms are often recaptured through prices unless essentials
  are decommodified.
\end{enumerate}

This is why policy packages that promise a painless return—``just tax the rich,'' ``just print money,'' ``just do a Green
New Deal inside the market''—run into the structural limits already developed in this pamphlet. (See also the discussion
of the capitalist veto and crisis management in Section~\ref{sec:policy-fixes-veto}.) Reforms matter, but
reformism fails: the system can shed a reform regime when profitability, competition, and class power demand it.

%---------------------------------------------------------
\subsection{What this means for strategy: reforms as struggle, socialism as a necessity}
\label{subsec:strategy-reforms-socialism}
%---------------------------------------------------------

The Marxist conclusion is not that reforms are pointless. It is that reforms must be understood as \emph{sites of struggle}
that can build power, expose limits, and create openings—rather than as a parliamentary substitute for transforming property
relations.

This has several practical implications.

\begin{itemize}[leftmargin=1.5em]
  \item Fight for reforms that \textbf{decommodify essentials}: universal healthcare, housing, education, transport, energy,
  and care. These reduce capital's ability to recapture gains through rents and administered prices.

  \item Build \textbf{workplace power} and democratic organisation, without which legal rights become paper promises.

  \item Demand measures that \textbf{confront the capitalist veto}: controls on capital flight, democratic control of credit
  and investment, and socialisation of key sectors (especially finance, energy, and strategic infrastructures).

  \item Treat internationalism as material, not moral: without confronting imperialist and debt relations, national reform
  projects are strangled by external constraint.
\end{itemize}

In this sense, socialism is not a slogan chosen because it is morally superior. It becomes a practical necessity because
the problems people face today—crisis management by austerity, debt discipline, ecological breakdown, and the far-right
turn—are produced by a system in which production and investment are governed by profit and protected by private property.
A democratic, planned, internationally coordinated reorganisation of the economy is not the ``extreme'' option; it is the
option adequate to the scale of the contradictions.

% Restore subsections in the TOC from here onward
\setcounter{tocdepth}{2}

%=========================================================
\section{Imperialism, global value chains, and unequal exchange}
\label{sec:imperialism}
%=========================================================

Capitalism has been global from early on. Marx wrote about world markets and
colonial plunder; later Marxists analysed imperialism as a stage where
capital exports, monopolies, and state power become central.\cite{LeninImperialism,PatnaikImperialism}

Today, global value chains and unequal exchange structure the relation
between global North and South.

\subsection{Unequal exchange in practice}

Consider again:

\begin{itemize}[leftmargin=1.5em]
  \item garment workers in Bangladesh producing for fast-fashion brands
  headquartered in Europe; and
  \item electronics workers in East and Southeast Asia assembling devices for
  firms headquartered in the US, Japan, or South Korea.
\end{itemize}

Studies~\cite{Smith2016,Hickel2022,DedrickKraemerLinden2010} of the cost structure of an iPhone or a pair of branded jeans show
that:

\begin{itemize}[leftmargin=1.5em]
  \item direct manufacturing wages are often less than 5--10\% of the retail
  price;
  \item most of the surplus value is realised and appropriated in the centres
  of finance, marketing, and intellectual property; and
  \item Northern states and firms control key chokepoints: technology,
  finance, logistics, trade rules.\cite{Smith2016,Hickel2022}
\end{itemize}

A growing portion of this power takes the form of \textbf{digital and intellectual-property enclosures}.
Patents, licensing, paywalls, proprietary platforms, and “terms of service” operate as mechanisms of monopoly
rent: they restrict access to knowledge and infrastructure that is socially produced, then charge for entry.
This is not a moral claim about individual consumers; it is a structural claim about ownership.
Under capitalism, even when the marginal cost of copying software, research, or cultural work is near zero,
property law is used to impose artificial scarcity so that revenue streams can be defended. This is a clear case where the use-value of knowledge---its capacity to be copied and shared at near-zero marginal cost---is subordinated to exchange-value through enforced exclusivity.

This matters for unequal exchange because it allows firms headquartered in the North to appropriate value
created across the world not only through manufacturing and trade, but through controlling chokepoints of
knowledge, standards, and access. The result is that Southern labour can produce, but Northern ownership can
capture: value creation is geographically distributed, while value appropriation is organised through
monopoly rights.

From a Marxian point of view, value is created wherever living labour is
expended, but the \emph{distribution} of surplus value is shaped by power,
monopoly, and imperial structures. Unequal exchange is not simply about
unfair prices; it is a mechanism for sustained transfer of value from South
to North.

\subsection{Debt, austerity, and structural adjustment}

Imperialist relations are reinforced through sovereign debt, IMF and World
Bank conditionalities, and trade agreements. Southern economies are pushed
towards export-oriented, low-wage specialisation, while austerity shrinks
public services and public employment.\cite{StrugglePK_WorldPerspectives}

This directly affects class struggle: workers, peasants, and oppressed
genders face both the local bourgeoisie and global institutions as they fight
for wages, land, services, and democratic rights.

%=========================================================
\section{Social reproduction, unpaid labour, and gender}
\label{sec:social-reproduction}

Class struggle does not occur only in the workplace. The reproduction of labour-power---the everyday and
generational work of keeping human beings alive, functional, and able to return to work---is itself a
site of struggle. This includes feeding, cleaning, rest and recovery, caring for children and elders,
emotional support, maintaining social ties, and community organising. It also includes the broader
social conditions that make labour-power available at all: housing, water, transport, schooling,
healthcare, and safety, whether provided privately in households, collectively in communities, or through
the state.

Capitalism depends on this reproductive work, but it does not automatically pay for it. The wage is the
money-form of the value of labour-power---the portion of social labour that reproduces the worker---yet
much of the work that actually reproduces labour-power is displaced onto households, disproportionately
onto women and marginalised genders, and onto racialised and migrant workers in low-paid ``care'' and
service sectors. Feminist Marxists speak of \textbf{social reproduction} to name this terrain: the
interlinked processes through which labour-power is produced, maintained, repaired, and renewed as a
condition for capitalist production.\cite{BhattacharyaSRT,FedericiCaliban}

This matters politically because it changes what we count as ``economic.'' If the workplace is one
frontline of exploitation, the household and community are another frontline where the costs of that
exploitation are absorbed, managed, and fought over. When public services are cut, when inflation hits
food and rent, when care is privatised, when migration regimes split families, or when violence and
criminalisation target gendered survival, the pressure shows up as longer unpaid hours, worse health,
and tighter control over bodies and time. Social reproduction is therefore not a ``side issue'' to class
politics; it is one of the main ways class power is organised and contested.

\subsection{Unpaid and underpaid labour}
\label{subsec:unpaid-underpaid}

Much of the labour required to maintain and reproduce \textbf{labour-power} is unpaid or severely
underpaid, and it is disproportionately carried by women and marginalised genders. By ``reproduction of
labour-power'' we mean not only keeping individual workers going day to day (food, rest, care, recovery),
but also the broader reproduction of the working class across time (child-rearing, education, health,
and the everyday social supports that make life possible). This includes:

\begin{itemize}[leftmargin=1.5em]
  \item domestic work in the household (cleaning, cooking, fetching water, managing the home);
  \item care work in families and communities (childcare, elder care, disability care, emotional labour, mutual aid);
  \item informal labour in home-based piecework, subsistence agriculture, and street vending;
  \item paid but low-waged reproductive/service work (domestic work, cleaning, nursing, teaching assistants, care homes),
  often done by migrant and racialised workers under coercive conditions;
  \item \textbf{intimate and sexual labour}, including sex work, which is frequently shaped by patriarchal violence,
  poverty, policing, and stigma. Where it is criminalised, workers face heightened exploitation and risk, and the
  boundary between ``consent'' and coercion is materially structured by survival, debt, and state repression.
\end{itemize}

Capital benefits from this arrangement because it can pay wages that cover only part of the true social
cost of reproducing labour-power. The missing costs do not disappear; they are shifted onto households,
onto unpaid time, onto stretched public services, and onto the bodies of those made responsible for care.
When that burden becomes intolerable, it becomes a site of struggle: demands for childcare, housing,
healthcare, shorter hours, safer workplaces, decriminalisation and labour rights, and collective provision
are not ``add-ons'' to class politics but part of the fight over how labour-power is reproduced and who
pays for it.

\subsection{A broader view of class struggle}

Struggles over public services, housing, food prices, and gendered violence
are thus \emph{also} class struggles. They contest who pays the costs of
reproducing labour-power: capital, the state, households, or communities.

For example:

\begin{itemize}[leftmargin=1.5em]
  \item When austerity cuts public healthcare, the burden shifts onto unpaid
  carers (mostly women).
  \item When housing is financialised and rents soar, workers must spend more
  of their wages on shelter, effectively raising necessary labour time and
  pressuring exploitation.
\end{itemize}

Marxian political economy enriched by social reproduction theory therefore
links factory, office, and platform struggles with feminist, queer, and
anti-racist movements.\cite{BhattacharyaSRT}

%=========================================================
\section{Technology, AI, and the future of work}
\label{sec:technology-ai}
%=========================================================

From automatic looms to industrial robots to AI systems, capital constantly
seeks to reduce its dependence on living labour. This is both a driver of
productivity and a source of contradiction.

\subsection{Automation and the law of value}

Automation tends to increase $\constcap$ relative to $\varcap$, raising \OCC{}
and putting downward pressure on \profitrate{} (\cref{sec:rate-profit}). At
the same time, it cheapens many goods, raises the technical capacity of
society, and potentially frees time for education, art, care, and play.

Under capitalism, however, the gains of automation are unevenly distributed:

\begin{itemize}[leftmargin=1.5em]
  \item some workers are displaced, others' labour is intensified;
  \item skills are polarised, with a few high-paid specialists and large layers of
  precarious, low-paid service workers;
  \item managerial and surveillance technologies extend the working day into
  the home via emails, apps, and metrics.
\end{itemize}

Despite the hype, AI does not abolish the law of value. Behind every
algorithm are infrastructures of material labour: data labellers, content
moderators, hardware assemblers, miners of rare-earth metals, and workers
maintaining data centres. Value is still created by human labour-power.

A particularly clear example is generative AI in offices, media, design, and “knowledge work.”
Much commentary treats it either as imminent mass automation or as a neutral productivity aid.
The more immediate dynamic is often neither: it is the \emph{moving benchmark} of socially necessary labour time.

When a tool allows a task to be completed faster under normal conditions, the socially enforced norm
for that task shifts. What used to be an acceptable daily output becomes “slow”; what used to require
two workers is redefined as one worker’s baseline. The gain is not automatically returned to workers
as shorter hours or lower intensity. If wages and hours do not change, the same paid time is made to yield more surplus labour in practice, raising the realised rate of exploitation \(e=s/v\) even when the contract form remains unchanged. Under capitalist control, it is typically converted into new
targets, tighter deadlines, and higher throughput per unit time.

In Marxian terms, the point is simple. The working day is still split between necessary and surplus labour.
If the wage does not rise with the new benchmark, and if working hours do not fall, then raising output
by tools and metrics is a practical route to raising the effective rate of exploitation: more surplus
is extracted from the same paid time. This is why many workers experience “productivity tools” as speed-up,
surveillance, and exhaustion rather than emancipation.


\subsection{AI as a class weapon}

In many workplaces, AI and digital systems are currently used as tools of
management and control:

\begin{itemize}[leftmargin=1.5em]
  \item automated scheduling, productivity tracking, and ratings in
  logistics and platform work;\cite{WoodEtAl2019}
  \item algorithmic management in call centres and warehouses;
  \item predictive policing and risk scoring in welfare systems.
\end{itemize}

From a Marxist perspective, the question is not ``Will AI take our jobs?'' in
the abstract, but who controls the technology, for what purpose, and how its
gains are distributed. Under capitalism, the default is to use AI to
intensify exploitation and surveillance, not to liberate time.

\subsection{Data centres, ecological cost, and democratic control of computation}

The same clarity applies to the material infrastructure behind “AI”: data centres, fibre networks,
server farms, and the mining and assembly labour embedded in hardware. These are not abstract clouds;
they are energy- and water-intensive industrial systems.\cite{IEADataCentres2019} The ecological question is therefore not
whether computation exists, but \emph{what it is used for}, \emph{who controls it}, and \emph{how its costs
and benefits are allocated}.

A key point is that modern cloud infrastructure is software-defined and multi-tenant: the same physical
systems can run radically different workloads side by side. One class of workloads can be socially
necessary and life-saving---storm tracking, flood mapping, early warning systems, public health modelling,
grid optimisation, drought monitoring. Another class can be socially low-value but profit-rich---advertising
optimisation, mass surveillance, speculative content factories, endless synthetic media churn.
It is not the existence of servers as such that decides the outcome, but the social relations that schedule
and govern them.

A socialist response, therefore, is not technological abolition but social reorganisation: treat major
computational infrastructures as part of the means of production and bring them under democratic control.
That implies at minimum: transparent accounting of energy and water use; public rules that prioritise
life-preserving and scientific workloads; strict limits on wasteful, speculative, and coercive uses;
and open access to essential datasets and models as public goods rather than private moats.
In transition, this is one concrete meaning of freeing technology from the profit motive: directing
digital capacity toward collective planning and safety rather than accumulation and control.


%=========================================================
\section{From critique to transition: why this matters for you}
\label{sec:transition}
%=========================================================

We can now step back and summarise what this conceptual and mathematical
journey offers.

\subsection{Diagnosing everyday exploitation}

The categories of value, surplus value, constant and variable capital,
the \profitratefull, the \organiccompfull, and the \TRPFfull\ are not abstract curiosities. They allow you to:

\begin{itemize}[leftmargin=1.5em]
  \item read your own wage, hours, and working conditions as part of a global
  struggle over necessary and surplus labour time;
  \item see how corporate profits, real-estate bubbles, and public austerity
  are linked through the circuits of capital;
  \item understand why reformist policies repeatedly run into structural
  constraints: profitability, capital flight, debt, and imperial pressure.
\end{itemize}

To know that your 8-hour shift contains 3 hours of necessary labour and 5
hours of unpaid surplus labour is not just a moral claim; it is a
quantitative, structural insight.

\subsection{Beyond capitalism: a world beyond exchange-value}

Marx did not think capitalism would collapse automatically. Crises create
openings, but the direction of change depends on struggle: between classes,
between oppressed and oppressor nations, between patriarchal and emancipatory
forces.

A world beyond capitalism would mean:

\begin{itemize}[leftmargin=1.5em]
  \item production organised directly for human needs and ecological
  sustainability rather than profit;
  \item democratic control over workplaces, infrastructures, and planning;
  \item the abolition of the wage relation and of private property in the
  main means of production;
  \item the withering away of the state as a specialised apparatus separate
  from society; and
  \item the freeing up of time and resources for care, creativity, and
  self-development.
\end{itemize}

In such a world, use-values rather than exchange-values would dominate. To be precise: a moneyless society is not achieved by merely banning banknotes. It requires dismantling
the social relations that make money necessary---generalised commodity production, wage-labour, and private
control over the main means of life. When core necessities are produced and distributed as rights, and when
access is organised directly through democratic planning and provisioning, money loses its function as the
universal key. The communist horizon is therefore the abolition of the value-form: not that ``nothing has
value,'' but that society no longer has to measure access and activity primarily through exchange-value and price. Money, classes, and borders would lose their reason for existence.

\subsection{Your place in the struggle}

No single pamphlet can tell you what to do. But it can suggest some
directions:

\begin{itemize}[leftmargin=1.5em]
  \item Organise in your workplace, campus, neighbourhood, and online
  networks; connect struggles over wages, housing, gender, race, and climate.
  \item Study collectively: read Marx, Engels, and contemporary Marxists;
  discuss; argue; relate theory to your experience.
  \item Build and support organisations---unions, feminist and queer
  collectives, socialist groups, tenant unions---that link immediate demands
  to the horizon of a classless, moneyless, stateless society.
\end{itemize}

Marx wrote that the point is not only to interpret the world, but to change
it.\cite{EngelsFuneral} Understanding the logic of capital is one step
towards participating consciously in that change.

% ... last paragraph of the main text ...

\clearpage
\appendix

% In article class, \appendix switches \thesection to letters and (normally) resets the counter,
% but we force the reset here to guarantee A, B, C... regardless of any class/package quirks.
\setcounter{section}{0}

% Keep section numbers short in the TOC (A, B, C...) to prevent overlap.
\renewcommand{\thesection}{\Alph{section}}

% ---------- Appendix A: Notation ----------
\section{Marxian notation at a glance}
\label{sec:notation}

\noindent This pamphlet uses a compact set of standard Marxian symbols. Where a symbol can mean different things in different contexts (for example, $k$ as ``cost price'' versus $k$ as a conversion factor), we state explicitly which meaning we use locally. We also use indices ($i,j,\dots$) as generic labels whose meaning is always stated in context (e.g.\ commodity $i$, firm $j$, industry $i$). We reserve $d$ for the discount (interest) rate in present-value expressions to avoid confusion with indices. Where we refer to prices or money magnitudes in numerical examples, we flag explicitly when we are using a simplifying proportionality between money magnitudes and values for expository clarity.

\begingroup
\footnotesize
\setlength{\LTpre}{6pt}      % space between paragraph and caption
\setlength{\LTpost}{0pt}
\setlength{\LTcapwidth}{\textwidth}
\setlength{\tabcolsep}{4pt}
\renewcommand{\arraystretch}{1.18} % a touch more breathing room

\begin{longtable}{@{}>{\centering\arraybackslash}p{0.12\textwidth}@{\hspace{8pt}}%
                  >{\raggedright\arraybackslash}p{\dimexpr\textwidth-0.12\textwidth-8pt\relax}@{}}

\caption{Key symbols and definitions used throughout the pamphlet.}\label{tab:notation}%
\tabularnewline[-0.6\baselineskip]   % pulls the top rule up toward the caption

% ---------- First page header ----------
\toprule
\textbf{Symbol} & \textbf{Meaning (as used in this pamphlet)} \\
\midrule
\endfirsthead

% ---------- Continuation page header ----------
\multicolumn{2}{@{}l@{}}{\footnotesize\itshape \tablename~\thetable\ (continued from previous page)}\\[-0.2\baselineskip]
\toprule
\textbf{Symbol} & \textbf{Meaning (as used in this pamphlet)} \\
\midrule
\endhead

% ---------- Continuation page footer (all but last page) ----------
\midrule
\multicolumn{2}{r@{}}{\footnotesize\itshape Continued on next page}\\[-0.2\baselineskip]
\bottomrule
\endfoot

% ---------- Last page footer ----------
\bottomrule
\endlastfoot

% --- rows start here ---

$i$ &
generic index used for a particular object (commodity $i$, industry $i$, etc.); the meaning is always stated locally at the point of use. \\

$j$ &
index for a producer/firm (firm $j$) when comparing multiple producers; in other contexts it may serve as a second commodity index, and the meaning is always stated locally. \\

$n$ &
number of firms/producers included in a simple multi-producer illustration (where needed). \\

$d$ &
discount (interest) rate used for present-value calculations (e.g.\ in the fictitious-capital formula). \\

$C$ &
a commodity (in circuit notation), or commodity-capital as the output of production. \\

$M$ &
money-capital advanced; in the circuit $M \rightarrow C \rightarrow M'$. \\

$M'$ &
money realised after sale, $M' = M + \Delta M$ (surplus-value realised in money-form). \\

$\Delta M$ &
increment of money in the circuit of capital: $\Delta M = M' - M$ (surplus-value realised in money-form). \\

$W$ &
total value of the commodity product of a given production period: $W = c + v + s$. \\

$c$ &
\textbf{constant capital}: value of used-up means of production (materials, energy, depreciation of machinery/buildings) transferred into the product; it does \emph{not} create new value. \\

$v$ &
\textbf{variable capital}: value advanced to purchase labour-power (the wage bill in value terms); ``variable'' because living labour can create more value than it costs. \\

$s$ &
\textbf{surplus value}: unpaid value created by living labour beyond $v$, appropriated by capital; in value terms $s = W - (c+v)$. \\

$e$ &
\textbf{rate of exploitation} (rate of surplus value): $e = \dfrac{s}{v}$. \\

$r$ &
\textbf{rate of profit}: $r = \dfrac{s}{c+v}$. \\

$\mathrm{OCC}$ &
\textbf{organic composition of capital}: $\mathrm{OCC} = \dfrac{c}{v}$. \\

$\mathrm{TRPF}$ &
\textbf{tendency of the rate of profit to fall}: \emph{ceteris paribus}, rising $\mathrm{OCC}$ tends to press $r$ downward over long periods (with counter-tendencies). \\

$T$ &
length of the working day (hours). \\

$T_{\mathrm{N}}$ &
\textbf{necessary labour time}: part of the working day that reproduces the value of labour-power (corresponding to $v$). \\

$T_{\mathrm{S}}$ &
\textbf{surplus labour time}: part of the working day beyond $T_{\mathrm{N}}$ that produces surplus value (corresponding to $s$); $T_{\mathrm{S}} = T - T_{\mathrm{N}}$. \\

$w_h$ &
money wage per hour under time-wages. \\

$w_p$ &
piece rate: money paid per unit under piece-wages. \\

$q$ &
output per hour (units/hour) in a simple piece-wage illustration. \\

$W_w$ &
money wage bill (total wages paid in money terms), introduced where needed to avoid clashing with $W$; for time-wages, $W_w = w_h\,T$. \\

$\mathrm{SNLT}$ &
\textbf{socially necessary labour time}: labour time required to produce a commodity under average conditions of production, with average skill and intensity, using prevailing technology. \\

$L^{\mathrm{SN}}_i$ &
SNLT for commodity $i$: socially necessary labour time required to produce one unit of commodity $i$ under average, prevailing conditions. \\

$L_{ij}$ &
labour time per unit of commodity $i$ at firm $j$ (as produced by $j$; in practice this stands in for both direct and indirect labour time embodied in the unit). \\

$q_{ij}$ &
output of commodity $i$ produced by firm $j$ in the period (used as a weight in output-weighted averages). \\

$k$ &
\textbf{cost price}: the money-capital laid out by the capitalist to purchase means of production and labour-power; in value terms, $k = c+v$. \\

$\MELT$ &
\textbf{monetary expression of labour time (MELT)}: conversion factor from labour-time to money magnitudes, used here as money per hour of \textbf{SNLT}. In this pamphlet we denote this conversion factor by $\MELT$ throughout. If $L^{\mathrm{SN}}_i$ is the socially necessary labour time embodied in one unit of commodity $i$, then its value in money terms is $V_i = \MELT\,L^{\mathrm{SN}}_i$. \\

$V_i$ &
value of one unit of commodity $i$ in money terms: $V_i = \MELT\,L^{\mathrm{SN}}_i$. \\

$K_f$ &
\textbf{fictitious capital}: market valuation (price) of a tradable financial claim on future income, typically modelled as the discounted present value of expected receipts. \\

$R_t$ &
expected net receipts (cash flows) at time $t$ for a financial claim (dividends, interest, rents, or other contractual payoffs), used in present-value expressions. \\

\end{longtable}
\endgroup

\subsection*{Reminder: two circuits of circulation}

We use standard circuit shorthand (with arrows indicating exchange, and with the dashed circuit notation \CMC{} and \MCM{} used elsewhere as compact labels):

\begin{itemize}[leftmargin=1.5em]
  \item $C \rightarrow M \rightarrow C$ (\CMC): selling in order to buy; the goal is use-value.
  \item $M \rightarrow C \rightarrow M'$ (\MCM): buying in order to sell; the goal is $\Delta M$.
\end{itemize}

\clearpage

% ---------- Appendix B: Glossary + Acronyms ----------
% Reset the “first glossary title” spacing flag right before printing.
\glsfirsttitletrue
% ============================
%  glossary.tex
%  Acronyms + glossary entries (glossaries-extra, no-index workflow)
%  ALSO prints Appendix B when % ============================
%  glossary.tex
%  Acronyms + glossary entries (glossaries-extra, no-index workflow)
%  ALSO prints Appendix B when % ============================
%  glossary.tex
%  Acronyms + glossary entries (glossaries-extra, no-index workflow)
%  ALSO prints Appendix B when \input{glossary} is used in content.tex.
%
%  IMPORTANT (preamble.tex):
%    \usepackage[acronym]{glossaries-extra}
%    \makenoidxglossaries
%    \def\GLOSSARYENTRIESONLY{}%
%    \loadglsentries{glossary}
%    \let\GLOSSARYENTRIESONLY\undefined
% ============================

\ProvidesFile{glossary.tex}[Glossary and acronym entries + Appendix printing]

% ----------------------------------------------------------------------
% 1) ENTRIES-ONLY MODE (when loaded in the preamble)
% ----------------------------------------------------------------------
\ifdefined\GLOSSARYENTRIESONLY

% ---------- Acronyms ----------
\newacronym{ai}{AI}{artificial intelligence}
\newacronym{ltv}{LTV}{labour theory of value}
\newacronym{snlt}{SNLT}{socially necessary labour time}
\newacronym{melt}{MELT}{monetary expression of labour time}
\newacronym{occ}{OCC}{organic composition of capital}
\newacronym{trpf}{TRPF}{tendency of the rate of profit to fall}

\newacronym{ubi}{UBI}{universal basic income}
\newacronym{ubs}{UBS}{universal basic services}
\newacronym{qe}{QE}{quantitative easing}
\newacronym{mmt}{MMT}{Modern Monetary Theory}
\newacronym{jg}{JG}{Job Guarantee}
\newacronym{esg}{ESG}{environmental, social, and governance}

\newacronym{imf}{IMF}{International Monetary Fund}
\newacronym{ipcc}{IPCC}{Intergovernmental Panel on Climate Change}

% ---------- Glossary terms (existing) ----------
\newglossaryentry{keynesianism}{
  name={Keynesianism},
  sort={Keynesianism},
  description={A macroeconomic approach associated with John Maynard Keynes that emphasises stabilising output and employment via demand management (especially fiscal policy and public spending), particularly during downturns.},
  first={Keynesianism (a macroeconomic approach that emphasises demand management—especially fiscal policy and public spending—to stabilise output and employment)}
}

\newglossaryentry{monetarism}{
  name={Monetarism},
  sort={Monetarism},
  description={A macroeconomic doctrine associated with Milton Friedman that prioritises controlling inflation by managing the money supply and/or interest rates, often via tight monetary policy.},
  first={Monetarism (a macroeconomic doctrine that prioritises inflation control via tight monetary policy—money supply and/or interest rates)}
}

\newglossaryentry{financialisation}{
  name={financialisation},
  sort={financialisation},
  description={A pattern in which profits, strategies, and power shift toward finance, asset price inflation, and rent extraction, rather than expanded productive investment and wage growth.}
}

\newglossaryentry{fictitious-capital}{
  name={fictitious capital},
  sort={fictitious capital},
  description={Tradable claims on future income streams (shares, bonds, securitised claims, many derivatives) whose market valuation can expand beyond surplus value currently produced, until crises force devaluation.}
}

\newglossaryentry{decommodification}{
  name={decommodification},
  sort={decommodification},
  description={Shifting access to essentials (housing, health, care, transport, energy, water) out of the market and away from ability to pay, toward rights-based provision.}
}

\newglossaryentry{capital-controls}{
  name={capital controls},
  sort={capital controls},
  description={Regulatory restrictions on cross-border movement of capital designed to limit capital flight, currency pressure, and the ability of owners to discipline reforms through financial exit.}
}

% ---------- Glossary terms (extended, targeted additions) ----------
\newglossaryentry{austerity}{
  name={austerity},
  sort={austerity},
  description={A policy package of spending cuts, hiring freezes, welfare retrenchment, regressive taxation, and/or user fees justified as “fiscal discipline”. In practice it often shifts crisis costs onto workers and the poor while protecting creditors and asset owners.}
}

\newglossaryentry{fiscal-consolidation}{
  name={fiscal consolidation},
  sort={fiscal consolidation},
  description={Reducing government deficits through spending cuts and/or tax rises. It is frequently presented as technocratic “budget repair,” but its class content depends on who is taxed, which services are cut, and whether interest payments to creditors are ring-fenced.}
}

\newglossaryentry{primary-balance}{
  name={primary balance},
  sort={primary balance},
  description={A government’s fiscal balance excluding interest payments on existing debt. A “primary surplus” can coexist with rising total debt burdens if interest costs remain high or growth is weak.}
}

\newglossaryentry{debt-service}{
  name={debt servicing},
  sort={debt servicing},
  description={Ongoing payments of interest and principal on debt. For many states, debt service becomes a prior claim on public revenue, structurally pressuring social spending and investment even without an explicit austerity programme.}
}

\newglossaryentry{conditionality}{
  name={conditionality},
  sort={conditionality},
  description={Policy conditions attached to loans or debt restructuring (commonly by the IMF or creditor blocs). Typical conditions include subsidy cuts, wage restraint, privatisation, deregulation, central bank “independence,” and fiscal consolidation.}
}

\newglossaryentry{structural-adjustment}{
  name={structural adjustment},
  sort={structural adjustment},
  description={A reform programme—historically associated with IMF/World Bank lending—that restructures economies toward export orientation, market pricing, privatisation, and reduced public provision. The “adjustment” is usually borne through depressed wages, weakened labour protections, and reduced social spending.}
}

\newglossaryentry{balance-of-payments}{
  name={balance of payments},
  sort={balance of payments},
  description={A country’s accounting of transactions with the rest of the world (trade in goods/services, income flows, and financial transfers). Persistent deficits often create pressure for devaluation, import compression, and external borrowing.}
}

\newglossaryentry{capital-flight}{
  name={capital flight},
  sort={capital flight},
  description={Rapid private movement of funds out of a country (or out of domestic investment into safer assets), often triggered by crisis, political conflict, or expectations of devaluation. Capital flight can force currency pressure, reserve loss, and harsher adjustment.}
}

\newglossaryentry{exchange-rate-pass-through}{
  name={exchange-rate pass-through},
  sort={exchange rate pass-through},
  description={The extent to which a currency devaluation raises domestic prices, especially for imported essentials (fuel, fertiliser, medicine, machinery). High pass-through can turn devaluation into immediate inflation and real-wage cuts.}
}

\newglossaryentry{inflation-targeting}{
  name={inflation targeting},
  sort={inflation targeting},
  description={A monetary-policy framework where the central bank prioritises hitting an inflation target, typically via interest-rate moves. Critics argue it can treat inflation as a purely monetary phenomenon while ignoring supply shocks, monopoly pricing, and import dependence.}
}

\newglossaryentry{protectionism}{
  name={protectionism},
  sort={protectionism},
  description={Using tariffs, quotas, licensing, local-content rules, or public procurement to shelter domestic producers from foreign competition. It can defend jobs and industrial capacity, but its effects depend on who controls protected firms, how prices/wages move, and whether technology/inputs are domestically available.}
}

\newglossaryentry{trade-liberalisation}{
  name={trade liberalisation},
  sort={trade liberalisation},
  description={Reducing tariffs, quotas, and other trade barriers. It is often sold as “efficiency,” but in unequal world markets it can accelerate deindustrialisation, worsen trade deficits, and deepen dependence on imported inputs and foreign currency.}
}

\newglossaryentry{import-substitution}{
  name={import-substitution industrialisation (ISI)},
  sort={import substitution industrialisation},
  description={A strategy to replace imports with domestic production through tariffs, credit allocation, industrial policy, and state procurement. ISI can build capacity, but it often hits constraints around technology, energy, foreign exchange, and class control of investment decisions.}
}

\newglossaryentry{qe-term}{
  name={Quantitative easing (QE)},
  sort={Quantitative easing},
  description={A central bank policy of purchasing government bonds and/or other financial assets to expand its balance sheet and push down longer-term interest rates. QE can stabilise financial markets, but it often inflates asset prices and does not automatically translate into productive investment or higher wages.}
}

\newglossaryentry{mmt-term}{
  name={Modern Monetary Theory (MMT)},
  sort={Modern Monetary Theory},
  description={A heterodox framework arguing that a state that issues its own currency cannot “run out” of money in the way households can; the binding constraints are real resources, productive capacity, and inflation dynamics. It stresses the role of taxation and bond issuance in managing demand, distribution, and monetary conditions rather than “funding” spending in a mechanical sense.}
}

\newglossaryentry{ubi-term}{
  name={Universal basic income (UBI)},
  sort={Universal basic income},
  description={An unconditional cash transfer to all residents or citizens. Proposals differ sharply: some are designed to replace welfare and subsidise low wages, while others are framed as an income floor that complements strong public services, labour rights, and decommodification.}
}

\newglossaryentry{ubs-term}{
  name={Universal basic services (UBS)},
  sort={Universal basic services},
  description={A model of guaranteeing key services—health, education, housing, transport, care, water/energy—through public provision or social rights rather than cash transfers. UBS centres decommodification and collective infrastructure, but requires fiscal capacity, democratic control, and organised labour to prevent deterioration or capture.}
}

\newglossaryentry{jg-term}{
  name={Job Guarantee (JG)},
  sort={Job Guarantee},
  description={A proposal that the state offers a public job at a socially defined wage to anyone willing to work. Advocates treat it as an employment floor and stabiliser; critics debate job quality, political control, and whether it can be insulated from austerity and patronage without strong democratic governance.}
}

\newglossaryentry{esg-term}{
  name={ESG},
  sort={ESG},
  description={A framework used by investors and firms to score “environmental, social, and governance” performance. ESG can pressure disclosure and some standards, but it is often criticised as compatible with continued extraction and financialisation, turning ecological crisis into a portfolio and reputational management problem.}
}

\newglossaryentry{troika}{
  name={the Troika},
  sort={Troika},
  description={A term commonly used for the European Commission (EC), the European Central Bank (ECB), and the IMF acting jointly in crisis programmes and conditional lending in the Eurozone.}
}

\newglossaryentry{eurozone}{
  name={Eurozone},
  sort={Eurozone},
  description={The group of EU member states using the euro. Eurozone membership removes independent monetary policy and exchange-rate adjustment, making fiscal policy and wage/price dynamics central sites of “internal devaluation” during crises.}
}

\newglossaryentry{syriza}{
  name={Syriza},
  sort={Syriza},
  description={A Greek left party (Coalition of the Radical Left) that came to power in 2015 on an anti-austerity mandate. Its confrontation with the Troika became a major reference point for debates on debt, monetary sovereignty, and the limits imposed by Eurozone institutions.}
}

% ----------------------------------------------------------------------
% 2) PRINTING MODE (when \input{glossary} is called in content.tex)
% ----------------------------------------------------------------------
\else

\section{Glossary and acronyms}
\label{sec:glossary}

\begingroup
\setlength{\parskip}{0pt}

% Print even if entries were not referenced yet:
\glsaddallunused

% Acronyms
\printnoidxglossary[type=\acronymtype,style=compactgls,title={Acronyms}]

\vspace{0.6\baselineskip}

% Terms
\printnoidxglossary[style=termscolon,title={Glossary of terms}]

\endgroup

\fi is used in content.tex.
%
%  IMPORTANT (preamble.tex):
%    \usepackage[acronym]{glossaries-extra}
%    \makenoidxglossaries
%    \def\GLOSSARYENTRIESONLY{}%
%    \loadglsentries{glossary}
%    \let\GLOSSARYENTRIESONLY\undefined
% ============================

\ProvidesFile{glossary.tex}[Glossary and acronym entries + Appendix printing]

% ----------------------------------------------------------------------
% 1) ENTRIES-ONLY MODE (when loaded in the preamble)
% ----------------------------------------------------------------------
\ifdefined\GLOSSARYENTRIESONLY

% ---------- Acronyms ----------
\newacronym{ai}{AI}{artificial intelligence}
\newacronym{ltv}{LTV}{labour theory of value}
\newacronym{snlt}{SNLT}{socially necessary labour time}
\newacronym{melt}{MELT}{monetary expression of labour time}
\newacronym{occ}{OCC}{organic composition of capital}
\newacronym{trpf}{TRPF}{tendency of the rate of profit to fall}

\newacronym{ubi}{UBI}{universal basic income}
\newacronym{ubs}{UBS}{universal basic services}
\newacronym{qe}{QE}{quantitative easing}
\newacronym{mmt}{MMT}{Modern Monetary Theory}
\newacronym{jg}{JG}{Job Guarantee}
\newacronym{esg}{ESG}{environmental, social, and governance}

\newacronym{imf}{IMF}{International Monetary Fund}
\newacronym{ipcc}{IPCC}{Intergovernmental Panel on Climate Change}

% ---------- Glossary terms (existing) ----------
\newglossaryentry{keynesianism}{
  name={Keynesianism},
  sort={Keynesianism},
  description={A macroeconomic approach associated with John Maynard Keynes that emphasises stabilising output and employment via demand management (especially fiscal policy and public spending), particularly during downturns.},
  first={Keynesianism (a macroeconomic approach that emphasises demand management—especially fiscal policy and public spending—to stabilise output and employment)}
}

\newglossaryentry{monetarism}{
  name={Monetarism},
  sort={Monetarism},
  description={A macroeconomic doctrine associated with Milton Friedman that prioritises controlling inflation by managing the money supply and/or interest rates, often via tight monetary policy.},
  first={Monetarism (a macroeconomic doctrine that prioritises inflation control via tight monetary policy—money supply and/or interest rates)}
}

\newglossaryentry{financialisation}{
  name={financialisation},
  sort={financialisation},
  description={A pattern in which profits, strategies, and power shift toward finance, asset price inflation, and rent extraction, rather than expanded productive investment and wage growth.}
}

\newglossaryentry{fictitious-capital}{
  name={fictitious capital},
  sort={fictitious capital},
  description={Tradable claims on future income streams (shares, bonds, securitised claims, many derivatives) whose market valuation can expand beyond surplus value currently produced, until crises force devaluation.}
}

\newglossaryentry{decommodification}{
  name={decommodification},
  sort={decommodification},
  description={Shifting access to essentials (housing, health, care, transport, energy, water) out of the market and away from ability to pay, toward rights-based provision.}
}

\newglossaryentry{capital-controls}{
  name={capital controls},
  sort={capital controls},
  description={Regulatory restrictions on cross-border movement of capital designed to limit capital flight, currency pressure, and the ability of owners to discipline reforms through financial exit.}
}

% ---------- Glossary terms (extended, targeted additions) ----------
\newglossaryentry{austerity}{
  name={austerity},
  sort={austerity},
  description={A policy package of spending cuts, hiring freezes, welfare retrenchment, regressive taxation, and/or user fees justified as “fiscal discipline”. In practice it often shifts crisis costs onto workers and the poor while protecting creditors and asset owners.}
}

\newglossaryentry{fiscal-consolidation}{
  name={fiscal consolidation},
  sort={fiscal consolidation},
  description={Reducing government deficits through spending cuts and/or tax rises. It is frequently presented as technocratic “budget repair,” but its class content depends on who is taxed, which services are cut, and whether interest payments to creditors are ring-fenced.}
}

\newglossaryentry{primary-balance}{
  name={primary balance},
  sort={primary balance},
  description={A government’s fiscal balance excluding interest payments on existing debt. A “primary surplus” can coexist with rising total debt burdens if interest costs remain high or growth is weak.}
}

\newglossaryentry{debt-service}{
  name={debt servicing},
  sort={debt servicing},
  description={Ongoing payments of interest and principal on debt. For many states, debt service becomes a prior claim on public revenue, structurally pressuring social spending and investment even without an explicit austerity programme.}
}

\newglossaryentry{conditionality}{
  name={conditionality},
  sort={conditionality},
  description={Policy conditions attached to loans or debt restructuring (commonly by the IMF or creditor blocs). Typical conditions include subsidy cuts, wage restraint, privatisation, deregulation, central bank “independence,” and fiscal consolidation.}
}

\newglossaryentry{structural-adjustment}{
  name={structural adjustment},
  sort={structural adjustment},
  description={A reform programme—historically associated with IMF/World Bank lending—that restructures economies toward export orientation, market pricing, privatisation, and reduced public provision. The “adjustment” is usually borne through depressed wages, weakened labour protections, and reduced social spending.}
}

\newglossaryentry{balance-of-payments}{
  name={balance of payments},
  sort={balance of payments},
  description={A country’s accounting of transactions with the rest of the world (trade in goods/services, income flows, and financial transfers). Persistent deficits often create pressure for devaluation, import compression, and external borrowing.}
}

\newglossaryentry{capital-flight}{
  name={capital flight},
  sort={capital flight},
  description={Rapid private movement of funds out of a country (or out of domestic investment into safer assets), often triggered by crisis, political conflict, or expectations of devaluation. Capital flight can force currency pressure, reserve loss, and harsher adjustment.}
}

\newglossaryentry{exchange-rate-pass-through}{
  name={exchange-rate pass-through},
  sort={exchange rate pass-through},
  description={The extent to which a currency devaluation raises domestic prices, especially for imported essentials (fuel, fertiliser, medicine, machinery). High pass-through can turn devaluation into immediate inflation and real-wage cuts.}
}

\newglossaryentry{inflation-targeting}{
  name={inflation targeting},
  sort={inflation targeting},
  description={A monetary-policy framework where the central bank prioritises hitting an inflation target, typically via interest-rate moves. Critics argue it can treat inflation as a purely monetary phenomenon while ignoring supply shocks, monopoly pricing, and import dependence.}
}

\newglossaryentry{protectionism}{
  name={protectionism},
  sort={protectionism},
  description={Using tariffs, quotas, licensing, local-content rules, or public procurement to shelter domestic producers from foreign competition. It can defend jobs and industrial capacity, but its effects depend on who controls protected firms, how prices/wages move, and whether technology/inputs are domestically available.}
}

\newglossaryentry{trade-liberalisation}{
  name={trade liberalisation},
  sort={trade liberalisation},
  description={Reducing tariffs, quotas, and other trade barriers. It is often sold as “efficiency,” but in unequal world markets it can accelerate deindustrialisation, worsen trade deficits, and deepen dependence on imported inputs and foreign currency.}
}

\newglossaryentry{import-substitution}{
  name={import-substitution industrialisation (ISI)},
  sort={import substitution industrialisation},
  description={A strategy to replace imports with domestic production through tariffs, credit allocation, industrial policy, and state procurement. ISI can build capacity, but it often hits constraints around technology, energy, foreign exchange, and class control of investment decisions.}
}

\newglossaryentry{qe-term}{
  name={Quantitative easing (QE)},
  sort={Quantitative easing},
  description={A central bank policy of purchasing government bonds and/or other financial assets to expand its balance sheet and push down longer-term interest rates. QE can stabilise financial markets, but it often inflates asset prices and does not automatically translate into productive investment or higher wages.}
}

\newglossaryentry{mmt-term}{
  name={Modern Monetary Theory (MMT)},
  sort={Modern Monetary Theory},
  description={A heterodox framework arguing that a state that issues its own currency cannot “run out” of money in the way households can; the binding constraints are real resources, productive capacity, and inflation dynamics. It stresses the role of taxation and bond issuance in managing demand, distribution, and monetary conditions rather than “funding” spending in a mechanical sense.}
}

\newglossaryentry{ubi-term}{
  name={Universal basic income (UBI)},
  sort={Universal basic income},
  description={An unconditional cash transfer to all residents or citizens. Proposals differ sharply: some are designed to replace welfare and subsidise low wages, while others are framed as an income floor that complements strong public services, labour rights, and decommodification.}
}

\newglossaryentry{ubs-term}{
  name={Universal basic services (UBS)},
  sort={Universal basic services},
  description={A model of guaranteeing key services—health, education, housing, transport, care, water/energy—through public provision or social rights rather than cash transfers. UBS centres decommodification and collective infrastructure, but requires fiscal capacity, democratic control, and organised labour to prevent deterioration or capture.}
}

\newglossaryentry{jg-term}{
  name={Job Guarantee (JG)},
  sort={Job Guarantee},
  description={A proposal that the state offers a public job at a socially defined wage to anyone willing to work. Advocates treat it as an employment floor and stabiliser; critics debate job quality, political control, and whether it can be insulated from austerity and patronage without strong democratic governance.}
}

\newglossaryentry{esg-term}{
  name={ESG},
  sort={ESG},
  description={A framework used by investors and firms to score “environmental, social, and governance” performance. ESG can pressure disclosure and some standards, but it is often criticised as compatible with continued extraction and financialisation, turning ecological crisis into a portfolio and reputational management problem.}
}

\newglossaryentry{troika}{
  name={the Troika},
  sort={Troika},
  description={A term commonly used for the European Commission (EC), the European Central Bank (ECB), and the IMF acting jointly in crisis programmes and conditional lending in the Eurozone.}
}

\newglossaryentry{eurozone}{
  name={Eurozone},
  sort={Eurozone},
  description={The group of EU member states using the euro. Eurozone membership removes independent monetary policy and exchange-rate adjustment, making fiscal policy and wage/price dynamics central sites of “internal devaluation” during crises.}
}

\newglossaryentry{syriza}{
  name={Syriza},
  sort={Syriza},
  description={A Greek left party (Coalition of the Radical Left) that came to power in 2015 on an anti-austerity mandate. Its confrontation with the Troika became a major reference point for debates on debt, monetary sovereignty, and the limits imposed by Eurozone institutions.}
}

% ----------------------------------------------------------------------
% 2) PRINTING MODE (when % ============================
%  glossary.tex
%  Acronyms + glossary entries (glossaries-extra, no-index workflow)
%  ALSO prints Appendix B when \input{glossary} is used in content.tex.
%
%  IMPORTANT (preamble.tex):
%    \usepackage[acronym]{glossaries-extra}
%    \makenoidxglossaries
%    \def\GLOSSARYENTRIESONLY{}%
%    \loadglsentries{glossary}
%    \let\GLOSSARYENTRIESONLY\undefined
% ============================

\ProvidesFile{glossary.tex}[Glossary and acronym entries + Appendix printing]

% ----------------------------------------------------------------------
% 1) ENTRIES-ONLY MODE (when loaded in the preamble)
% ----------------------------------------------------------------------
\ifdefined\GLOSSARYENTRIESONLY

% ---------- Acronyms ----------
\newacronym{ai}{AI}{artificial intelligence}
\newacronym{ltv}{LTV}{labour theory of value}
\newacronym{snlt}{SNLT}{socially necessary labour time}
\newacronym{melt}{MELT}{monetary expression of labour time}
\newacronym{occ}{OCC}{organic composition of capital}
\newacronym{trpf}{TRPF}{tendency of the rate of profit to fall}

\newacronym{ubi}{UBI}{universal basic income}
\newacronym{ubs}{UBS}{universal basic services}
\newacronym{qe}{QE}{quantitative easing}
\newacronym{mmt}{MMT}{Modern Monetary Theory}
\newacronym{jg}{JG}{Job Guarantee}
\newacronym{esg}{ESG}{environmental, social, and governance}

\newacronym{imf}{IMF}{International Monetary Fund}
\newacronym{ipcc}{IPCC}{Intergovernmental Panel on Climate Change}

% ---------- Glossary terms (existing) ----------
\newglossaryentry{keynesianism}{
  name={Keynesianism},
  sort={Keynesianism},
  description={A macroeconomic approach associated with John Maynard Keynes that emphasises stabilising output and employment via demand management (especially fiscal policy and public spending), particularly during downturns.},
  first={Keynesianism (a macroeconomic approach that emphasises demand management—especially fiscal policy and public spending—to stabilise output and employment)}
}

\newglossaryentry{monetarism}{
  name={Monetarism},
  sort={Monetarism},
  description={A macroeconomic doctrine associated with Milton Friedman that prioritises controlling inflation by managing the money supply and/or interest rates, often via tight monetary policy.},
  first={Monetarism (a macroeconomic doctrine that prioritises inflation control via tight monetary policy—money supply and/or interest rates)}
}

\newglossaryentry{financialisation}{
  name={financialisation},
  sort={financialisation},
  description={A pattern in which profits, strategies, and power shift toward finance, asset price inflation, and rent extraction, rather than expanded productive investment and wage growth.}
}

\newglossaryentry{fictitious-capital}{
  name={fictitious capital},
  sort={fictitious capital},
  description={Tradable claims on future income streams (shares, bonds, securitised claims, many derivatives) whose market valuation can expand beyond surplus value currently produced, until crises force devaluation.}
}

\newglossaryentry{decommodification}{
  name={decommodification},
  sort={decommodification},
  description={Shifting access to essentials (housing, health, care, transport, energy, water) out of the market and away from ability to pay, toward rights-based provision.}
}

\newglossaryentry{capital-controls}{
  name={capital controls},
  sort={capital controls},
  description={Regulatory restrictions on cross-border movement of capital designed to limit capital flight, currency pressure, and the ability of owners to discipline reforms through financial exit.}
}

% ---------- Glossary terms (extended, targeted additions) ----------
\newglossaryentry{austerity}{
  name={austerity},
  sort={austerity},
  description={A policy package of spending cuts, hiring freezes, welfare retrenchment, regressive taxation, and/or user fees justified as “fiscal discipline”. In practice it often shifts crisis costs onto workers and the poor while protecting creditors and asset owners.}
}

\newglossaryentry{fiscal-consolidation}{
  name={fiscal consolidation},
  sort={fiscal consolidation},
  description={Reducing government deficits through spending cuts and/or tax rises. It is frequently presented as technocratic “budget repair,” but its class content depends on who is taxed, which services are cut, and whether interest payments to creditors are ring-fenced.}
}

\newglossaryentry{primary-balance}{
  name={primary balance},
  sort={primary balance},
  description={A government’s fiscal balance excluding interest payments on existing debt. A “primary surplus” can coexist with rising total debt burdens if interest costs remain high or growth is weak.}
}

\newglossaryentry{debt-service}{
  name={debt servicing},
  sort={debt servicing},
  description={Ongoing payments of interest and principal on debt. For many states, debt service becomes a prior claim on public revenue, structurally pressuring social spending and investment even without an explicit austerity programme.}
}

\newglossaryentry{conditionality}{
  name={conditionality},
  sort={conditionality},
  description={Policy conditions attached to loans or debt restructuring (commonly by the IMF or creditor blocs). Typical conditions include subsidy cuts, wage restraint, privatisation, deregulation, central bank “independence,” and fiscal consolidation.}
}

\newglossaryentry{structural-adjustment}{
  name={structural adjustment},
  sort={structural adjustment},
  description={A reform programme—historically associated with IMF/World Bank lending—that restructures economies toward export orientation, market pricing, privatisation, and reduced public provision. The “adjustment” is usually borne through depressed wages, weakened labour protections, and reduced social spending.}
}

\newglossaryentry{balance-of-payments}{
  name={balance of payments},
  sort={balance of payments},
  description={A country’s accounting of transactions with the rest of the world (trade in goods/services, income flows, and financial transfers). Persistent deficits often create pressure for devaluation, import compression, and external borrowing.}
}

\newglossaryentry{capital-flight}{
  name={capital flight},
  sort={capital flight},
  description={Rapid private movement of funds out of a country (or out of domestic investment into safer assets), often triggered by crisis, political conflict, or expectations of devaluation. Capital flight can force currency pressure, reserve loss, and harsher adjustment.}
}

\newglossaryentry{exchange-rate-pass-through}{
  name={exchange-rate pass-through},
  sort={exchange rate pass-through},
  description={The extent to which a currency devaluation raises domestic prices, especially for imported essentials (fuel, fertiliser, medicine, machinery). High pass-through can turn devaluation into immediate inflation and real-wage cuts.}
}

\newglossaryentry{inflation-targeting}{
  name={inflation targeting},
  sort={inflation targeting},
  description={A monetary-policy framework where the central bank prioritises hitting an inflation target, typically via interest-rate moves. Critics argue it can treat inflation as a purely monetary phenomenon while ignoring supply shocks, monopoly pricing, and import dependence.}
}

\newglossaryentry{protectionism}{
  name={protectionism},
  sort={protectionism},
  description={Using tariffs, quotas, licensing, local-content rules, or public procurement to shelter domestic producers from foreign competition. It can defend jobs and industrial capacity, but its effects depend on who controls protected firms, how prices/wages move, and whether technology/inputs are domestically available.}
}

\newglossaryentry{trade-liberalisation}{
  name={trade liberalisation},
  sort={trade liberalisation},
  description={Reducing tariffs, quotas, and other trade barriers. It is often sold as “efficiency,” but in unequal world markets it can accelerate deindustrialisation, worsen trade deficits, and deepen dependence on imported inputs and foreign currency.}
}

\newglossaryentry{import-substitution}{
  name={import-substitution industrialisation (ISI)},
  sort={import substitution industrialisation},
  description={A strategy to replace imports with domestic production through tariffs, credit allocation, industrial policy, and state procurement. ISI can build capacity, but it often hits constraints around technology, energy, foreign exchange, and class control of investment decisions.}
}

\newglossaryentry{qe-term}{
  name={Quantitative easing (QE)},
  sort={Quantitative easing},
  description={A central bank policy of purchasing government bonds and/or other financial assets to expand its balance sheet and push down longer-term interest rates. QE can stabilise financial markets, but it often inflates asset prices and does not automatically translate into productive investment or higher wages.}
}

\newglossaryentry{mmt-term}{
  name={Modern Monetary Theory (MMT)},
  sort={Modern Monetary Theory},
  description={A heterodox framework arguing that a state that issues its own currency cannot “run out” of money in the way households can; the binding constraints are real resources, productive capacity, and inflation dynamics. It stresses the role of taxation and bond issuance in managing demand, distribution, and monetary conditions rather than “funding” spending in a mechanical sense.}
}

\newglossaryentry{ubi-term}{
  name={Universal basic income (UBI)},
  sort={Universal basic income},
  description={An unconditional cash transfer to all residents or citizens. Proposals differ sharply: some are designed to replace welfare and subsidise low wages, while others are framed as an income floor that complements strong public services, labour rights, and decommodification.}
}

\newglossaryentry{ubs-term}{
  name={Universal basic services (UBS)},
  sort={Universal basic services},
  description={A model of guaranteeing key services—health, education, housing, transport, care, water/energy—through public provision or social rights rather than cash transfers. UBS centres decommodification and collective infrastructure, but requires fiscal capacity, democratic control, and organised labour to prevent deterioration or capture.}
}

\newglossaryentry{jg-term}{
  name={Job Guarantee (JG)},
  sort={Job Guarantee},
  description={A proposal that the state offers a public job at a socially defined wage to anyone willing to work. Advocates treat it as an employment floor and stabiliser; critics debate job quality, political control, and whether it can be insulated from austerity and patronage without strong democratic governance.}
}

\newglossaryentry{esg-term}{
  name={ESG},
  sort={ESG},
  description={A framework used by investors and firms to score “environmental, social, and governance” performance. ESG can pressure disclosure and some standards, but it is often criticised as compatible with continued extraction and financialisation, turning ecological crisis into a portfolio and reputational management problem.}
}

\newglossaryentry{troika}{
  name={the Troika},
  sort={Troika},
  description={A term commonly used for the European Commission (EC), the European Central Bank (ECB), and the IMF acting jointly in crisis programmes and conditional lending in the Eurozone.}
}

\newglossaryentry{eurozone}{
  name={Eurozone},
  sort={Eurozone},
  description={The group of EU member states using the euro. Eurozone membership removes independent monetary policy and exchange-rate adjustment, making fiscal policy and wage/price dynamics central sites of “internal devaluation” during crises.}
}

\newglossaryentry{syriza}{
  name={Syriza},
  sort={Syriza},
  description={A Greek left party (Coalition of the Radical Left) that came to power in 2015 on an anti-austerity mandate. Its confrontation with the Troika became a major reference point for debates on debt, monetary sovereignty, and the limits imposed by Eurozone institutions.}
}

% ----------------------------------------------------------------------
% 2) PRINTING MODE (when \input{glossary} is called in content.tex)
% ----------------------------------------------------------------------
\else

\section{Glossary and acronyms}
\label{sec:glossary}

\begingroup
\setlength{\parskip}{0pt}

% Print even if entries were not referenced yet:
\glsaddallunused

% Acronyms
\printnoidxglossary[type=\acronymtype,style=compactgls,title={Acronyms}]

\vspace{0.6\baselineskip}

% Terms
\printnoidxglossary[style=termscolon,title={Glossary of terms}]

\endgroup

\fi is called in content.tex)
% ----------------------------------------------------------------------
\else

\section{Glossary and acronyms}
\label{sec:glossary}

\begingroup
\setlength{\parskip}{0pt}

% Print even if entries were not referenced yet:
\glsaddallunused

% Acronyms
\printnoidxglossary[type=\acronymtype,style=compactgls,title={Acronyms}]

\vspace{0.6\baselineskip}

% Terms
\printnoidxglossary[style=termscolon,title={Glossary of terms}]

\endgroup

\fi is used in content.tex.
%
%  IMPORTANT (preamble.tex):
%    \usepackage[acronym]{glossaries-extra}
%    \makenoidxglossaries
%    \def\GLOSSARYENTRIESONLY{}%
%    \loadglsentries{glossary}
%    \let\GLOSSARYENTRIESONLY\undefined
% ============================

\ProvidesFile{glossary.tex}[Glossary and acronym entries + Appendix printing]

% ----------------------------------------------------------------------
% 1) ENTRIES-ONLY MODE (when loaded in the preamble)
% ----------------------------------------------------------------------
\ifdefined\GLOSSARYENTRIESONLY

% ---------- Acronyms ----------
\newacronym{ai}{AI}{artificial intelligence}
\newacronym{ltv}{LTV}{labour theory of value}
\newacronym{snlt}{SNLT}{socially necessary labour time}
\newacronym{melt}{MELT}{monetary expression of labour time}
\newacronym{occ}{OCC}{organic composition of capital}
\newacronym{trpf}{TRPF}{tendency of the rate of profit to fall}

\newacronym{ubi}{UBI}{universal basic income}
\newacronym{ubs}{UBS}{universal basic services}
\newacronym{qe}{QE}{quantitative easing}
\newacronym{mmt}{MMT}{Modern Monetary Theory}
\newacronym{jg}{JG}{Job Guarantee}
\newacronym{esg}{ESG}{environmental, social, and governance}

\newacronym{imf}{IMF}{International Monetary Fund}
\newacronym{ipcc}{IPCC}{Intergovernmental Panel on Climate Change}

% ---------- Glossary terms (existing) ----------
\newglossaryentry{keynesianism}{
  name={Keynesianism},
  sort={Keynesianism},
  description={A macroeconomic approach associated with John Maynard Keynes that emphasises stabilising output and employment via demand management (especially fiscal policy and public spending), particularly during downturns.},
  first={Keynesianism (a macroeconomic approach that emphasises demand management—especially fiscal policy and public spending—to stabilise output and employment)}
}

\newglossaryentry{monetarism}{
  name={Monetarism},
  sort={Monetarism},
  description={A macroeconomic doctrine associated with Milton Friedman that prioritises controlling inflation by managing the money supply and/or interest rates, often via tight monetary policy.},
  first={Monetarism (a macroeconomic doctrine that prioritises inflation control via tight monetary policy—money supply and/or interest rates)}
}

\newglossaryentry{financialisation}{
  name={financialisation},
  sort={financialisation},
  description={A pattern in which profits, strategies, and power shift toward finance, asset price inflation, and rent extraction, rather than expanded productive investment and wage growth.}
}

\newglossaryentry{fictitious-capital}{
  name={fictitious capital},
  sort={fictitious capital},
  description={Tradable claims on future income streams (shares, bonds, securitised claims, many derivatives) whose market valuation can expand beyond surplus value currently produced, until crises force devaluation.}
}

\newglossaryentry{decommodification}{
  name={decommodification},
  sort={decommodification},
  description={Shifting access to essentials (housing, health, care, transport, energy, water) out of the market and away from ability to pay, toward rights-based provision.}
}

\newglossaryentry{capital-controls}{
  name={capital controls},
  sort={capital controls},
  description={Regulatory restrictions on cross-border movement of capital designed to limit capital flight, currency pressure, and the ability of owners to discipline reforms through financial exit.}
}

% ---------- Glossary terms (extended, targeted additions) ----------
\newglossaryentry{austerity}{
  name={austerity},
  sort={austerity},
  description={A policy package of spending cuts, hiring freezes, welfare retrenchment, regressive taxation, and/or user fees justified as “fiscal discipline”. In practice it often shifts crisis costs onto workers and the poor while protecting creditors and asset owners.}
}

\newglossaryentry{fiscal-consolidation}{
  name={fiscal consolidation},
  sort={fiscal consolidation},
  description={Reducing government deficits through spending cuts and/or tax rises. It is frequently presented as technocratic “budget repair,” but its class content depends on who is taxed, which services are cut, and whether interest payments to creditors are ring-fenced.}
}

\newglossaryentry{primary-balance}{
  name={primary balance},
  sort={primary balance},
  description={A government’s fiscal balance excluding interest payments on existing debt. A “primary surplus” can coexist with rising total debt burdens if interest costs remain high or growth is weak.}
}

\newglossaryentry{debt-service}{
  name={debt servicing},
  sort={debt servicing},
  description={Ongoing payments of interest and principal on debt. For many states, debt service becomes a prior claim on public revenue, structurally pressuring social spending and investment even without an explicit austerity programme.}
}

\newglossaryentry{conditionality}{
  name={conditionality},
  sort={conditionality},
  description={Policy conditions attached to loans or debt restructuring (commonly by the IMF or creditor blocs). Typical conditions include subsidy cuts, wage restraint, privatisation, deregulation, central bank “independence,” and fiscal consolidation.}
}

\newglossaryentry{structural-adjustment}{
  name={structural adjustment},
  sort={structural adjustment},
  description={A reform programme—historically associated with IMF/World Bank lending—that restructures economies toward export orientation, market pricing, privatisation, and reduced public provision. The “adjustment” is usually borne through depressed wages, weakened labour protections, and reduced social spending.}
}

\newglossaryentry{balance-of-payments}{
  name={balance of payments},
  sort={balance of payments},
  description={A country’s accounting of transactions with the rest of the world (trade in goods/services, income flows, and financial transfers). Persistent deficits often create pressure for devaluation, import compression, and external borrowing.}
}

\newglossaryentry{capital-flight}{
  name={capital flight},
  sort={capital flight},
  description={Rapid private movement of funds out of a country (or out of domestic investment into safer assets), often triggered by crisis, political conflict, or expectations of devaluation. Capital flight can force currency pressure, reserve loss, and harsher adjustment.}
}

\newglossaryentry{exchange-rate-pass-through}{
  name={exchange-rate pass-through},
  sort={exchange rate pass-through},
  description={The extent to which a currency devaluation raises domestic prices, especially for imported essentials (fuel, fertiliser, medicine, machinery). High pass-through can turn devaluation into immediate inflation and real-wage cuts.}
}

\newglossaryentry{inflation-targeting}{
  name={inflation targeting},
  sort={inflation targeting},
  description={A monetary-policy framework where the central bank prioritises hitting an inflation target, typically via interest-rate moves. Critics argue it can treat inflation as a purely monetary phenomenon while ignoring supply shocks, monopoly pricing, and import dependence.}
}

\newglossaryentry{protectionism}{
  name={protectionism},
  sort={protectionism},
  description={Using tariffs, quotas, licensing, local-content rules, or public procurement to shelter domestic producers from foreign competition. It can defend jobs and industrial capacity, but its effects depend on who controls protected firms, how prices/wages move, and whether technology/inputs are domestically available.}
}

\newglossaryentry{trade-liberalisation}{
  name={trade liberalisation},
  sort={trade liberalisation},
  description={Reducing tariffs, quotas, and other trade barriers. It is often sold as “efficiency,” but in unequal world markets it can accelerate deindustrialisation, worsen trade deficits, and deepen dependence on imported inputs and foreign currency.}
}

\newglossaryentry{import-substitution}{
  name={import-substitution industrialisation (ISI)},
  sort={import substitution industrialisation},
  description={A strategy to replace imports with domestic production through tariffs, credit allocation, industrial policy, and state procurement. ISI can build capacity, but it often hits constraints around technology, energy, foreign exchange, and class control of investment decisions.}
}

\newglossaryentry{qe-term}{
  name={Quantitative easing (QE)},
  sort={Quantitative easing},
  description={A central bank policy of purchasing government bonds and/or other financial assets to expand its balance sheet and push down longer-term interest rates. QE can stabilise financial markets, but it often inflates asset prices and does not automatically translate into productive investment or higher wages.}
}

\newglossaryentry{mmt-term}{
  name={Modern Monetary Theory (MMT)},
  sort={Modern Monetary Theory},
  description={A heterodox framework arguing that a state that issues its own currency cannot “run out” of money in the way households can; the binding constraints are real resources, productive capacity, and inflation dynamics. It stresses the role of taxation and bond issuance in managing demand, distribution, and monetary conditions rather than “funding” spending in a mechanical sense.}
}

\newglossaryentry{ubi-term}{
  name={Universal basic income (UBI)},
  sort={Universal basic income},
  description={An unconditional cash transfer to all residents or citizens. Proposals differ sharply: some are designed to replace welfare and subsidise low wages, while others are framed as an income floor that complements strong public services, labour rights, and decommodification.}
}

\newglossaryentry{ubs-term}{
  name={Universal basic services (UBS)},
  sort={Universal basic services},
  description={A model of guaranteeing key services—health, education, housing, transport, care, water/energy—through public provision or social rights rather than cash transfers. UBS centres decommodification and collective infrastructure, but requires fiscal capacity, democratic control, and organised labour to prevent deterioration or capture.}
}

\newglossaryentry{jg-term}{
  name={Job Guarantee (JG)},
  sort={Job Guarantee},
  description={A proposal that the state offers a public job at a socially defined wage to anyone willing to work. Advocates treat it as an employment floor and stabiliser; critics debate job quality, political control, and whether it can be insulated from austerity and patronage without strong democratic governance.}
}

\newglossaryentry{esg-term}{
  name={ESG},
  sort={ESG},
  description={A framework used by investors and firms to score “environmental, social, and governance” performance. ESG can pressure disclosure and some standards, but it is often criticised as compatible with continued extraction and financialisation, turning ecological crisis into a portfolio and reputational management problem.}
}

\newglossaryentry{troika}{
  name={the Troika},
  sort={Troika},
  description={A term commonly used for the European Commission (EC), the European Central Bank (ECB), and the IMF acting jointly in crisis programmes and conditional lending in the Eurozone.}
}

\newglossaryentry{eurozone}{
  name={Eurozone},
  sort={Eurozone},
  description={The group of EU member states using the euro. Eurozone membership removes independent monetary policy and exchange-rate adjustment, making fiscal policy and wage/price dynamics central sites of “internal devaluation” during crises.}
}

\newglossaryentry{syriza}{
  name={Syriza},
  sort={Syriza},
  description={A Greek left party (Coalition of the Radical Left) that came to power in 2015 on an anti-austerity mandate. Its confrontation with the Troika became a major reference point for debates on debt, monetary sovereignty, and the limits imposed by Eurozone institutions.}
}

% ----------------------------------------------------------------------
% 2) PRINTING MODE (when % ============================
%  glossary.tex
%  Acronyms + glossary entries (glossaries-extra, no-index workflow)
%  ALSO prints Appendix B when % ============================
%  glossary.tex
%  Acronyms + glossary entries (glossaries-extra, no-index workflow)
%  ALSO prints Appendix B when \input{glossary} is used in content.tex.
%
%  IMPORTANT (preamble.tex):
%    \usepackage[acronym]{glossaries-extra}
%    \makenoidxglossaries
%    \def\GLOSSARYENTRIESONLY{}%
%    \loadglsentries{glossary}
%    \let\GLOSSARYENTRIESONLY\undefined
% ============================

\ProvidesFile{glossary.tex}[Glossary and acronym entries + Appendix printing]

% ----------------------------------------------------------------------
% 1) ENTRIES-ONLY MODE (when loaded in the preamble)
% ----------------------------------------------------------------------
\ifdefined\GLOSSARYENTRIESONLY

% ---------- Acronyms ----------
\newacronym{ai}{AI}{artificial intelligence}
\newacronym{ltv}{LTV}{labour theory of value}
\newacronym{snlt}{SNLT}{socially necessary labour time}
\newacronym{melt}{MELT}{monetary expression of labour time}
\newacronym{occ}{OCC}{organic composition of capital}
\newacronym{trpf}{TRPF}{tendency of the rate of profit to fall}

\newacronym{ubi}{UBI}{universal basic income}
\newacronym{ubs}{UBS}{universal basic services}
\newacronym{qe}{QE}{quantitative easing}
\newacronym{mmt}{MMT}{Modern Monetary Theory}
\newacronym{jg}{JG}{Job Guarantee}
\newacronym{esg}{ESG}{environmental, social, and governance}

\newacronym{imf}{IMF}{International Monetary Fund}
\newacronym{ipcc}{IPCC}{Intergovernmental Panel on Climate Change}

% ---------- Glossary terms (existing) ----------
\newglossaryentry{keynesianism}{
  name={Keynesianism},
  sort={Keynesianism},
  description={A macroeconomic approach associated with John Maynard Keynes that emphasises stabilising output and employment via demand management (especially fiscal policy and public spending), particularly during downturns.},
  first={Keynesianism (a macroeconomic approach that emphasises demand management—especially fiscal policy and public spending—to stabilise output and employment)}
}

\newglossaryentry{monetarism}{
  name={Monetarism},
  sort={Monetarism},
  description={A macroeconomic doctrine associated with Milton Friedman that prioritises controlling inflation by managing the money supply and/or interest rates, often via tight monetary policy.},
  first={Monetarism (a macroeconomic doctrine that prioritises inflation control via tight monetary policy—money supply and/or interest rates)}
}

\newglossaryentry{financialisation}{
  name={financialisation},
  sort={financialisation},
  description={A pattern in which profits, strategies, and power shift toward finance, asset price inflation, and rent extraction, rather than expanded productive investment and wage growth.}
}

\newglossaryentry{fictitious-capital}{
  name={fictitious capital},
  sort={fictitious capital},
  description={Tradable claims on future income streams (shares, bonds, securitised claims, many derivatives) whose market valuation can expand beyond surplus value currently produced, until crises force devaluation.}
}

\newglossaryentry{decommodification}{
  name={decommodification},
  sort={decommodification},
  description={Shifting access to essentials (housing, health, care, transport, energy, water) out of the market and away from ability to pay, toward rights-based provision.}
}

\newglossaryentry{capital-controls}{
  name={capital controls},
  sort={capital controls},
  description={Regulatory restrictions on cross-border movement of capital designed to limit capital flight, currency pressure, and the ability of owners to discipline reforms through financial exit.}
}

% ---------- Glossary terms (extended, targeted additions) ----------
\newglossaryentry{austerity}{
  name={austerity},
  sort={austerity},
  description={A policy package of spending cuts, hiring freezes, welfare retrenchment, regressive taxation, and/or user fees justified as “fiscal discipline”. In practice it often shifts crisis costs onto workers and the poor while protecting creditors and asset owners.}
}

\newglossaryentry{fiscal-consolidation}{
  name={fiscal consolidation},
  sort={fiscal consolidation},
  description={Reducing government deficits through spending cuts and/or tax rises. It is frequently presented as technocratic “budget repair,” but its class content depends on who is taxed, which services are cut, and whether interest payments to creditors are ring-fenced.}
}

\newglossaryentry{primary-balance}{
  name={primary balance},
  sort={primary balance},
  description={A government’s fiscal balance excluding interest payments on existing debt. A “primary surplus” can coexist with rising total debt burdens if interest costs remain high or growth is weak.}
}

\newglossaryentry{debt-service}{
  name={debt servicing},
  sort={debt servicing},
  description={Ongoing payments of interest and principal on debt. For many states, debt service becomes a prior claim on public revenue, structurally pressuring social spending and investment even without an explicit austerity programme.}
}

\newglossaryentry{conditionality}{
  name={conditionality},
  sort={conditionality},
  description={Policy conditions attached to loans or debt restructuring (commonly by the IMF or creditor blocs). Typical conditions include subsidy cuts, wage restraint, privatisation, deregulation, central bank “independence,” and fiscal consolidation.}
}

\newglossaryentry{structural-adjustment}{
  name={structural adjustment},
  sort={structural adjustment},
  description={A reform programme—historically associated with IMF/World Bank lending—that restructures economies toward export orientation, market pricing, privatisation, and reduced public provision. The “adjustment” is usually borne through depressed wages, weakened labour protections, and reduced social spending.}
}

\newglossaryentry{balance-of-payments}{
  name={balance of payments},
  sort={balance of payments},
  description={A country’s accounting of transactions with the rest of the world (trade in goods/services, income flows, and financial transfers). Persistent deficits often create pressure for devaluation, import compression, and external borrowing.}
}

\newglossaryentry{capital-flight}{
  name={capital flight},
  sort={capital flight},
  description={Rapid private movement of funds out of a country (or out of domestic investment into safer assets), often triggered by crisis, political conflict, or expectations of devaluation. Capital flight can force currency pressure, reserve loss, and harsher adjustment.}
}

\newglossaryentry{exchange-rate-pass-through}{
  name={exchange-rate pass-through},
  sort={exchange rate pass-through},
  description={The extent to which a currency devaluation raises domestic prices, especially for imported essentials (fuel, fertiliser, medicine, machinery). High pass-through can turn devaluation into immediate inflation and real-wage cuts.}
}

\newglossaryentry{inflation-targeting}{
  name={inflation targeting},
  sort={inflation targeting},
  description={A monetary-policy framework where the central bank prioritises hitting an inflation target, typically via interest-rate moves. Critics argue it can treat inflation as a purely monetary phenomenon while ignoring supply shocks, monopoly pricing, and import dependence.}
}

\newglossaryentry{protectionism}{
  name={protectionism},
  sort={protectionism},
  description={Using tariffs, quotas, licensing, local-content rules, or public procurement to shelter domestic producers from foreign competition. It can defend jobs and industrial capacity, but its effects depend on who controls protected firms, how prices/wages move, and whether technology/inputs are domestically available.}
}

\newglossaryentry{trade-liberalisation}{
  name={trade liberalisation},
  sort={trade liberalisation},
  description={Reducing tariffs, quotas, and other trade barriers. It is often sold as “efficiency,” but in unequal world markets it can accelerate deindustrialisation, worsen trade deficits, and deepen dependence on imported inputs and foreign currency.}
}

\newglossaryentry{import-substitution}{
  name={import-substitution industrialisation (ISI)},
  sort={import substitution industrialisation},
  description={A strategy to replace imports with domestic production through tariffs, credit allocation, industrial policy, and state procurement. ISI can build capacity, but it often hits constraints around technology, energy, foreign exchange, and class control of investment decisions.}
}

\newglossaryentry{qe-term}{
  name={Quantitative easing (QE)},
  sort={Quantitative easing},
  description={A central bank policy of purchasing government bonds and/or other financial assets to expand its balance sheet and push down longer-term interest rates. QE can stabilise financial markets, but it often inflates asset prices and does not automatically translate into productive investment or higher wages.}
}

\newglossaryentry{mmt-term}{
  name={Modern Monetary Theory (MMT)},
  sort={Modern Monetary Theory},
  description={A heterodox framework arguing that a state that issues its own currency cannot “run out” of money in the way households can; the binding constraints are real resources, productive capacity, and inflation dynamics. It stresses the role of taxation and bond issuance in managing demand, distribution, and monetary conditions rather than “funding” spending in a mechanical sense.}
}

\newglossaryentry{ubi-term}{
  name={Universal basic income (UBI)},
  sort={Universal basic income},
  description={An unconditional cash transfer to all residents or citizens. Proposals differ sharply: some are designed to replace welfare and subsidise low wages, while others are framed as an income floor that complements strong public services, labour rights, and decommodification.}
}

\newglossaryentry{ubs-term}{
  name={Universal basic services (UBS)},
  sort={Universal basic services},
  description={A model of guaranteeing key services—health, education, housing, transport, care, water/energy—through public provision or social rights rather than cash transfers. UBS centres decommodification and collective infrastructure, but requires fiscal capacity, democratic control, and organised labour to prevent deterioration or capture.}
}

\newglossaryentry{jg-term}{
  name={Job Guarantee (JG)},
  sort={Job Guarantee},
  description={A proposal that the state offers a public job at a socially defined wage to anyone willing to work. Advocates treat it as an employment floor and stabiliser; critics debate job quality, political control, and whether it can be insulated from austerity and patronage without strong democratic governance.}
}

\newglossaryentry{esg-term}{
  name={ESG},
  sort={ESG},
  description={A framework used by investors and firms to score “environmental, social, and governance” performance. ESG can pressure disclosure and some standards, but it is often criticised as compatible with continued extraction and financialisation, turning ecological crisis into a portfolio and reputational management problem.}
}

\newglossaryentry{troika}{
  name={the Troika},
  sort={Troika},
  description={A term commonly used for the European Commission (EC), the European Central Bank (ECB), and the IMF acting jointly in crisis programmes and conditional lending in the Eurozone.}
}

\newglossaryentry{eurozone}{
  name={Eurozone},
  sort={Eurozone},
  description={The group of EU member states using the euro. Eurozone membership removes independent monetary policy and exchange-rate adjustment, making fiscal policy and wage/price dynamics central sites of “internal devaluation” during crises.}
}

\newglossaryentry{syriza}{
  name={Syriza},
  sort={Syriza},
  description={A Greek left party (Coalition of the Radical Left) that came to power in 2015 on an anti-austerity mandate. Its confrontation with the Troika became a major reference point for debates on debt, monetary sovereignty, and the limits imposed by Eurozone institutions.}
}

% ----------------------------------------------------------------------
% 2) PRINTING MODE (when \input{glossary} is called in content.tex)
% ----------------------------------------------------------------------
\else

\section{Glossary and acronyms}
\label{sec:glossary}

\begingroup
\setlength{\parskip}{0pt}

% Print even if entries were not referenced yet:
\glsaddallunused

% Acronyms
\printnoidxglossary[type=\acronymtype,style=compactgls,title={Acronyms}]

\vspace{0.6\baselineskip}

% Terms
\printnoidxglossary[style=termscolon,title={Glossary of terms}]

\endgroup

\fi is used in content.tex.
%
%  IMPORTANT (preamble.tex):
%    \usepackage[acronym]{glossaries-extra}
%    \makenoidxglossaries
%    \def\GLOSSARYENTRIESONLY{}%
%    \loadglsentries{glossary}
%    \let\GLOSSARYENTRIESONLY\undefined
% ============================

\ProvidesFile{glossary.tex}[Glossary and acronym entries + Appendix printing]

% ----------------------------------------------------------------------
% 1) ENTRIES-ONLY MODE (when loaded in the preamble)
% ----------------------------------------------------------------------
\ifdefined\GLOSSARYENTRIESONLY

% ---------- Acronyms ----------
\newacronym{ai}{AI}{artificial intelligence}
\newacronym{ltv}{LTV}{labour theory of value}
\newacronym{snlt}{SNLT}{socially necessary labour time}
\newacronym{melt}{MELT}{monetary expression of labour time}
\newacronym{occ}{OCC}{organic composition of capital}
\newacronym{trpf}{TRPF}{tendency of the rate of profit to fall}

\newacronym{ubi}{UBI}{universal basic income}
\newacronym{ubs}{UBS}{universal basic services}
\newacronym{qe}{QE}{quantitative easing}
\newacronym{mmt}{MMT}{Modern Monetary Theory}
\newacronym{jg}{JG}{Job Guarantee}
\newacronym{esg}{ESG}{environmental, social, and governance}

\newacronym{imf}{IMF}{International Monetary Fund}
\newacronym{ipcc}{IPCC}{Intergovernmental Panel on Climate Change}

% ---------- Glossary terms (existing) ----------
\newglossaryentry{keynesianism}{
  name={Keynesianism},
  sort={Keynesianism},
  description={A macroeconomic approach associated with John Maynard Keynes that emphasises stabilising output and employment via demand management (especially fiscal policy and public spending), particularly during downturns.},
  first={Keynesianism (a macroeconomic approach that emphasises demand management—especially fiscal policy and public spending—to stabilise output and employment)}
}

\newglossaryentry{monetarism}{
  name={Monetarism},
  sort={Monetarism},
  description={A macroeconomic doctrine associated with Milton Friedman that prioritises controlling inflation by managing the money supply and/or interest rates, often via tight monetary policy.},
  first={Monetarism (a macroeconomic doctrine that prioritises inflation control via tight monetary policy—money supply and/or interest rates)}
}

\newglossaryentry{financialisation}{
  name={financialisation},
  sort={financialisation},
  description={A pattern in which profits, strategies, and power shift toward finance, asset price inflation, and rent extraction, rather than expanded productive investment and wage growth.}
}

\newglossaryentry{fictitious-capital}{
  name={fictitious capital},
  sort={fictitious capital},
  description={Tradable claims on future income streams (shares, bonds, securitised claims, many derivatives) whose market valuation can expand beyond surplus value currently produced, until crises force devaluation.}
}

\newglossaryentry{decommodification}{
  name={decommodification},
  sort={decommodification},
  description={Shifting access to essentials (housing, health, care, transport, energy, water) out of the market and away from ability to pay, toward rights-based provision.}
}

\newglossaryentry{capital-controls}{
  name={capital controls},
  sort={capital controls},
  description={Regulatory restrictions on cross-border movement of capital designed to limit capital flight, currency pressure, and the ability of owners to discipline reforms through financial exit.}
}

% ---------- Glossary terms (extended, targeted additions) ----------
\newglossaryentry{austerity}{
  name={austerity},
  sort={austerity},
  description={A policy package of spending cuts, hiring freezes, welfare retrenchment, regressive taxation, and/or user fees justified as “fiscal discipline”. In practice it often shifts crisis costs onto workers and the poor while protecting creditors and asset owners.}
}

\newglossaryentry{fiscal-consolidation}{
  name={fiscal consolidation},
  sort={fiscal consolidation},
  description={Reducing government deficits through spending cuts and/or tax rises. It is frequently presented as technocratic “budget repair,” but its class content depends on who is taxed, which services are cut, and whether interest payments to creditors are ring-fenced.}
}

\newglossaryentry{primary-balance}{
  name={primary balance},
  sort={primary balance},
  description={A government’s fiscal balance excluding interest payments on existing debt. A “primary surplus” can coexist with rising total debt burdens if interest costs remain high or growth is weak.}
}

\newglossaryentry{debt-service}{
  name={debt servicing},
  sort={debt servicing},
  description={Ongoing payments of interest and principal on debt. For many states, debt service becomes a prior claim on public revenue, structurally pressuring social spending and investment even without an explicit austerity programme.}
}

\newglossaryentry{conditionality}{
  name={conditionality},
  sort={conditionality},
  description={Policy conditions attached to loans or debt restructuring (commonly by the IMF or creditor blocs). Typical conditions include subsidy cuts, wage restraint, privatisation, deregulation, central bank “independence,” and fiscal consolidation.}
}

\newglossaryentry{structural-adjustment}{
  name={structural adjustment},
  sort={structural adjustment},
  description={A reform programme—historically associated with IMF/World Bank lending—that restructures economies toward export orientation, market pricing, privatisation, and reduced public provision. The “adjustment” is usually borne through depressed wages, weakened labour protections, and reduced social spending.}
}

\newglossaryentry{balance-of-payments}{
  name={balance of payments},
  sort={balance of payments},
  description={A country’s accounting of transactions with the rest of the world (trade in goods/services, income flows, and financial transfers). Persistent deficits often create pressure for devaluation, import compression, and external borrowing.}
}

\newglossaryentry{capital-flight}{
  name={capital flight},
  sort={capital flight},
  description={Rapid private movement of funds out of a country (or out of domestic investment into safer assets), often triggered by crisis, political conflict, or expectations of devaluation. Capital flight can force currency pressure, reserve loss, and harsher adjustment.}
}

\newglossaryentry{exchange-rate-pass-through}{
  name={exchange-rate pass-through},
  sort={exchange rate pass-through},
  description={The extent to which a currency devaluation raises domestic prices, especially for imported essentials (fuel, fertiliser, medicine, machinery). High pass-through can turn devaluation into immediate inflation and real-wage cuts.}
}

\newglossaryentry{inflation-targeting}{
  name={inflation targeting},
  sort={inflation targeting},
  description={A monetary-policy framework where the central bank prioritises hitting an inflation target, typically via interest-rate moves. Critics argue it can treat inflation as a purely monetary phenomenon while ignoring supply shocks, monopoly pricing, and import dependence.}
}

\newglossaryentry{protectionism}{
  name={protectionism},
  sort={protectionism},
  description={Using tariffs, quotas, licensing, local-content rules, or public procurement to shelter domestic producers from foreign competition. It can defend jobs and industrial capacity, but its effects depend on who controls protected firms, how prices/wages move, and whether technology/inputs are domestically available.}
}

\newglossaryentry{trade-liberalisation}{
  name={trade liberalisation},
  sort={trade liberalisation},
  description={Reducing tariffs, quotas, and other trade barriers. It is often sold as “efficiency,” but in unequal world markets it can accelerate deindustrialisation, worsen trade deficits, and deepen dependence on imported inputs and foreign currency.}
}

\newglossaryentry{import-substitution}{
  name={import-substitution industrialisation (ISI)},
  sort={import substitution industrialisation},
  description={A strategy to replace imports with domestic production through tariffs, credit allocation, industrial policy, and state procurement. ISI can build capacity, but it often hits constraints around technology, energy, foreign exchange, and class control of investment decisions.}
}

\newglossaryentry{qe-term}{
  name={Quantitative easing (QE)},
  sort={Quantitative easing},
  description={A central bank policy of purchasing government bonds and/or other financial assets to expand its balance sheet and push down longer-term interest rates. QE can stabilise financial markets, but it often inflates asset prices and does not automatically translate into productive investment or higher wages.}
}

\newglossaryentry{mmt-term}{
  name={Modern Monetary Theory (MMT)},
  sort={Modern Monetary Theory},
  description={A heterodox framework arguing that a state that issues its own currency cannot “run out” of money in the way households can; the binding constraints are real resources, productive capacity, and inflation dynamics. It stresses the role of taxation and bond issuance in managing demand, distribution, and monetary conditions rather than “funding” spending in a mechanical sense.}
}

\newglossaryentry{ubi-term}{
  name={Universal basic income (UBI)},
  sort={Universal basic income},
  description={An unconditional cash transfer to all residents or citizens. Proposals differ sharply: some are designed to replace welfare and subsidise low wages, while others are framed as an income floor that complements strong public services, labour rights, and decommodification.}
}

\newglossaryentry{ubs-term}{
  name={Universal basic services (UBS)},
  sort={Universal basic services},
  description={A model of guaranteeing key services—health, education, housing, transport, care, water/energy—through public provision or social rights rather than cash transfers. UBS centres decommodification and collective infrastructure, but requires fiscal capacity, democratic control, and organised labour to prevent deterioration or capture.}
}

\newglossaryentry{jg-term}{
  name={Job Guarantee (JG)},
  sort={Job Guarantee},
  description={A proposal that the state offers a public job at a socially defined wage to anyone willing to work. Advocates treat it as an employment floor and stabiliser; critics debate job quality, political control, and whether it can be insulated from austerity and patronage without strong democratic governance.}
}

\newglossaryentry{esg-term}{
  name={ESG},
  sort={ESG},
  description={A framework used by investors and firms to score “environmental, social, and governance” performance. ESG can pressure disclosure and some standards, but it is often criticised as compatible with continued extraction and financialisation, turning ecological crisis into a portfolio and reputational management problem.}
}

\newglossaryentry{troika}{
  name={the Troika},
  sort={Troika},
  description={A term commonly used for the European Commission (EC), the European Central Bank (ECB), and the IMF acting jointly in crisis programmes and conditional lending in the Eurozone.}
}

\newglossaryentry{eurozone}{
  name={Eurozone},
  sort={Eurozone},
  description={The group of EU member states using the euro. Eurozone membership removes independent monetary policy and exchange-rate adjustment, making fiscal policy and wage/price dynamics central sites of “internal devaluation” during crises.}
}

\newglossaryentry{syriza}{
  name={Syriza},
  sort={Syriza},
  description={A Greek left party (Coalition of the Radical Left) that came to power in 2015 on an anti-austerity mandate. Its confrontation with the Troika became a major reference point for debates on debt, monetary sovereignty, and the limits imposed by Eurozone institutions.}
}

% ----------------------------------------------------------------------
% 2) PRINTING MODE (when % ============================
%  glossary.tex
%  Acronyms + glossary entries (glossaries-extra, no-index workflow)
%  ALSO prints Appendix B when \input{glossary} is used in content.tex.
%
%  IMPORTANT (preamble.tex):
%    \usepackage[acronym]{glossaries-extra}
%    \makenoidxglossaries
%    \def\GLOSSARYENTRIESONLY{}%
%    \loadglsentries{glossary}
%    \let\GLOSSARYENTRIESONLY\undefined
% ============================

\ProvidesFile{glossary.tex}[Glossary and acronym entries + Appendix printing]

% ----------------------------------------------------------------------
% 1) ENTRIES-ONLY MODE (when loaded in the preamble)
% ----------------------------------------------------------------------
\ifdefined\GLOSSARYENTRIESONLY

% ---------- Acronyms ----------
\newacronym{ai}{AI}{artificial intelligence}
\newacronym{ltv}{LTV}{labour theory of value}
\newacronym{snlt}{SNLT}{socially necessary labour time}
\newacronym{melt}{MELT}{monetary expression of labour time}
\newacronym{occ}{OCC}{organic composition of capital}
\newacronym{trpf}{TRPF}{tendency of the rate of profit to fall}

\newacronym{ubi}{UBI}{universal basic income}
\newacronym{ubs}{UBS}{universal basic services}
\newacronym{qe}{QE}{quantitative easing}
\newacronym{mmt}{MMT}{Modern Monetary Theory}
\newacronym{jg}{JG}{Job Guarantee}
\newacronym{esg}{ESG}{environmental, social, and governance}

\newacronym{imf}{IMF}{International Monetary Fund}
\newacronym{ipcc}{IPCC}{Intergovernmental Panel on Climate Change}

% ---------- Glossary terms (existing) ----------
\newglossaryentry{keynesianism}{
  name={Keynesianism},
  sort={Keynesianism},
  description={A macroeconomic approach associated with John Maynard Keynes that emphasises stabilising output and employment via demand management (especially fiscal policy and public spending), particularly during downturns.},
  first={Keynesianism (a macroeconomic approach that emphasises demand management—especially fiscal policy and public spending—to stabilise output and employment)}
}

\newglossaryentry{monetarism}{
  name={Monetarism},
  sort={Monetarism},
  description={A macroeconomic doctrine associated with Milton Friedman that prioritises controlling inflation by managing the money supply and/or interest rates, often via tight monetary policy.},
  first={Monetarism (a macroeconomic doctrine that prioritises inflation control via tight monetary policy—money supply and/or interest rates)}
}

\newglossaryentry{financialisation}{
  name={financialisation},
  sort={financialisation},
  description={A pattern in which profits, strategies, and power shift toward finance, asset price inflation, and rent extraction, rather than expanded productive investment and wage growth.}
}

\newglossaryentry{fictitious-capital}{
  name={fictitious capital},
  sort={fictitious capital},
  description={Tradable claims on future income streams (shares, bonds, securitised claims, many derivatives) whose market valuation can expand beyond surplus value currently produced, until crises force devaluation.}
}

\newglossaryentry{decommodification}{
  name={decommodification},
  sort={decommodification},
  description={Shifting access to essentials (housing, health, care, transport, energy, water) out of the market and away from ability to pay, toward rights-based provision.}
}

\newglossaryentry{capital-controls}{
  name={capital controls},
  sort={capital controls},
  description={Regulatory restrictions on cross-border movement of capital designed to limit capital flight, currency pressure, and the ability of owners to discipline reforms through financial exit.}
}

% ---------- Glossary terms (extended, targeted additions) ----------
\newglossaryentry{austerity}{
  name={austerity},
  sort={austerity},
  description={A policy package of spending cuts, hiring freezes, welfare retrenchment, regressive taxation, and/or user fees justified as “fiscal discipline”. In practice it often shifts crisis costs onto workers and the poor while protecting creditors and asset owners.}
}

\newglossaryentry{fiscal-consolidation}{
  name={fiscal consolidation},
  sort={fiscal consolidation},
  description={Reducing government deficits through spending cuts and/or tax rises. It is frequently presented as technocratic “budget repair,” but its class content depends on who is taxed, which services are cut, and whether interest payments to creditors are ring-fenced.}
}

\newglossaryentry{primary-balance}{
  name={primary balance},
  sort={primary balance},
  description={A government’s fiscal balance excluding interest payments on existing debt. A “primary surplus” can coexist with rising total debt burdens if interest costs remain high or growth is weak.}
}

\newglossaryentry{debt-service}{
  name={debt servicing},
  sort={debt servicing},
  description={Ongoing payments of interest and principal on debt. For many states, debt service becomes a prior claim on public revenue, structurally pressuring social spending and investment even without an explicit austerity programme.}
}

\newglossaryentry{conditionality}{
  name={conditionality},
  sort={conditionality},
  description={Policy conditions attached to loans or debt restructuring (commonly by the IMF or creditor blocs). Typical conditions include subsidy cuts, wage restraint, privatisation, deregulation, central bank “independence,” and fiscal consolidation.}
}

\newglossaryentry{structural-adjustment}{
  name={structural adjustment},
  sort={structural adjustment},
  description={A reform programme—historically associated with IMF/World Bank lending—that restructures economies toward export orientation, market pricing, privatisation, and reduced public provision. The “adjustment” is usually borne through depressed wages, weakened labour protections, and reduced social spending.}
}

\newglossaryentry{balance-of-payments}{
  name={balance of payments},
  sort={balance of payments},
  description={A country’s accounting of transactions with the rest of the world (trade in goods/services, income flows, and financial transfers). Persistent deficits often create pressure for devaluation, import compression, and external borrowing.}
}

\newglossaryentry{capital-flight}{
  name={capital flight},
  sort={capital flight},
  description={Rapid private movement of funds out of a country (or out of domestic investment into safer assets), often triggered by crisis, political conflict, or expectations of devaluation. Capital flight can force currency pressure, reserve loss, and harsher adjustment.}
}

\newglossaryentry{exchange-rate-pass-through}{
  name={exchange-rate pass-through},
  sort={exchange rate pass-through},
  description={The extent to which a currency devaluation raises domestic prices, especially for imported essentials (fuel, fertiliser, medicine, machinery). High pass-through can turn devaluation into immediate inflation and real-wage cuts.}
}

\newglossaryentry{inflation-targeting}{
  name={inflation targeting},
  sort={inflation targeting},
  description={A monetary-policy framework where the central bank prioritises hitting an inflation target, typically via interest-rate moves. Critics argue it can treat inflation as a purely monetary phenomenon while ignoring supply shocks, monopoly pricing, and import dependence.}
}

\newglossaryentry{protectionism}{
  name={protectionism},
  sort={protectionism},
  description={Using tariffs, quotas, licensing, local-content rules, or public procurement to shelter domestic producers from foreign competition. It can defend jobs and industrial capacity, but its effects depend on who controls protected firms, how prices/wages move, and whether technology/inputs are domestically available.}
}

\newglossaryentry{trade-liberalisation}{
  name={trade liberalisation},
  sort={trade liberalisation},
  description={Reducing tariffs, quotas, and other trade barriers. It is often sold as “efficiency,” but in unequal world markets it can accelerate deindustrialisation, worsen trade deficits, and deepen dependence on imported inputs and foreign currency.}
}

\newglossaryentry{import-substitution}{
  name={import-substitution industrialisation (ISI)},
  sort={import substitution industrialisation},
  description={A strategy to replace imports with domestic production through tariffs, credit allocation, industrial policy, and state procurement. ISI can build capacity, but it often hits constraints around technology, energy, foreign exchange, and class control of investment decisions.}
}

\newglossaryentry{qe-term}{
  name={Quantitative easing (QE)},
  sort={Quantitative easing},
  description={A central bank policy of purchasing government bonds and/or other financial assets to expand its balance sheet and push down longer-term interest rates. QE can stabilise financial markets, but it often inflates asset prices and does not automatically translate into productive investment or higher wages.}
}

\newglossaryentry{mmt-term}{
  name={Modern Monetary Theory (MMT)},
  sort={Modern Monetary Theory},
  description={A heterodox framework arguing that a state that issues its own currency cannot “run out” of money in the way households can; the binding constraints are real resources, productive capacity, and inflation dynamics. It stresses the role of taxation and bond issuance in managing demand, distribution, and monetary conditions rather than “funding” spending in a mechanical sense.}
}

\newglossaryentry{ubi-term}{
  name={Universal basic income (UBI)},
  sort={Universal basic income},
  description={An unconditional cash transfer to all residents or citizens. Proposals differ sharply: some are designed to replace welfare and subsidise low wages, while others are framed as an income floor that complements strong public services, labour rights, and decommodification.}
}

\newglossaryentry{ubs-term}{
  name={Universal basic services (UBS)},
  sort={Universal basic services},
  description={A model of guaranteeing key services—health, education, housing, transport, care, water/energy—through public provision or social rights rather than cash transfers. UBS centres decommodification and collective infrastructure, but requires fiscal capacity, democratic control, and organised labour to prevent deterioration or capture.}
}

\newglossaryentry{jg-term}{
  name={Job Guarantee (JG)},
  sort={Job Guarantee},
  description={A proposal that the state offers a public job at a socially defined wage to anyone willing to work. Advocates treat it as an employment floor and stabiliser; critics debate job quality, political control, and whether it can be insulated from austerity and patronage without strong democratic governance.}
}

\newglossaryentry{esg-term}{
  name={ESG},
  sort={ESG},
  description={A framework used by investors and firms to score “environmental, social, and governance” performance. ESG can pressure disclosure and some standards, but it is often criticised as compatible with continued extraction and financialisation, turning ecological crisis into a portfolio and reputational management problem.}
}

\newglossaryentry{troika}{
  name={the Troika},
  sort={Troika},
  description={A term commonly used for the European Commission (EC), the European Central Bank (ECB), and the IMF acting jointly in crisis programmes and conditional lending in the Eurozone.}
}

\newglossaryentry{eurozone}{
  name={Eurozone},
  sort={Eurozone},
  description={The group of EU member states using the euro. Eurozone membership removes independent monetary policy and exchange-rate adjustment, making fiscal policy and wage/price dynamics central sites of “internal devaluation” during crises.}
}

\newglossaryentry{syriza}{
  name={Syriza},
  sort={Syriza},
  description={A Greek left party (Coalition of the Radical Left) that came to power in 2015 on an anti-austerity mandate. Its confrontation with the Troika became a major reference point for debates on debt, monetary sovereignty, and the limits imposed by Eurozone institutions.}
}

% ----------------------------------------------------------------------
% 2) PRINTING MODE (when \input{glossary} is called in content.tex)
% ----------------------------------------------------------------------
\else

\section{Glossary and acronyms}
\label{sec:glossary}

\begingroup
\setlength{\parskip}{0pt}

% Print even if entries were not referenced yet:
\glsaddallunused

% Acronyms
\printnoidxglossary[type=\acronymtype,style=compactgls,title={Acronyms}]

\vspace{0.6\baselineskip}

% Terms
\printnoidxglossary[style=termscolon,title={Glossary of terms}]

\endgroup

\fi is called in content.tex)
% ----------------------------------------------------------------------
\else

\section{Glossary and acronyms}
\label{sec:glossary}

\begingroup
\setlength{\parskip}{0pt}

% Print even if entries were not referenced yet:
\glsaddallunused

% Acronyms
\printnoidxglossary[type=\acronymtype,style=compactgls,title={Acronyms}]

\vspace{0.6\baselineskip}

% Terms
\printnoidxglossary[style=termscolon,title={Glossary of terms}]

\endgroup

\fi is called in content.tex)
% ----------------------------------------------------------------------
\else

\section{Glossary and acronyms}
\label{sec:glossary}

\begingroup
\setlength{\parskip}{0pt}

% Print even if entries were not referenced yet:
\glsaddallunused

% Acronyms
\printnoidxglossary[type=\acronymtype,style=compactgls,title={Acronyms}]

\vspace{0.6\baselineskip}

% Terms
\printnoidxglossary[style=termscolon,title={Glossary of terms}]

\endgroup

\fi

% ... (anything else you place after appendices, if applicable) ...
