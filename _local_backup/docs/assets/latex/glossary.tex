% ============================
%  glossary.tex
%  Acronyms + glossary entries (glossaries-extra, no-index workflow)
%  ALSO prints Appendix B when % ============================
%  glossary.tex
%  Acronyms + glossary entries (glossaries-extra, no-index workflow)
%  ALSO prints Appendix B when % ============================
%  glossary.tex
%  Acronyms + glossary entries (glossaries-extra, no-index workflow)
%  ALSO prints Appendix B when % ============================
%  glossary.tex
%  Acronyms + glossary entries (glossaries-extra, no-index workflow)
%  ALSO prints Appendix B when \input{glossary} is used in content.tex.
%
%  IMPORTANT (preamble.tex):
%    \usepackage[acronym]{glossaries-extra}
%    \makenoidxglossaries
%    \def\GLOSSARYENTRIESONLY{}%
%    \loadglsentries{glossary}
%    \let\GLOSSARYENTRIESONLY\undefined
% ============================

\ProvidesFile{glossary.tex}[Glossary and acronym entries + Appendix printing]

% ----------------------------------------------------------------------
% 1) ENTRIES-ONLY MODE (when loaded in the preamble)
% ----------------------------------------------------------------------
\ifdefined\GLOSSARYENTRIESONLY

% ---------- Acronyms ----------
\newacronym{ai}{AI}{artificial intelligence}
\newacronym{ltv}{LTV}{labour theory of value}
\newacronym{snlt}{SNLT}{socially necessary labour time}
\newacronym{melt}{MELT}{monetary expression of labour time}
\newacronym{occ}{OCC}{organic composition of capital}
\newacronym{trpf}{TRPF}{tendency of the rate of profit to fall}

\newacronym{ubi}{UBI}{universal basic income}
\newacronym{ubs}{UBS}{universal basic services}
\newacronym{qe}{QE}{quantitative easing}
\newacronym{mmt}{MMT}{Modern Monetary Theory}
\newacronym{jg}{JG}{Job Guarantee}
\newacronym{esg}{ESG}{environmental, social, and governance}

\newacronym{imf}{IMF}{International Monetary Fund}
\newacronym{ipcc}{IPCC}{Intergovernmental Panel on Climate Change}

% ---------- Glossary terms (existing) ----------
\newglossaryentry{keynesianism}{
  name={Keynesianism},
  sort={Keynesianism},
  description={A macroeconomic approach associated with John Maynard Keynes that emphasises stabilising output and employment via demand management (especially fiscal policy and public spending), particularly during downturns.},
  first={Keynesianism (a macroeconomic approach that emphasises demand management—especially fiscal policy and public spending—to stabilise output and employment)}
}

\newglossaryentry{monetarism}{
  name={Monetarism},
  sort={Monetarism},
  description={A macroeconomic doctrine associated with Milton Friedman that prioritises controlling inflation by managing the money supply and/or interest rates, often via tight monetary policy.},
  first={Monetarism (a macroeconomic doctrine that prioritises inflation control via tight monetary policy—money supply and/or interest rates)}
}

\newglossaryentry{financialisation}{
  name={financialisation},
  sort={financialisation},
  description={A pattern in which profits, strategies, and power shift toward finance, asset price inflation, and rent extraction, rather than expanded productive investment and wage growth.}
}

\newglossaryentry{fictitious-capital}{
  name={fictitious capital},
  sort={fictitious capital},
  description={Tradable claims on future income streams (shares, bonds, securitised claims, many derivatives) whose market valuation can expand beyond surplus value currently produced, until crises force devaluation.}
}

\newglossaryentry{decommodification}{
  name={decommodification},
  sort={decommodification},
  description={Shifting access to essentials (housing, health, care, transport, energy, water) out of the market and away from ability to pay, toward rights-based provision.}
}

\newglossaryentry{capital-controls}{
  name={capital controls},
  sort={capital controls},
  description={Regulatory restrictions on cross-border movement of capital designed to limit capital flight, currency pressure, and the ability of owners to discipline reforms through financial exit.}
}

% ---------- Glossary terms (extended, targeted additions) ----------
\newglossaryentry{austerity}{
  name={austerity},
  sort={austerity},
  description={A policy package of spending cuts, hiring freezes, welfare retrenchment, regressive taxation, and/or user fees justified as “fiscal discipline”. In practice it often shifts crisis costs onto workers and the poor while protecting creditors and asset owners.}
}

\newglossaryentry{fiscal-consolidation}{
  name={fiscal consolidation},
  sort={fiscal consolidation},
  description={Reducing government deficits through spending cuts and/or tax rises. It is frequently presented as technocratic “budget repair,” but its class content depends on who is taxed, which services are cut, and whether interest payments to creditors are ring-fenced.}
}

\newglossaryentry{primary-balance}{
  name={primary balance},
  sort={primary balance},
  description={A government’s fiscal balance excluding interest payments on existing debt. A “primary surplus” can coexist with rising total debt burdens if interest costs remain high or growth is weak.}
}

\newglossaryentry{debt-service}{
  name={debt servicing},
  sort={debt servicing},
  description={Ongoing payments of interest and principal on debt. For many states, debt service becomes a prior claim on public revenue, structurally pressuring social spending and investment even without an explicit austerity programme.}
}

\newglossaryentry{conditionality}{
  name={conditionality},
  sort={conditionality},
  description={Policy conditions attached to loans or debt restructuring (commonly by the IMF or creditor blocs). Typical conditions include subsidy cuts, wage restraint, privatisation, deregulation, central bank “independence,” and fiscal consolidation.}
}

\newglossaryentry{structural-adjustment}{
  name={structural adjustment},
  sort={structural adjustment},
  description={A reform programme—historically associated with IMF/World Bank lending—that restructures economies toward export orientation, market pricing, privatisation, and reduced public provision. The “adjustment” is usually borne through depressed wages, weakened labour protections, and reduced social spending.}
}

\newglossaryentry{balance-of-payments}{
  name={balance of payments},
  sort={balance of payments},
  description={A country’s accounting of transactions with the rest of the world (trade in goods/services, income flows, and financial transfers). Persistent deficits often create pressure for devaluation, import compression, and external borrowing.}
}

\newglossaryentry{capital-flight}{
  name={capital flight},
  sort={capital flight},
  description={Rapid private movement of funds out of a country (or out of domestic investment into safer assets), often triggered by crisis, political conflict, or expectations of devaluation. Capital flight can force currency pressure, reserve loss, and harsher adjustment.}
}

\newglossaryentry{exchange-rate-pass-through}{
  name={exchange-rate pass-through},
  sort={exchange rate pass-through},
  description={The extent to which a currency devaluation raises domestic prices, especially for imported essentials (fuel, fertiliser, medicine, machinery). High pass-through can turn devaluation into immediate inflation and real-wage cuts.}
}

\newglossaryentry{inflation-targeting}{
  name={inflation targeting},
  sort={inflation targeting},
  description={A monetary-policy framework where the central bank prioritises hitting an inflation target, typically via interest-rate moves. Critics argue it can treat inflation as a purely monetary phenomenon while ignoring supply shocks, monopoly pricing, and import dependence.}
}

\newglossaryentry{protectionism}{
  name={protectionism},
  sort={protectionism},
  description={Using tariffs, quotas, licensing, local-content rules, or public procurement to shelter domestic producers from foreign competition. It can defend jobs and industrial capacity, but its effects depend on who controls protected firms, how prices/wages move, and whether technology/inputs are domestically available.}
}

\newglossaryentry{trade-liberalisation}{
  name={trade liberalisation},
  sort={trade liberalisation},
  description={Reducing tariffs, quotas, and other trade barriers. It is often sold as “efficiency,” but in unequal world markets it can accelerate deindustrialisation, worsen trade deficits, and deepen dependence on imported inputs and foreign currency.}
}

\newglossaryentry{import-substitution}{
  name={import-substitution industrialisation (ISI)},
  sort={import substitution industrialisation},
  description={A strategy to replace imports with domestic production through tariffs, credit allocation, industrial policy, and state procurement. ISI can build capacity, but it often hits constraints around technology, energy, foreign exchange, and class control of investment decisions.}
}

\newglossaryentry{qe-term}{
  name={Quantitative easing (QE)},
  sort={Quantitative easing},
  description={A central bank policy of purchasing government bonds and/or other financial assets to expand its balance sheet and push down longer-term interest rates. QE can stabilise financial markets, but it often inflates asset prices and does not automatically translate into productive investment or higher wages.}
}

\newglossaryentry{mmt-term}{
  name={Modern Monetary Theory (MMT)},
  sort={Modern Monetary Theory},
  description={A heterodox framework arguing that a state that issues its own currency cannot “run out” of money in the way households can; the binding constraints are real resources, productive capacity, and inflation dynamics. It stresses the role of taxation and bond issuance in managing demand, distribution, and monetary conditions rather than “funding” spending in a mechanical sense.}
}

\newglossaryentry{ubi-term}{
  name={Universal basic income (UBI)},
  sort={Universal basic income},
  description={An unconditional cash transfer to all residents or citizens. Proposals differ sharply: some are designed to replace welfare and subsidise low wages, while others are framed as an income floor that complements strong public services, labour rights, and decommodification.}
}

\newglossaryentry{ubs-term}{
  name={Universal basic services (UBS)},
  sort={Universal basic services},
  description={A model of guaranteeing key services—health, education, housing, transport, care, water/energy—through public provision or social rights rather than cash transfers. UBS centres decommodification and collective infrastructure, but requires fiscal capacity, democratic control, and organised labour to prevent deterioration or capture.}
}

\newglossaryentry{jg-term}{
  name={Job Guarantee (JG)},
  sort={Job Guarantee},
  description={A proposal that the state offers a public job at a socially defined wage to anyone willing to work. Advocates treat it as an employment floor and stabiliser; critics debate job quality, political control, and whether it can be insulated from austerity and patronage without strong democratic governance.}
}

\newglossaryentry{esg-term}{
  name={ESG},
  sort={ESG},
  description={A framework used by investors and firms to score “environmental, social, and governance” performance. ESG can pressure disclosure and some standards, but it is often criticised as compatible with continued extraction and financialisation, turning ecological crisis into a portfolio and reputational management problem.}
}

\newglossaryentry{troika}{
  name={the Troika},
  sort={Troika},
  description={A term commonly used for the European Commission (EC), the European Central Bank (ECB), and the IMF acting jointly in crisis programmes and conditional lending in the Eurozone.}
}

\newglossaryentry{eurozone}{
  name={Eurozone},
  sort={Eurozone},
  description={The group of EU member states using the euro. Eurozone membership removes independent monetary policy and exchange-rate adjustment, making fiscal policy and wage/price dynamics central sites of “internal devaluation” during crises.}
}

\newglossaryentry{syriza}{
  name={Syriza},
  sort={Syriza},
  description={A Greek left party (Coalition of the Radical Left) that came to power in 2015 on an anti-austerity mandate. Its confrontation with the Troika became a major reference point for debates on debt, monetary sovereignty, and the limits imposed by Eurozone institutions.}
}

% ----------------------------------------------------------------------
% 2) PRINTING MODE (when \input{glossary} is called in content.tex)
% ----------------------------------------------------------------------
\else

\section{Glossary and acronyms}
\label{sec:glossary}

\begingroup
\setlength{\parskip}{0pt}

% Print even if entries were not referenced yet:
\glsaddallunused

% Acronyms
\printnoidxglossary[type=\acronymtype,style=compactgls,title={Acronyms}]

\vspace{0.6\baselineskip}

% Terms
\printnoidxglossary[style=termscolon,title={Glossary of terms}]

\endgroup

\fi is used in content.tex.
%
%  IMPORTANT (preamble.tex):
%    \usepackage[acronym]{glossaries-extra}
%    \makenoidxglossaries
%    \def\GLOSSARYENTRIESONLY{}%
%    \loadglsentries{glossary}
%    \let\GLOSSARYENTRIESONLY\undefined
% ============================

\ProvidesFile{glossary.tex}[Glossary and acronym entries + Appendix printing]

% ----------------------------------------------------------------------
% 1) ENTRIES-ONLY MODE (when loaded in the preamble)
% ----------------------------------------------------------------------
\ifdefined\GLOSSARYENTRIESONLY

% ---------- Acronyms ----------
\newacronym{ai}{AI}{artificial intelligence}
\newacronym{ltv}{LTV}{labour theory of value}
\newacronym{snlt}{SNLT}{socially necessary labour time}
\newacronym{melt}{MELT}{monetary expression of labour time}
\newacronym{occ}{OCC}{organic composition of capital}
\newacronym{trpf}{TRPF}{tendency of the rate of profit to fall}

\newacronym{ubi}{UBI}{universal basic income}
\newacronym{ubs}{UBS}{universal basic services}
\newacronym{qe}{QE}{quantitative easing}
\newacronym{mmt}{MMT}{Modern Monetary Theory}
\newacronym{jg}{JG}{Job Guarantee}
\newacronym{esg}{ESG}{environmental, social, and governance}

\newacronym{imf}{IMF}{International Monetary Fund}
\newacronym{ipcc}{IPCC}{Intergovernmental Panel on Climate Change}

% ---------- Glossary terms (existing) ----------
\newglossaryentry{keynesianism}{
  name={Keynesianism},
  sort={Keynesianism},
  description={A macroeconomic approach associated with John Maynard Keynes that emphasises stabilising output and employment via demand management (especially fiscal policy and public spending), particularly during downturns.},
  first={Keynesianism (a macroeconomic approach that emphasises demand management—especially fiscal policy and public spending—to stabilise output and employment)}
}

\newglossaryentry{monetarism}{
  name={Monetarism},
  sort={Monetarism},
  description={A macroeconomic doctrine associated with Milton Friedman that prioritises controlling inflation by managing the money supply and/or interest rates, often via tight monetary policy.},
  first={Monetarism (a macroeconomic doctrine that prioritises inflation control via tight monetary policy—money supply and/or interest rates)}
}

\newglossaryentry{financialisation}{
  name={financialisation},
  sort={financialisation},
  description={A pattern in which profits, strategies, and power shift toward finance, asset price inflation, and rent extraction, rather than expanded productive investment and wage growth.}
}

\newglossaryentry{fictitious-capital}{
  name={fictitious capital},
  sort={fictitious capital},
  description={Tradable claims on future income streams (shares, bonds, securitised claims, many derivatives) whose market valuation can expand beyond surplus value currently produced, until crises force devaluation.}
}

\newglossaryentry{decommodification}{
  name={decommodification},
  sort={decommodification},
  description={Shifting access to essentials (housing, health, care, transport, energy, water) out of the market and away from ability to pay, toward rights-based provision.}
}

\newglossaryentry{capital-controls}{
  name={capital controls},
  sort={capital controls},
  description={Regulatory restrictions on cross-border movement of capital designed to limit capital flight, currency pressure, and the ability of owners to discipline reforms through financial exit.}
}

% ---------- Glossary terms (extended, targeted additions) ----------
\newglossaryentry{austerity}{
  name={austerity},
  sort={austerity},
  description={A policy package of spending cuts, hiring freezes, welfare retrenchment, regressive taxation, and/or user fees justified as “fiscal discipline”. In practice it often shifts crisis costs onto workers and the poor while protecting creditors and asset owners.}
}

\newglossaryentry{fiscal-consolidation}{
  name={fiscal consolidation},
  sort={fiscal consolidation},
  description={Reducing government deficits through spending cuts and/or tax rises. It is frequently presented as technocratic “budget repair,” but its class content depends on who is taxed, which services are cut, and whether interest payments to creditors are ring-fenced.}
}

\newglossaryentry{primary-balance}{
  name={primary balance},
  sort={primary balance},
  description={A government’s fiscal balance excluding interest payments on existing debt. A “primary surplus” can coexist with rising total debt burdens if interest costs remain high or growth is weak.}
}

\newglossaryentry{debt-service}{
  name={debt servicing},
  sort={debt servicing},
  description={Ongoing payments of interest and principal on debt. For many states, debt service becomes a prior claim on public revenue, structurally pressuring social spending and investment even without an explicit austerity programme.}
}

\newglossaryentry{conditionality}{
  name={conditionality},
  sort={conditionality},
  description={Policy conditions attached to loans or debt restructuring (commonly by the IMF or creditor blocs). Typical conditions include subsidy cuts, wage restraint, privatisation, deregulation, central bank “independence,” and fiscal consolidation.}
}

\newglossaryentry{structural-adjustment}{
  name={structural adjustment},
  sort={structural adjustment},
  description={A reform programme—historically associated with IMF/World Bank lending—that restructures economies toward export orientation, market pricing, privatisation, and reduced public provision. The “adjustment” is usually borne through depressed wages, weakened labour protections, and reduced social spending.}
}

\newglossaryentry{balance-of-payments}{
  name={balance of payments},
  sort={balance of payments},
  description={A country’s accounting of transactions with the rest of the world (trade in goods/services, income flows, and financial transfers). Persistent deficits often create pressure for devaluation, import compression, and external borrowing.}
}

\newglossaryentry{capital-flight}{
  name={capital flight},
  sort={capital flight},
  description={Rapid private movement of funds out of a country (or out of domestic investment into safer assets), often triggered by crisis, political conflict, or expectations of devaluation. Capital flight can force currency pressure, reserve loss, and harsher adjustment.}
}

\newglossaryentry{exchange-rate-pass-through}{
  name={exchange-rate pass-through},
  sort={exchange rate pass-through},
  description={The extent to which a currency devaluation raises domestic prices, especially for imported essentials (fuel, fertiliser, medicine, machinery). High pass-through can turn devaluation into immediate inflation and real-wage cuts.}
}

\newglossaryentry{inflation-targeting}{
  name={inflation targeting},
  sort={inflation targeting},
  description={A monetary-policy framework where the central bank prioritises hitting an inflation target, typically via interest-rate moves. Critics argue it can treat inflation as a purely monetary phenomenon while ignoring supply shocks, monopoly pricing, and import dependence.}
}

\newglossaryentry{protectionism}{
  name={protectionism},
  sort={protectionism},
  description={Using tariffs, quotas, licensing, local-content rules, or public procurement to shelter domestic producers from foreign competition. It can defend jobs and industrial capacity, but its effects depend on who controls protected firms, how prices/wages move, and whether technology/inputs are domestically available.}
}

\newglossaryentry{trade-liberalisation}{
  name={trade liberalisation},
  sort={trade liberalisation},
  description={Reducing tariffs, quotas, and other trade barriers. It is often sold as “efficiency,” but in unequal world markets it can accelerate deindustrialisation, worsen trade deficits, and deepen dependence on imported inputs and foreign currency.}
}

\newglossaryentry{import-substitution}{
  name={import-substitution industrialisation (ISI)},
  sort={import substitution industrialisation},
  description={A strategy to replace imports with domestic production through tariffs, credit allocation, industrial policy, and state procurement. ISI can build capacity, but it often hits constraints around technology, energy, foreign exchange, and class control of investment decisions.}
}

\newglossaryentry{qe-term}{
  name={Quantitative easing (QE)},
  sort={Quantitative easing},
  description={A central bank policy of purchasing government bonds and/or other financial assets to expand its balance sheet and push down longer-term interest rates. QE can stabilise financial markets, but it often inflates asset prices and does not automatically translate into productive investment or higher wages.}
}

\newglossaryentry{mmt-term}{
  name={Modern Monetary Theory (MMT)},
  sort={Modern Monetary Theory},
  description={A heterodox framework arguing that a state that issues its own currency cannot “run out” of money in the way households can; the binding constraints are real resources, productive capacity, and inflation dynamics. It stresses the role of taxation and bond issuance in managing demand, distribution, and monetary conditions rather than “funding” spending in a mechanical sense.}
}

\newglossaryentry{ubi-term}{
  name={Universal basic income (UBI)},
  sort={Universal basic income},
  description={An unconditional cash transfer to all residents or citizens. Proposals differ sharply: some are designed to replace welfare and subsidise low wages, while others are framed as an income floor that complements strong public services, labour rights, and decommodification.}
}

\newglossaryentry{ubs-term}{
  name={Universal basic services (UBS)},
  sort={Universal basic services},
  description={A model of guaranteeing key services—health, education, housing, transport, care, water/energy—through public provision or social rights rather than cash transfers. UBS centres decommodification and collective infrastructure, but requires fiscal capacity, democratic control, and organised labour to prevent deterioration or capture.}
}

\newglossaryentry{jg-term}{
  name={Job Guarantee (JG)},
  sort={Job Guarantee},
  description={A proposal that the state offers a public job at a socially defined wage to anyone willing to work. Advocates treat it as an employment floor and stabiliser; critics debate job quality, political control, and whether it can be insulated from austerity and patronage without strong democratic governance.}
}

\newglossaryentry{esg-term}{
  name={ESG},
  sort={ESG},
  description={A framework used by investors and firms to score “environmental, social, and governance” performance. ESG can pressure disclosure and some standards, but it is often criticised as compatible with continued extraction and financialisation, turning ecological crisis into a portfolio and reputational management problem.}
}

\newglossaryentry{troika}{
  name={the Troika},
  sort={Troika},
  description={A term commonly used for the European Commission (EC), the European Central Bank (ECB), and the IMF acting jointly in crisis programmes and conditional lending in the Eurozone.}
}

\newglossaryentry{eurozone}{
  name={Eurozone},
  sort={Eurozone},
  description={The group of EU member states using the euro. Eurozone membership removes independent monetary policy and exchange-rate adjustment, making fiscal policy and wage/price dynamics central sites of “internal devaluation” during crises.}
}

\newglossaryentry{syriza}{
  name={Syriza},
  sort={Syriza},
  description={A Greek left party (Coalition of the Radical Left) that came to power in 2015 on an anti-austerity mandate. Its confrontation with the Troika became a major reference point for debates on debt, monetary sovereignty, and the limits imposed by Eurozone institutions.}
}

% ----------------------------------------------------------------------
% 2) PRINTING MODE (when % ============================
%  glossary.tex
%  Acronyms + glossary entries (glossaries-extra, no-index workflow)
%  ALSO prints Appendix B when \input{glossary} is used in content.tex.
%
%  IMPORTANT (preamble.tex):
%    \usepackage[acronym]{glossaries-extra}
%    \makenoidxglossaries
%    \def\GLOSSARYENTRIESONLY{}%
%    \loadglsentries{glossary}
%    \let\GLOSSARYENTRIESONLY\undefined
% ============================

\ProvidesFile{glossary.tex}[Glossary and acronym entries + Appendix printing]

% ----------------------------------------------------------------------
% 1) ENTRIES-ONLY MODE (when loaded in the preamble)
% ----------------------------------------------------------------------
\ifdefined\GLOSSARYENTRIESONLY

% ---------- Acronyms ----------
\newacronym{ai}{AI}{artificial intelligence}
\newacronym{ltv}{LTV}{labour theory of value}
\newacronym{snlt}{SNLT}{socially necessary labour time}
\newacronym{melt}{MELT}{monetary expression of labour time}
\newacronym{occ}{OCC}{organic composition of capital}
\newacronym{trpf}{TRPF}{tendency of the rate of profit to fall}

\newacronym{ubi}{UBI}{universal basic income}
\newacronym{ubs}{UBS}{universal basic services}
\newacronym{qe}{QE}{quantitative easing}
\newacronym{mmt}{MMT}{Modern Monetary Theory}
\newacronym{jg}{JG}{Job Guarantee}
\newacronym{esg}{ESG}{environmental, social, and governance}

\newacronym{imf}{IMF}{International Monetary Fund}
\newacronym{ipcc}{IPCC}{Intergovernmental Panel on Climate Change}

% ---------- Glossary terms (existing) ----------
\newglossaryentry{keynesianism}{
  name={Keynesianism},
  sort={Keynesianism},
  description={A macroeconomic approach associated with John Maynard Keynes that emphasises stabilising output and employment via demand management (especially fiscal policy and public spending), particularly during downturns.},
  first={Keynesianism (a macroeconomic approach that emphasises demand management—especially fiscal policy and public spending—to stabilise output and employment)}
}

\newglossaryentry{monetarism}{
  name={Monetarism},
  sort={Monetarism},
  description={A macroeconomic doctrine associated with Milton Friedman that prioritises controlling inflation by managing the money supply and/or interest rates, often via tight monetary policy.},
  first={Monetarism (a macroeconomic doctrine that prioritises inflation control via tight monetary policy—money supply and/or interest rates)}
}

\newglossaryentry{financialisation}{
  name={financialisation},
  sort={financialisation},
  description={A pattern in which profits, strategies, and power shift toward finance, asset price inflation, and rent extraction, rather than expanded productive investment and wage growth.}
}

\newglossaryentry{fictitious-capital}{
  name={fictitious capital},
  sort={fictitious capital},
  description={Tradable claims on future income streams (shares, bonds, securitised claims, many derivatives) whose market valuation can expand beyond surplus value currently produced, until crises force devaluation.}
}

\newglossaryentry{decommodification}{
  name={decommodification},
  sort={decommodification},
  description={Shifting access to essentials (housing, health, care, transport, energy, water) out of the market and away from ability to pay, toward rights-based provision.}
}

\newglossaryentry{capital-controls}{
  name={capital controls},
  sort={capital controls},
  description={Regulatory restrictions on cross-border movement of capital designed to limit capital flight, currency pressure, and the ability of owners to discipline reforms through financial exit.}
}

% ---------- Glossary terms (extended, targeted additions) ----------
\newglossaryentry{austerity}{
  name={austerity},
  sort={austerity},
  description={A policy package of spending cuts, hiring freezes, welfare retrenchment, regressive taxation, and/or user fees justified as “fiscal discipline”. In practice it often shifts crisis costs onto workers and the poor while protecting creditors and asset owners.}
}

\newglossaryentry{fiscal-consolidation}{
  name={fiscal consolidation},
  sort={fiscal consolidation},
  description={Reducing government deficits through spending cuts and/or tax rises. It is frequently presented as technocratic “budget repair,” but its class content depends on who is taxed, which services are cut, and whether interest payments to creditors are ring-fenced.}
}

\newglossaryentry{primary-balance}{
  name={primary balance},
  sort={primary balance},
  description={A government’s fiscal balance excluding interest payments on existing debt. A “primary surplus” can coexist with rising total debt burdens if interest costs remain high or growth is weak.}
}

\newglossaryentry{debt-service}{
  name={debt servicing},
  sort={debt servicing},
  description={Ongoing payments of interest and principal on debt. For many states, debt service becomes a prior claim on public revenue, structurally pressuring social spending and investment even without an explicit austerity programme.}
}

\newglossaryentry{conditionality}{
  name={conditionality},
  sort={conditionality},
  description={Policy conditions attached to loans or debt restructuring (commonly by the IMF or creditor blocs). Typical conditions include subsidy cuts, wage restraint, privatisation, deregulation, central bank “independence,” and fiscal consolidation.}
}

\newglossaryentry{structural-adjustment}{
  name={structural adjustment},
  sort={structural adjustment},
  description={A reform programme—historically associated with IMF/World Bank lending—that restructures economies toward export orientation, market pricing, privatisation, and reduced public provision. The “adjustment” is usually borne through depressed wages, weakened labour protections, and reduced social spending.}
}

\newglossaryentry{balance-of-payments}{
  name={balance of payments},
  sort={balance of payments},
  description={A country’s accounting of transactions with the rest of the world (trade in goods/services, income flows, and financial transfers). Persistent deficits often create pressure for devaluation, import compression, and external borrowing.}
}

\newglossaryentry{capital-flight}{
  name={capital flight},
  sort={capital flight},
  description={Rapid private movement of funds out of a country (or out of domestic investment into safer assets), often triggered by crisis, political conflict, or expectations of devaluation. Capital flight can force currency pressure, reserve loss, and harsher adjustment.}
}

\newglossaryentry{exchange-rate-pass-through}{
  name={exchange-rate pass-through},
  sort={exchange rate pass-through},
  description={The extent to which a currency devaluation raises domestic prices, especially for imported essentials (fuel, fertiliser, medicine, machinery). High pass-through can turn devaluation into immediate inflation and real-wage cuts.}
}

\newglossaryentry{inflation-targeting}{
  name={inflation targeting},
  sort={inflation targeting},
  description={A monetary-policy framework where the central bank prioritises hitting an inflation target, typically via interest-rate moves. Critics argue it can treat inflation as a purely monetary phenomenon while ignoring supply shocks, monopoly pricing, and import dependence.}
}

\newglossaryentry{protectionism}{
  name={protectionism},
  sort={protectionism},
  description={Using tariffs, quotas, licensing, local-content rules, or public procurement to shelter domestic producers from foreign competition. It can defend jobs and industrial capacity, but its effects depend on who controls protected firms, how prices/wages move, and whether technology/inputs are domestically available.}
}

\newglossaryentry{trade-liberalisation}{
  name={trade liberalisation},
  sort={trade liberalisation},
  description={Reducing tariffs, quotas, and other trade barriers. It is often sold as “efficiency,” but in unequal world markets it can accelerate deindustrialisation, worsen trade deficits, and deepen dependence on imported inputs and foreign currency.}
}

\newglossaryentry{import-substitution}{
  name={import-substitution industrialisation (ISI)},
  sort={import substitution industrialisation},
  description={A strategy to replace imports with domestic production through tariffs, credit allocation, industrial policy, and state procurement. ISI can build capacity, but it often hits constraints around technology, energy, foreign exchange, and class control of investment decisions.}
}

\newglossaryentry{qe-term}{
  name={Quantitative easing (QE)},
  sort={Quantitative easing},
  description={A central bank policy of purchasing government bonds and/or other financial assets to expand its balance sheet and push down longer-term interest rates. QE can stabilise financial markets, but it often inflates asset prices and does not automatically translate into productive investment or higher wages.}
}

\newglossaryentry{mmt-term}{
  name={Modern Monetary Theory (MMT)},
  sort={Modern Monetary Theory},
  description={A heterodox framework arguing that a state that issues its own currency cannot “run out” of money in the way households can; the binding constraints are real resources, productive capacity, and inflation dynamics. It stresses the role of taxation and bond issuance in managing demand, distribution, and monetary conditions rather than “funding” spending in a mechanical sense.}
}

\newglossaryentry{ubi-term}{
  name={Universal basic income (UBI)},
  sort={Universal basic income},
  description={An unconditional cash transfer to all residents or citizens. Proposals differ sharply: some are designed to replace welfare and subsidise low wages, while others are framed as an income floor that complements strong public services, labour rights, and decommodification.}
}

\newglossaryentry{ubs-term}{
  name={Universal basic services (UBS)},
  sort={Universal basic services},
  description={A model of guaranteeing key services—health, education, housing, transport, care, water/energy—through public provision or social rights rather than cash transfers. UBS centres decommodification and collective infrastructure, but requires fiscal capacity, democratic control, and organised labour to prevent deterioration or capture.}
}

\newglossaryentry{jg-term}{
  name={Job Guarantee (JG)},
  sort={Job Guarantee},
  description={A proposal that the state offers a public job at a socially defined wage to anyone willing to work. Advocates treat it as an employment floor and stabiliser; critics debate job quality, political control, and whether it can be insulated from austerity and patronage without strong democratic governance.}
}

\newglossaryentry{esg-term}{
  name={ESG},
  sort={ESG},
  description={A framework used by investors and firms to score “environmental, social, and governance” performance. ESG can pressure disclosure and some standards, but it is often criticised as compatible with continued extraction and financialisation, turning ecological crisis into a portfolio and reputational management problem.}
}

\newglossaryentry{troika}{
  name={the Troika},
  sort={Troika},
  description={A term commonly used for the European Commission (EC), the European Central Bank (ECB), and the IMF acting jointly in crisis programmes and conditional lending in the Eurozone.}
}

\newglossaryentry{eurozone}{
  name={Eurozone},
  sort={Eurozone},
  description={The group of EU member states using the euro. Eurozone membership removes independent monetary policy and exchange-rate adjustment, making fiscal policy and wage/price dynamics central sites of “internal devaluation” during crises.}
}

\newglossaryentry{syriza}{
  name={Syriza},
  sort={Syriza},
  description={A Greek left party (Coalition of the Radical Left) that came to power in 2015 on an anti-austerity mandate. Its confrontation with the Troika became a major reference point for debates on debt, monetary sovereignty, and the limits imposed by Eurozone institutions.}
}

% ----------------------------------------------------------------------
% 2) PRINTING MODE (when \input{glossary} is called in content.tex)
% ----------------------------------------------------------------------
\else

\section{Glossary and acronyms}
\label{sec:glossary}

\begingroup
\setlength{\parskip}{0pt}

% Print even if entries were not referenced yet:
\glsaddallunused

% Acronyms
\printnoidxglossary[type=\acronymtype,style=compactgls,title={Acronyms}]

\vspace{0.6\baselineskip}

% Terms
\printnoidxglossary[style=termscolon,title={Glossary of terms}]

\endgroup

\fi is called in content.tex)
% ----------------------------------------------------------------------
\else

\section{Glossary and acronyms}
\label{sec:glossary}

\begingroup
\setlength{\parskip}{0pt}

% Print even if entries were not referenced yet:
\glsaddallunused

% Acronyms
\printnoidxglossary[type=\acronymtype,style=compactgls,title={Acronyms}]

\vspace{0.6\baselineskip}

% Terms
\printnoidxglossary[style=termscolon,title={Glossary of terms}]

\endgroup

\fi is used in content.tex.
%
%  IMPORTANT (preamble.tex):
%    \usepackage[acronym]{glossaries-extra}
%    \makenoidxglossaries
%    \def\GLOSSARYENTRIESONLY{}%
%    \loadglsentries{glossary}
%    \let\GLOSSARYENTRIESONLY\undefined
% ============================

\ProvidesFile{glossary.tex}[Glossary and acronym entries + Appendix printing]

% ----------------------------------------------------------------------
% 1) ENTRIES-ONLY MODE (when loaded in the preamble)
% ----------------------------------------------------------------------
\ifdefined\GLOSSARYENTRIESONLY

% ---------- Acronyms ----------
\newacronym{ai}{AI}{artificial intelligence}
\newacronym{ltv}{LTV}{labour theory of value}
\newacronym{snlt}{SNLT}{socially necessary labour time}
\newacronym{melt}{MELT}{monetary expression of labour time}
\newacronym{occ}{OCC}{organic composition of capital}
\newacronym{trpf}{TRPF}{tendency of the rate of profit to fall}

\newacronym{ubi}{UBI}{universal basic income}
\newacronym{ubs}{UBS}{universal basic services}
\newacronym{qe}{QE}{quantitative easing}
\newacronym{mmt}{MMT}{Modern Monetary Theory}
\newacronym{jg}{JG}{Job Guarantee}
\newacronym{esg}{ESG}{environmental, social, and governance}

\newacronym{imf}{IMF}{International Monetary Fund}
\newacronym{ipcc}{IPCC}{Intergovernmental Panel on Climate Change}

% ---------- Glossary terms (existing) ----------
\newglossaryentry{keynesianism}{
  name={Keynesianism},
  sort={Keynesianism},
  description={A macroeconomic approach associated with John Maynard Keynes that emphasises stabilising output and employment via demand management (especially fiscal policy and public spending), particularly during downturns.},
  first={Keynesianism (a macroeconomic approach that emphasises demand management—especially fiscal policy and public spending—to stabilise output and employment)}
}

\newglossaryentry{monetarism}{
  name={Monetarism},
  sort={Monetarism},
  description={A macroeconomic doctrine associated with Milton Friedman that prioritises controlling inflation by managing the money supply and/or interest rates, often via tight monetary policy.},
  first={Monetarism (a macroeconomic doctrine that prioritises inflation control via tight monetary policy—money supply and/or interest rates)}
}

\newglossaryentry{financialisation}{
  name={financialisation},
  sort={financialisation},
  description={A pattern in which profits, strategies, and power shift toward finance, asset price inflation, and rent extraction, rather than expanded productive investment and wage growth.}
}

\newglossaryentry{fictitious-capital}{
  name={fictitious capital},
  sort={fictitious capital},
  description={Tradable claims on future income streams (shares, bonds, securitised claims, many derivatives) whose market valuation can expand beyond surplus value currently produced, until crises force devaluation.}
}

\newglossaryentry{decommodification}{
  name={decommodification},
  sort={decommodification},
  description={Shifting access to essentials (housing, health, care, transport, energy, water) out of the market and away from ability to pay, toward rights-based provision.}
}

\newglossaryentry{capital-controls}{
  name={capital controls},
  sort={capital controls},
  description={Regulatory restrictions on cross-border movement of capital designed to limit capital flight, currency pressure, and the ability of owners to discipline reforms through financial exit.}
}

% ---------- Glossary terms (extended, targeted additions) ----------
\newglossaryentry{austerity}{
  name={austerity},
  sort={austerity},
  description={A policy package of spending cuts, hiring freezes, welfare retrenchment, regressive taxation, and/or user fees justified as “fiscal discipline”. In practice it often shifts crisis costs onto workers and the poor while protecting creditors and asset owners.}
}

\newglossaryentry{fiscal-consolidation}{
  name={fiscal consolidation},
  sort={fiscal consolidation},
  description={Reducing government deficits through spending cuts and/or tax rises. It is frequently presented as technocratic “budget repair,” but its class content depends on who is taxed, which services are cut, and whether interest payments to creditors are ring-fenced.}
}

\newglossaryentry{primary-balance}{
  name={primary balance},
  sort={primary balance},
  description={A government’s fiscal balance excluding interest payments on existing debt. A “primary surplus” can coexist with rising total debt burdens if interest costs remain high or growth is weak.}
}

\newglossaryentry{debt-service}{
  name={debt servicing},
  sort={debt servicing},
  description={Ongoing payments of interest and principal on debt. For many states, debt service becomes a prior claim on public revenue, structurally pressuring social spending and investment even without an explicit austerity programme.}
}

\newglossaryentry{conditionality}{
  name={conditionality},
  sort={conditionality},
  description={Policy conditions attached to loans or debt restructuring (commonly by the IMF or creditor blocs). Typical conditions include subsidy cuts, wage restraint, privatisation, deregulation, central bank “independence,” and fiscal consolidation.}
}

\newglossaryentry{structural-adjustment}{
  name={structural adjustment},
  sort={structural adjustment},
  description={A reform programme—historically associated with IMF/World Bank lending—that restructures economies toward export orientation, market pricing, privatisation, and reduced public provision. The “adjustment” is usually borne through depressed wages, weakened labour protections, and reduced social spending.}
}

\newglossaryentry{balance-of-payments}{
  name={balance of payments},
  sort={balance of payments},
  description={A country’s accounting of transactions with the rest of the world (trade in goods/services, income flows, and financial transfers). Persistent deficits often create pressure for devaluation, import compression, and external borrowing.}
}

\newglossaryentry{capital-flight}{
  name={capital flight},
  sort={capital flight},
  description={Rapid private movement of funds out of a country (or out of domestic investment into safer assets), often triggered by crisis, political conflict, or expectations of devaluation. Capital flight can force currency pressure, reserve loss, and harsher adjustment.}
}

\newglossaryentry{exchange-rate-pass-through}{
  name={exchange-rate pass-through},
  sort={exchange rate pass-through},
  description={The extent to which a currency devaluation raises domestic prices, especially for imported essentials (fuel, fertiliser, medicine, machinery). High pass-through can turn devaluation into immediate inflation and real-wage cuts.}
}

\newglossaryentry{inflation-targeting}{
  name={inflation targeting},
  sort={inflation targeting},
  description={A monetary-policy framework where the central bank prioritises hitting an inflation target, typically via interest-rate moves. Critics argue it can treat inflation as a purely monetary phenomenon while ignoring supply shocks, monopoly pricing, and import dependence.}
}

\newglossaryentry{protectionism}{
  name={protectionism},
  sort={protectionism},
  description={Using tariffs, quotas, licensing, local-content rules, or public procurement to shelter domestic producers from foreign competition. It can defend jobs and industrial capacity, but its effects depend on who controls protected firms, how prices/wages move, and whether technology/inputs are domestically available.}
}

\newglossaryentry{trade-liberalisation}{
  name={trade liberalisation},
  sort={trade liberalisation},
  description={Reducing tariffs, quotas, and other trade barriers. It is often sold as “efficiency,” but in unequal world markets it can accelerate deindustrialisation, worsen trade deficits, and deepen dependence on imported inputs and foreign currency.}
}

\newglossaryentry{import-substitution}{
  name={import-substitution industrialisation (ISI)},
  sort={import substitution industrialisation},
  description={A strategy to replace imports with domestic production through tariffs, credit allocation, industrial policy, and state procurement. ISI can build capacity, but it often hits constraints around technology, energy, foreign exchange, and class control of investment decisions.}
}

\newglossaryentry{qe-term}{
  name={Quantitative easing (QE)},
  sort={Quantitative easing},
  description={A central bank policy of purchasing government bonds and/or other financial assets to expand its balance sheet and push down longer-term interest rates. QE can stabilise financial markets, but it often inflates asset prices and does not automatically translate into productive investment or higher wages.}
}

\newglossaryentry{mmt-term}{
  name={Modern Monetary Theory (MMT)},
  sort={Modern Monetary Theory},
  description={A heterodox framework arguing that a state that issues its own currency cannot “run out” of money in the way households can; the binding constraints are real resources, productive capacity, and inflation dynamics. It stresses the role of taxation and bond issuance in managing demand, distribution, and monetary conditions rather than “funding” spending in a mechanical sense.}
}

\newglossaryentry{ubi-term}{
  name={Universal basic income (UBI)},
  sort={Universal basic income},
  description={An unconditional cash transfer to all residents or citizens. Proposals differ sharply: some are designed to replace welfare and subsidise low wages, while others are framed as an income floor that complements strong public services, labour rights, and decommodification.}
}

\newglossaryentry{ubs-term}{
  name={Universal basic services (UBS)},
  sort={Universal basic services},
  description={A model of guaranteeing key services—health, education, housing, transport, care, water/energy—through public provision or social rights rather than cash transfers. UBS centres decommodification and collective infrastructure, but requires fiscal capacity, democratic control, and organised labour to prevent deterioration or capture.}
}

\newglossaryentry{jg-term}{
  name={Job Guarantee (JG)},
  sort={Job Guarantee},
  description={A proposal that the state offers a public job at a socially defined wage to anyone willing to work. Advocates treat it as an employment floor and stabiliser; critics debate job quality, political control, and whether it can be insulated from austerity and patronage without strong democratic governance.}
}

\newglossaryentry{esg-term}{
  name={ESG},
  sort={ESG},
  description={A framework used by investors and firms to score “environmental, social, and governance” performance. ESG can pressure disclosure and some standards, but it is often criticised as compatible with continued extraction and financialisation, turning ecological crisis into a portfolio and reputational management problem.}
}

\newglossaryentry{troika}{
  name={the Troika},
  sort={Troika},
  description={A term commonly used for the European Commission (EC), the European Central Bank (ECB), and the IMF acting jointly in crisis programmes and conditional lending in the Eurozone.}
}

\newglossaryentry{eurozone}{
  name={Eurozone},
  sort={Eurozone},
  description={The group of EU member states using the euro. Eurozone membership removes independent monetary policy and exchange-rate adjustment, making fiscal policy and wage/price dynamics central sites of “internal devaluation” during crises.}
}

\newglossaryentry{syriza}{
  name={Syriza},
  sort={Syriza},
  description={A Greek left party (Coalition of the Radical Left) that came to power in 2015 on an anti-austerity mandate. Its confrontation with the Troika became a major reference point for debates on debt, monetary sovereignty, and the limits imposed by Eurozone institutions.}
}

% ----------------------------------------------------------------------
% 2) PRINTING MODE (when % ============================
%  glossary.tex
%  Acronyms + glossary entries (glossaries-extra, no-index workflow)
%  ALSO prints Appendix B when % ============================
%  glossary.tex
%  Acronyms + glossary entries (glossaries-extra, no-index workflow)
%  ALSO prints Appendix B when \input{glossary} is used in content.tex.
%
%  IMPORTANT (preamble.tex):
%    \usepackage[acronym]{glossaries-extra}
%    \makenoidxglossaries
%    \def\GLOSSARYENTRIESONLY{}%
%    \loadglsentries{glossary}
%    \let\GLOSSARYENTRIESONLY\undefined
% ============================

\ProvidesFile{glossary.tex}[Glossary and acronym entries + Appendix printing]

% ----------------------------------------------------------------------
% 1) ENTRIES-ONLY MODE (when loaded in the preamble)
% ----------------------------------------------------------------------
\ifdefined\GLOSSARYENTRIESONLY

% ---------- Acronyms ----------
\newacronym{ai}{AI}{artificial intelligence}
\newacronym{ltv}{LTV}{labour theory of value}
\newacronym{snlt}{SNLT}{socially necessary labour time}
\newacronym{melt}{MELT}{monetary expression of labour time}
\newacronym{occ}{OCC}{organic composition of capital}
\newacronym{trpf}{TRPF}{tendency of the rate of profit to fall}

\newacronym{ubi}{UBI}{universal basic income}
\newacronym{ubs}{UBS}{universal basic services}
\newacronym{qe}{QE}{quantitative easing}
\newacronym{mmt}{MMT}{Modern Monetary Theory}
\newacronym{jg}{JG}{Job Guarantee}
\newacronym{esg}{ESG}{environmental, social, and governance}

\newacronym{imf}{IMF}{International Monetary Fund}
\newacronym{ipcc}{IPCC}{Intergovernmental Panel on Climate Change}

% ---------- Glossary terms (existing) ----------
\newglossaryentry{keynesianism}{
  name={Keynesianism},
  sort={Keynesianism},
  description={A macroeconomic approach associated with John Maynard Keynes that emphasises stabilising output and employment via demand management (especially fiscal policy and public spending), particularly during downturns.},
  first={Keynesianism (a macroeconomic approach that emphasises demand management—especially fiscal policy and public spending—to stabilise output and employment)}
}

\newglossaryentry{monetarism}{
  name={Monetarism},
  sort={Monetarism},
  description={A macroeconomic doctrine associated with Milton Friedman that prioritises controlling inflation by managing the money supply and/or interest rates, often via tight monetary policy.},
  first={Monetarism (a macroeconomic doctrine that prioritises inflation control via tight monetary policy—money supply and/or interest rates)}
}

\newglossaryentry{financialisation}{
  name={financialisation},
  sort={financialisation},
  description={A pattern in which profits, strategies, and power shift toward finance, asset price inflation, and rent extraction, rather than expanded productive investment and wage growth.}
}

\newglossaryentry{fictitious-capital}{
  name={fictitious capital},
  sort={fictitious capital},
  description={Tradable claims on future income streams (shares, bonds, securitised claims, many derivatives) whose market valuation can expand beyond surplus value currently produced, until crises force devaluation.}
}

\newglossaryentry{decommodification}{
  name={decommodification},
  sort={decommodification},
  description={Shifting access to essentials (housing, health, care, transport, energy, water) out of the market and away from ability to pay, toward rights-based provision.}
}

\newglossaryentry{capital-controls}{
  name={capital controls},
  sort={capital controls},
  description={Regulatory restrictions on cross-border movement of capital designed to limit capital flight, currency pressure, and the ability of owners to discipline reforms through financial exit.}
}

% ---------- Glossary terms (extended, targeted additions) ----------
\newglossaryentry{austerity}{
  name={austerity},
  sort={austerity},
  description={A policy package of spending cuts, hiring freezes, welfare retrenchment, regressive taxation, and/or user fees justified as “fiscal discipline”. In practice it often shifts crisis costs onto workers and the poor while protecting creditors and asset owners.}
}

\newglossaryentry{fiscal-consolidation}{
  name={fiscal consolidation},
  sort={fiscal consolidation},
  description={Reducing government deficits through spending cuts and/or tax rises. It is frequently presented as technocratic “budget repair,” but its class content depends on who is taxed, which services are cut, and whether interest payments to creditors are ring-fenced.}
}

\newglossaryentry{primary-balance}{
  name={primary balance},
  sort={primary balance},
  description={A government’s fiscal balance excluding interest payments on existing debt. A “primary surplus” can coexist with rising total debt burdens if interest costs remain high or growth is weak.}
}

\newglossaryentry{debt-service}{
  name={debt servicing},
  sort={debt servicing},
  description={Ongoing payments of interest and principal on debt. For many states, debt service becomes a prior claim on public revenue, structurally pressuring social spending and investment even without an explicit austerity programme.}
}

\newglossaryentry{conditionality}{
  name={conditionality},
  sort={conditionality},
  description={Policy conditions attached to loans or debt restructuring (commonly by the IMF or creditor blocs). Typical conditions include subsidy cuts, wage restraint, privatisation, deregulation, central bank “independence,” and fiscal consolidation.}
}

\newglossaryentry{structural-adjustment}{
  name={structural adjustment},
  sort={structural adjustment},
  description={A reform programme—historically associated with IMF/World Bank lending—that restructures economies toward export orientation, market pricing, privatisation, and reduced public provision. The “adjustment” is usually borne through depressed wages, weakened labour protections, and reduced social spending.}
}

\newglossaryentry{balance-of-payments}{
  name={balance of payments},
  sort={balance of payments},
  description={A country’s accounting of transactions with the rest of the world (trade in goods/services, income flows, and financial transfers). Persistent deficits often create pressure for devaluation, import compression, and external borrowing.}
}

\newglossaryentry{capital-flight}{
  name={capital flight},
  sort={capital flight},
  description={Rapid private movement of funds out of a country (or out of domestic investment into safer assets), often triggered by crisis, political conflict, or expectations of devaluation. Capital flight can force currency pressure, reserve loss, and harsher adjustment.}
}

\newglossaryentry{exchange-rate-pass-through}{
  name={exchange-rate pass-through},
  sort={exchange rate pass-through},
  description={The extent to which a currency devaluation raises domestic prices, especially for imported essentials (fuel, fertiliser, medicine, machinery). High pass-through can turn devaluation into immediate inflation and real-wage cuts.}
}

\newglossaryentry{inflation-targeting}{
  name={inflation targeting},
  sort={inflation targeting},
  description={A monetary-policy framework where the central bank prioritises hitting an inflation target, typically via interest-rate moves. Critics argue it can treat inflation as a purely monetary phenomenon while ignoring supply shocks, monopoly pricing, and import dependence.}
}

\newglossaryentry{protectionism}{
  name={protectionism},
  sort={protectionism},
  description={Using tariffs, quotas, licensing, local-content rules, or public procurement to shelter domestic producers from foreign competition. It can defend jobs and industrial capacity, but its effects depend on who controls protected firms, how prices/wages move, and whether technology/inputs are domestically available.}
}

\newglossaryentry{trade-liberalisation}{
  name={trade liberalisation},
  sort={trade liberalisation},
  description={Reducing tariffs, quotas, and other trade barriers. It is often sold as “efficiency,” but in unequal world markets it can accelerate deindustrialisation, worsen trade deficits, and deepen dependence on imported inputs and foreign currency.}
}

\newglossaryentry{import-substitution}{
  name={import-substitution industrialisation (ISI)},
  sort={import substitution industrialisation},
  description={A strategy to replace imports with domestic production through tariffs, credit allocation, industrial policy, and state procurement. ISI can build capacity, but it often hits constraints around technology, energy, foreign exchange, and class control of investment decisions.}
}

\newglossaryentry{qe-term}{
  name={Quantitative easing (QE)},
  sort={Quantitative easing},
  description={A central bank policy of purchasing government bonds and/or other financial assets to expand its balance sheet and push down longer-term interest rates. QE can stabilise financial markets, but it often inflates asset prices and does not automatically translate into productive investment or higher wages.}
}

\newglossaryentry{mmt-term}{
  name={Modern Monetary Theory (MMT)},
  sort={Modern Monetary Theory},
  description={A heterodox framework arguing that a state that issues its own currency cannot “run out” of money in the way households can; the binding constraints are real resources, productive capacity, and inflation dynamics. It stresses the role of taxation and bond issuance in managing demand, distribution, and monetary conditions rather than “funding” spending in a mechanical sense.}
}

\newglossaryentry{ubi-term}{
  name={Universal basic income (UBI)},
  sort={Universal basic income},
  description={An unconditional cash transfer to all residents or citizens. Proposals differ sharply: some are designed to replace welfare and subsidise low wages, while others are framed as an income floor that complements strong public services, labour rights, and decommodification.}
}

\newglossaryentry{ubs-term}{
  name={Universal basic services (UBS)},
  sort={Universal basic services},
  description={A model of guaranteeing key services—health, education, housing, transport, care, water/energy—through public provision or social rights rather than cash transfers. UBS centres decommodification and collective infrastructure, but requires fiscal capacity, democratic control, and organised labour to prevent deterioration or capture.}
}

\newglossaryentry{jg-term}{
  name={Job Guarantee (JG)},
  sort={Job Guarantee},
  description={A proposal that the state offers a public job at a socially defined wage to anyone willing to work. Advocates treat it as an employment floor and stabiliser; critics debate job quality, political control, and whether it can be insulated from austerity and patronage without strong democratic governance.}
}

\newglossaryentry{esg-term}{
  name={ESG},
  sort={ESG},
  description={A framework used by investors and firms to score “environmental, social, and governance” performance. ESG can pressure disclosure and some standards, but it is often criticised as compatible with continued extraction and financialisation, turning ecological crisis into a portfolio and reputational management problem.}
}

\newglossaryentry{troika}{
  name={the Troika},
  sort={Troika},
  description={A term commonly used for the European Commission (EC), the European Central Bank (ECB), and the IMF acting jointly in crisis programmes and conditional lending in the Eurozone.}
}

\newglossaryentry{eurozone}{
  name={Eurozone},
  sort={Eurozone},
  description={The group of EU member states using the euro. Eurozone membership removes independent monetary policy and exchange-rate adjustment, making fiscal policy and wage/price dynamics central sites of “internal devaluation” during crises.}
}

\newglossaryentry{syriza}{
  name={Syriza},
  sort={Syriza},
  description={A Greek left party (Coalition of the Radical Left) that came to power in 2015 on an anti-austerity mandate. Its confrontation with the Troika became a major reference point for debates on debt, monetary sovereignty, and the limits imposed by Eurozone institutions.}
}

% ----------------------------------------------------------------------
% 2) PRINTING MODE (when \input{glossary} is called in content.tex)
% ----------------------------------------------------------------------
\else

\section{Glossary and acronyms}
\label{sec:glossary}

\begingroup
\setlength{\parskip}{0pt}

% Print even if entries were not referenced yet:
\glsaddallunused

% Acronyms
\printnoidxglossary[type=\acronymtype,style=compactgls,title={Acronyms}]

\vspace{0.6\baselineskip}

% Terms
\printnoidxglossary[style=termscolon,title={Glossary of terms}]

\endgroup

\fi is used in content.tex.
%
%  IMPORTANT (preamble.tex):
%    \usepackage[acronym]{glossaries-extra}
%    \makenoidxglossaries
%    \def\GLOSSARYENTRIESONLY{}%
%    \loadglsentries{glossary}
%    \let\GLOSSARYENTRIESONLY\undefined
% ============================

\ProvidesFile{glossary.tex}[Glossary and acronym entries + Appendix printing]

% ----------------------------------------------------------------------
% 1) ENTRIES-ONLY MODE (when loaded in the preamble)
% ----------------------------------------------------------------------
\ifdefined\GLOSSARYENTRIESONLY

% ---------- Acronyms ----------
\newacronym{ai}{AI}{artificial intelligence}
\newacronym{ltv}{LTV}{labour theory of value}
\newacronym{snlt}{SNLT}{socially necessary labour time}
\newacronym{melt}{MELT}{monetary expression of labour time}
\newacronym{occ}{OCC}{organic composition of capital}
\newacronym{trpf}{TRPF}{tendency of the rate of profit to fall}

\newacronym{ubi}{UBI}{universal basic income}
\newacronym{ubs}{UBS}{universal basic services}
\newacronym{qe}{QE}{quantitative easing}
\newacronym{mmt}{MMT}{Modern Monetary Theory}
\newacronym{jg}{JG}{Job Guarantee}
\newacronym{esg}{ESG}{environmental, social, and governance}

\newacronym{imf}{IMF}{International Monetary Fund}
\newacronym{ipcc}{IPCC}{Intergovernmental Panel on Climate Change}

% ---------- Glossary terms (existing) ----------
\newglossaryentry{keynesianism}{
  name={Keynesianism},
  sort={Keynesianism},
  description={A macroeconomic approach associated with John Maynard Keynes that emphasises stabilising output and employment via demand management (especially fiscal policy and public spending), particularly during downturns.},
  first={Keynesianism (a macroeconomic approach that emphasises demand management—especially fiscal policy and public spending—to stabilise output and employment)}
}

\newglossaryentry{monetarism}{
  name={Monetarism},
  sort={Monetarism},
  description={A macroeconomic doctrine associated with Milton Friedman that prioritises controlling inflation by managing the money supply and/or interest rates, often via tight monetary policy.},
  first={Monetarism (a macroeconomic doctrine that prioritises inflation control via tight monetary policy—money supply and/or interest rates)}
}

\newglossaryentry{financialisation}{
  name={financialisation},
  sort={financialisation},
  description={A pattern in which profits, strategies, and power shift toward finance, asset price inflation, and rent extraction, rather than expanded productive investment and wage growth.}
}

\newglossaryentry{fictitious-capital}{
  name={fictitious capital},
  sort={fictitious capital},
  description={Tradable claims on future income streams (shares, bonds, securitised claims, many derivatives) whose market valuation can expand beyond surplus value currently produced, until crises force devaluation.}
}

\newglossaryentry{decommodification}{
  name={decommodification},
  sort={decommodification},
  description={Shifting access to essentials (housing, health, care, transport, energy, water) out of the market and away from ability to pay, toward rights-based provision.}
}

\newglossaryentry{capital-controls}{
  name={capital controls},
  sort={capital controls},
  description={Regulatory restrictions on cross-border movement of capital designed to limit capital flight, currency pressure, and the ability of owners to discipline reforms through financial exit.}
}

% ---------- Glossary terms (extended, targeted additions) ----------
\newglossaryentry{austerity}{
  name={austerity},
  sort={austerity},
  description={A policy package of spending cuts, hiring freezes, welfare retrenchment, regressive taxation, and/or user fees justified as “fiscal discipline”. In practice it often shifts crisis costs onto workers and the poor while protecting creditors and asset owners.}
}

\newglossaryentry{fiscal-consolidation}{
  name={fiscal consolidation},
  sort={fiscal consolidation},
  description={Reducing government deficits through spending cuts and/or tax rises. It is frequently presented as technocratic “budget repair,” but its class content depends on who is taxed, which services are cut, and whether interest payments to creditors are ring-fenced.}
}

\newglossaryentry{primary-balance}{
  name={primary balance},
  sort={primary balance},
  description={A government’s fiscal balance excluding interest payments on existing debt. A “primary surplus” can coexist with rising total debt burdens if interest costs remain high or growth is weak.}
}

\newglossaryentry{debt-service}{
  name={debt servicing},
  sort={debt servicing},
  description={Ongoing payments of interest and principal on debt. For many states, debt service becomes a prior claim on public revenue, structurally pressuring social spending and investment even without an explicit austerity programme.}
}

\newglossaryentry{conditionality}{
  name={conditionality},
  sort={conditionality},
  description={Policy conditions attached to loans or debt restructuring (commonly by the IMF or creditor blocs). Typical conditions include subsidy cuts, wage restraint, privatisation, deregulation, central bank “independence,” and fiscal consolidation.}
}

\newglossaryentry{structural-adjustment}{
  name={structural adjustment},
  sort={structural adjustment},
  description={A reform programme—historically associated with IMF/World Bank lending—that restructures economies toward export orientation, market pricing, privatisation, and reduced public provision. The “adjustment” is usually borne through depressed wages, weakened labour protections, and reduced social spending.}
}

\newglossaryentry{balance-of-payments}{
  name={balance of payments},
  sort={balance of payments},
  description={A country’s accounting of transactions with the rest of the world (trade in goods/services, income flows, and financial transfers). Persistent deficits often create pressure for devaluation, import compression, and external borrowing.}
}

\newglossaryentry{capital-flight}{
  name={capital flight},
  sort={capital flight},
  description={Rapid private movement of funds out of a country (or out of domestic investment into safer assets), often triggered by crisis, political conflict, or expectations of devaluation. Capital flight can force currency pressure, reserve loss, and harsher adjustment.}
}

\newglossaryentry{exchange-rate-pass-through}{
  name={exchange-rate pass-through},
  sort={exchange rate pass-through},
  description={The extent to which a currency devaluation raises domestic prices, especially for imported essentials (fuel, fertiliser, medicine, machinery). High pass-through can turn devaluation into immediate inflation and real-wage cuts.}
}

\newglossaryentry{inflation-targeting}{
  name={inflation targeting},
  sort={inflation targeting},
  description={A monetary-policy framework where the central bank prioritises hitting an inflation target, typically via interest-rate moves. Critics argue it can treat inflation as a purely monetary phenomenon while ignoring supply shocks, monopoly pricing, and import dependence.}
}

\newglossaryentry{protectionism}{
  name={protectionism},
  sort={protectionism},
  description={Using tariffs, quotas, licensing, local-content rules, or public procurement to shelter domestic producers from foreign competition. It can defend jobs and industrial capacity, but its effects depend on who controls protected firms, how prices/wages move, and whether technology/inputs are domestically available.}
}

\newglossaryentry{trade-liberalisation}{
  name={trade liberalisation},
  sort={trade liberalisation},
  description={Reducing tariffs, quotas, and other trade barriers. It is often sold as “efficiency,” but in unequal world markets it can accelerate deindustrialisation, worsen trade deficits, and deepen dependence on imported inputs and foreign currency.}
}

\newglossaryentry{import-substitution}{
  name={import-substitution industrialisation (ISI)},
  sort={import substitution industrialisation},
  description={A strategy to replace imports with domestic production through tariffs, credit allocation, industrial policy, and state procurement. ISI can build capacity, but it often hits constraints around technology, energy, foreign exchange, and class control of investment decisions.}
}

\newglossaryentry{qe-term}{
  name={Quantitative easing (QE)},
  sort={Quantitative easing},
  description={A central bank policy of purchasing government bonds and/or other financial assets to expand its balance sheet and push down longer-term interest rates. QE can stabilise financial markets, but it often inflates asset prices and does not automatically translate into productive investment or higher wages.}
}

\newglossaryentry{mmt-term}{
  name={Modern Monetary Theory (MMT)},
  sort={Modern Monetary Theory},
  description={A heterodox framework arguing that a state that issues its own currency cannot “run out” of money in the way households can; the binding constraints are real resources, productive capacity, and inflation dynamics. It stresses the role of taxation and bond issuance in managing demand, distribution, and monetary conditions rather than “funding” spending in a mechanical sense.}
}

\newglossaryentry{ubi-term}{
  name={Universal basic income (UBI)},
  sort={Universal basic income},
  description={An unconditional cash transfer to all residents or citizens. Proposals differ sharply: some are designed to replace welfare and subsidise low wages, while others are framed as an income floor that complements strong public services, labour rights, and decommodification.}
}

\newglossaryentry{ubs-term}{
  name={Universal basic services (UBS)},
  sort={Universal basic services},
  description={A model of guaranteeing key services—health, education, housing, transport, care, water/energy—through public provision or social rights rather than cash transfers. UBS centres decommodification and collective infrastructure, but requires fiscal capacity, democratic control, and organised labour to prevent deterioration or capture.}
}

\newglossaryentry{jg-term}{
  name={Job Guarantee (JG)},
  sort={Job Guarantee},
  description={A proposal that the state offers a public job at a socially defined wage to anyone willing to work. Advocates treat it as an employment floor and stabiliser; critics debate job quality, political control, and whether it can be insulated from austerity and patronage without strong democratic governance.}
}

\newglossaryentry{esg-term}{
  name={ESG},
  sort={ESG},
  description={A framework used by investors and firms to score “environmental, social, and governance” performance. ESG can pressure disclosure and some standards, but it is often criticised as compatible with continued extraction and financialisation, turning ecological crisis into a portfolio and reputational management problem.}
}

\newglossaryentry{troika}{
  name={the Troika},
  sort={Troika},
  description={A term commonly used for the European Commission (EC), the European Central Bank (ECB), and the IMF acting jointly in crisis programmes and conditional lending in the Eurozone.}
}

\newglossaryentry{eurozone}{
  name={Eurozone},
  sort={Eurozone},
  description={The group of EU member states using the euro. Eurozone membership removes independent monetary policy and exchange-rate adjustment, making fiscal policy and wage/price dynamics central sites of “internal devaluation” during crises.}
}

\newglossaryentry{syriza}{
  name={Syriza},
  sort={Syriza},
  description={A Greek left party (Coalition of the Radical Left) that came to power in 2015 on an anti-austerity mandate. Its confrontation with the Troika became a major reference point for debates on debt, monetary sovereignty, and the limits imposed by Eurozone institutions.}
}

% ----------------------------------------------------------------------
% 2) PRINTING MODE (when % ============================
%  glossary.tex
%  Acronyms + glossary entries (glossaries-extra, no-index workflow)
%  ALSO prints Appendix B when \input{glossary} is used in content.tex.
%
%  IMPORTANT (preamble.tex):
%    \usepackage[acronym]{glossaries-extra}
%    \makenoidxglossaries
%    \def\GLOSSARYENTRIESONLY{}%
%    \loadglsentries{glossary}
%    \let\GLOSSARYENTRIESONLY\undefined
% ============================

\ProvidesFile{glossary.tex}[Glossary and acronym entries + Appendix printing]

% ----------------------------------------------------------------------
% 1) ENTRIES-ONLY MODE (when loaded in the preamble)
% ----------------------------------------------------------------------
\ifdefined\GLOSSARYENTRIESONLY

% ---------- Acronyms ----------
\newacronym{ai}{AI}{artificial intelligence}
\newacronym{ltv}{LTV}{labour theory of value}
\newacronym{snlt}{SNLT}{socially necessary labour time}
\newacronym{melt}{MELT}{monetary expression of labour time}
\newacronym{occ}{OCC}{organic composition of capital}
\newacronym{trpf}{TRPF}{tendency of the rate of profit to fall}

\newacronym{ubi}{UBI}{universal basic income}
\newacronym{ubs}{UBS}{universal basic services}
\newacronym{qe}{QE}{quantitative easing}
\newacronym{mmt}{MMT}{Modern Monetary Theory}
\newacronym{jg}{JG}{Job Guarantee}
\newacronym{esg}{ESG}{environmental, social, and governance}

\newacronym{imf}{IMF}{International Monetary Fund}
\newacronym{ipcc}{IPCC}{Intergovernmental Panel on Climate Change}

% ---------- Glossary terms (existing) ----------
\newglossaryentry{keynesianism}{
  name={Keynesianism},
  sort={Keynesianism},
  description={A macroeconomic approach associated with John Maynard Keynes that emphasises stabilising output and employment via demand management (especially fiscal policy and public spending), particularly during downturns.},
  first={Keynesianism (a macroeconomic approach that emphasises demand management—especially fiscal policy and public spending—to stabilise output and employment)}
}

\newglossaryentry{monetarism}{
  name={Monetarism},
  sort={Monetarism},
  description={A macroeconomic doctrine associated with Milton Friedman that prioritises controlling inflation by managing the money supply and/or interest rates, often via tight monetary policy.},
  first={Monetarism (a macroeconomic doctrine that prioritises inflation control via tight monetary policy—money supply and/or interest rates)}
}

\newglossaryentry{financialisation}{
  name={financialisation},
  sort={financialisation},
  description={A pattern in which profits, strategies, and power shift toward finance, asset price inflation, and rent extraction, rather than expanded productive investment and wage growth.}
}

\newglossaryentry{fictitious-capital}{
  name={fictitious capital},
  sort={fictitious capital},
  description={Tradable claims on future income streams (shares, bonds, securitised claims, many derivatives) whose market valuation can expand beyond surplus value currently produced, until crises force devaluation.}
}

\newglossaryentry{decommodification}{
  name={decommodification},
  sort={decommodification},
  description={Shifting access to essentials (housing, health, care, transport, energy, water) out of the market and away from ability to pay, toward rights-based provision.}
}

\newglossaryentry{capital-controls}{
  name={capital controls},
  sort={capital controls},
  description={Regulatory restrictions on cross-border movement of capital designed to limit capital flight, currency pressure, and the ability of owners to discipline reforms through financial exit.}
}

% ---------- Glossary terms (extended, targeted additions) ----------
\newglossaryentry{austerity}{
  name={austerity},
  sort={austerity},
  description={A policy package of spending cuts, hiring freezes, welfare retrenchment, regressive taxation, and/or user fees justified as “fiscal discipline”. In practice it often shifts crisis costs onto workers and the poor while protecting creditors and asset owners.}
}

\newglossaryentry{fiscal-consolidation}{
  name={fiscal consolidation},
  sort={fiscal consolidation},
  description={Reducing government deficits through spending cuts and/or tax rises. It is frequently presented as technocratic “budget repair,” but its class content depends on who is taxed, which services are cut, and whether interest payments to creditors are ring-fenced.}
}

\newglossaryentry{primary-balance}{
  name={primary balance},
  sort={primary balance},
  description={A government’s fiscal balance excluding interest payments on existing debt. A “primary surplus” can coexist with rising total debt burdens if interest costs remain high or growth is weak.}
}

\newglossaryentry{debt-service}{
  name={debt servicing},
  sort={debt servicing},
  description={Ongoing payments of interest and principal on debt. For many states, debt service becomes a prior claim on public revenue, structurally pressuring social spending and investment even without an explicit austerity programme.}
}

\newglossaryentry{conditionality}{
  name={conditionality},
  sort={conditionality},
  description={Policy conditions attached to loans or debt restructuring (commonly by the IMF or creditor blocs). Typical conditions include subsidy cuts, wage restraint, privatisation, deregulation, central bank “independence,” and fiscal consolidation.}
}

\newglossaryentry{structural-adjustment}{
  name={structural adjustment},
  sort={structural adjustment},
  description={A reform programme—historically associated with IMF/World Bank lending—that restructures economies toward export orientation, market pricing, privatisation, and reduced public provision. The “adjustment” is usually borne through depressed wages, weakened labour protections, and reduced social spending.}
}

\newglossaryentry{balance-of-payments}{
  name={balance of payments},
  sort={balance of payments},
  description={A country’s accounting of transactions with the rest of the world (trade in goods/services, income flows, and financial transfers). Persistent deficits often create pressure for devaluation, import compression, and external borrowing.}
}

\newglossaryentry{capital-flight}{
  name={capital flight},
  sort={capital flight},
  description={Rapid private movement of funds out of a country (or out of domestic investment into safer assets), often triggered by crisis, political conflict, or expectations of devaluation. Capital flight can force currency pressure, reserve loss, and harsher adjustment.}
}

\newglossaryentry{exchange-rate-pass-through}{
  name={exchange-rate pass-through},
  sort={exchange rate pass-through},
  description={The extent to which a currency devaluation raises domestic prices, especially for imported essentials (fuel, fertiliser, medicine, machinery). High pass-through can turn devaluation into immediate inflation and real-wage cuts.}
}

\newglossaryentry{inflation-targeting}{
  name={inflation targeting},
  sort={inflation targeting},
  description={A monetary-policy framework where the central bank prioritises hitting an inflation target, typically via interest-rate moves. Critics argue it can treat inflation as a purely monetary phenomenon while ignoring supply shocks, monopoly pricing, and import dependence.}
}

\newglossaryentry{protectionism}{
  name={protectionism},
  sort={protectionism},
  description={Using tariffs, quotas, licensing, local-content rules, or public procurement to shelter domestic producers from foreign competition. It can defend jobs and industrial capacity, but its effects depend on who controls protected firms, how prices/wages move, and whether technology/inputs are domestically available.}
}

\newglossaryentry{trade-liberalisation}{
  name={trade liberalisation},
  sort={trade liberalisation},
  description={Reducing tariffs, quotas, and other trade barriers. It is often sold as “efficiency,” but in unequal world markets it can accelerate deindustrialisation, worsen trade deficits, and deepen dependence on imported inputs and foreign currency.}
}

\newglossaryentry{import-substitution}{
  name={import-substitution industrialisation (ISI)},
  sort={import substitution industrialisation},
  description={A strategy to replace imports with domestic production through tariffs, credit allocation, industrial policy, and state procurement. ISI can build capacity, but it often hits constraints around technology, energy, foreign exchange, and class control of investment decisions.}
}

\newglossaryentry{qe-term}{
  name={Quantitative easing (QE)},
  sort={Quantitative easing},
  description={A central bank policy of purchasing government bonds and/or other financial assets to expand its balance sheet and push down longer-term interest rates. QE can stabilise financial markets, but it often inflates asset prices and does not automatically translate into productive investment or higher wages.}
}

\newglossaryentry{mmt-term}{
  name={Modern Monetary Theory (MMT)},
  sort={Modern Monetary Theory},
  description={A heterodox framework arguing that a state that issues its own currency cannot “run out” of money in the way households can; the binding constraints are real resources, productive capacity, and inflation dynamics. It stresses the role of taxation and bond issuance in managing demand, distribution, and monetary conditions rather than “funding” spending in a mechanical sense.}
}

\newglossaryentry{ubi-term}{
  name={Universal basic income (UBI)},
  sort={Universal basic income},
  description={An unconditional cash transfer to all residents or citizens. Proposals differ sharply: some are designed to replace welfare and subsidise low wages, while others are framed as an income floor that complements strong public services, labour rights, and decommodification.}
}

\newglossaryentry{ubs-term}{
  name={Universal basic services (UBS)},
  sort={Universal basic services},
  description={A model of guaranteeing key services—health, education, housing, transport, care, water/energy—through public provision or social rights rather than cash transfers. UBS centres decommodification and collective infrastructure, but requires fiscal capacity, democratic control, and organised labour to prevent deterioration or capture.}
}

\newglossaryentry{jg-term}{
  name={Job Guarantee (JG)},
  sort={Job Guarantee},
  description={A proposal that the state offers a public job at a socially defined wage to anyone willing to work. Advocates treat it as an employment floor and stabiliser; critics debate job quality, political control, and whether it can be insulated from austerity and patronage without strong democratic governance.}
}

\newglossaryentry{esg-term}{
  name={ESG},
  sort={ESG},
  description={A framework used by investors and firms to score “environmental, social, and governance” performance. ESG can pressure disclosure and some standards, but it is often criticised as compatible with continued extraction and financialisation, turning ecological crisis into a portfolio and reputational management problem.}
}

\newglossaryentry{troika}{
  name={the Troika},
  sort={Troika},
  description={A term commonly used for the European Commission (EC), the European Central Bank (ECB), and the IMF acting jointly in crisis programmes and conditional lending in the Eurozone.}
}

\newglossaryentry{eurozone}{
  name={Eurozone},
  sort={Eurozone},
  description={The group of EU member states using the euro. Eurozone membership removes independent monetary policy and exchange-rate adjustment, making fiscal policy and wage/price dynamics central sites of “internal devaluation” during crises.}
}

\newglossaryentry{syriza}{
  name={Syriza},
  sort={Syriza},
  description={A Greek left party (Coalition of the Radical Left) that came to power in 2015 on an anti-austerity mandate. Its confrontation with the Troika became a major reference point for debates on debt, monetary sovereignty, and the limits imposed by Eurozone institutions.}
}

% ----------------------------------------------------------------------
% 2) PRINTING MODE (when \input{glossary} is called in content.tex)
% ----------------------------------------------------------------------
\else

\section{Glossary and acronyms}
\label{sec:glossary}

\begingroup
\setlength{\parskip}{0pt}

% Print even if entries were not referenced yet:
\glsaddallunused

% Acronyms
\printnoidxglossary[type=\acronymtype,style=compactgls,title={Acronyms}]

\vspace{0.6\baselineskip}

% Terms
\printnoidxglossary[style=termscolon,title={Glossary of terms}]

\endgroup

\fi is called in content.tex)
% ----------------------------------------------------------------------
\else

\section{Glossary and acronyms}
\label{sec:glossary}

\begingroup
\setlength{\parskip}{0pt}

% Print even if entries were not referenced yet:
\glsaddallunused

% Acronyms
\printnoidxglossary[type=\acronymtype,style=compactgls,title={Acronyms}]

\vspace{0.6\baselineskip}

% Terms
\printnoidxglossary[style=termscolon,title={Glossary of terms}]

\endgroup

\fi is called in content.tex)
% ----------------------------------------------------------------------
\else

\section{Glossary and acronyms}
\label{sec:glossary}

\begingroup
\setlength{\parskip}{0pt}

% Print even if entries were not referenced yet:
\glsaddallunused

% Acronyms
\printnoidxglossary[type=\acronymtype,style=compactgls,title={Acronyms}]

\vspace{0.6\baselineskip}

% Terms
\printnoidxglossary[style=termscolon,title={Glossary of terms}]

\endgroup

\fi is used in content.tex.
%
%  IMPORTANT (preamble.tex):
%    \usepackage[acronym]{glossaries-extra}
%    \makenoidxglossaries
%    \def\GLOSSARYENTRIESONLY{}%
%    \loadglsentries{glossary}
%    \let\GLOSSARYENTRIESONLY\undefined
% ============================

\ProvidesFile{glossary.tex}[Glossary and acronym entries + Appendix printing]

% ----------------------------------------------------------------------
% 1) ENTRIES-ONLY MODE (when loaded in the preamble)
% ----------------------------------------------------------------------
\ifdefined\GLOSSARYENTRIESONLY

% ---------- Acronyms ----------
\newacronym{ai}{AI}{artificial intelligence}
\newacronym{ltv}{LTV}{labour theory of value}
\newacronym{snlt}{SNLT}{socially necessary labour time}
\newacronym{melt}{MELT}{monetary expression of labour time}
\newacronym{occ}{OCC}{organic composition of capital}
\newacronym{trpf}{TRPF}{tendency of the rate of profit to fall}

\newacronym{ubi}{UBI}{universal basic income}
\newacronym{ubs}{UBS}{universal basic services}
\newacronym{qe}{QE}{quantitative easing}
\newacronym{mmt}{MMT}{Modern Monetary Theory}
\newacronym{jg}{JG}{Job Guarantee}
\newacronym{esg}{ESG}{environmental, social, and governance}

\newacronym{imf}{IMF}{International Monetary Fund}
\newacronym{ipcc}{IPCC}{Intergovernmental Panel on Climate Change}

% ---------- Glossary terms (existing) ----------
\newglossaryentry{keynesianism}{
  name={Keynesianism},
  sort={Keynesianism},
  description={A macroeconomic approach associated with John Maynard Keynes that emphasises stabilising output and employment via demand management (especially fiscal policy and public spending), particularly during downturns.},
  first={Keynesianism (a macroeconomic approach that emphasises demand management—especially fiscal policy and public spending—to stabilise output and employment)}
}

\newglossaryentry{monetarism}{
  name={Monetarism},
  sort={Monetarism},
  description={A macroeconomic doctrine associated with Milton Friedman that prioritises controlling inflation by managing the money supply and/or interest rates, often via tight monetary policy.},
  first={Monetarism (a macroeconomic doctrine that prioritises inflation control via tight monetary policy—money supply and/or interest rates)}
}

\newglossaryentry{financialisation}{
  name={financialisation},
  sort={financialisation},
  description={A pattern in which profits, strategies, and power shift toward finance, asset price inflation, and rent extraction, rather than expanded productive investment and wage growth.}
}

\newglossaryentry{fictitious-capital}{
  name={fictitious capital},
  sort={fictitious capital},
  description={Tradable claims on future income streams (shares, bonds, securitised claims, many derivatives) whose market valuation can expand beyond surplus value currently produced, until crises force devaluation.}
}

\newglossaryentry{decommodification}{
  name={decommodification},
  sort={decommodification},
  description={Shifting access to essentials (housing, health, care, transport, energy, water) out of the market and away from ability to pay, toward rights-based provision.}
}

\newglossaryentry{capital-controls}{
  name={capital controls},
  sort={capital controls},
  description={Regulatory restrictions on cross-border movement of capital designed to limit capital flight, currency pressure, and the ability of owners to discipline reforms through financial exit.}
}

% ---------- Glossary terms (extended, targeted additions) ----------
\newglossaryentry{austerity}{
  name={austerity},
  sort={austerity},
  description={A policy package of spending cuts, hiring freezes, welfare retrenchment, regressive taxation, and/or user fees justified as “fiscal discipline”. In practice it often shifts crisis costs onto workers and the poor while protecting creditors and asset owners.}
}

\newglossaryentry{fiscal-consolidation}{
  name={fiscal consolidation},
  sort={fiscal consolidation},
  description={Reducing government deficits through spending cuts and/or tax rises. It is frequently presented as technocratic “budget repair,” but its class content depends on who is taxed, which services are cut, and whether interest payments to creditors are ring-fenced.}
}

\newglossaryentry{primary-balance}{
  name={primary balance},
  sort={primary balance},
  description={A government’s fiscal balance excluding interest payments on existing debt. A “primary surplus” can coexist with rising total debt burdens if interest costs remain high or growth is weak.}
}

\newglossaryentry{debt-service}{
  name={debt servicing},
  sort={debt servicing},
  description={Ongoing payments of interest and principal on debt. For many states, debt service becomes a prior claim on public revenue, structurally pressuring social spending and investment even without an explicit austerity programme.}
}

\newglossaryentry{conditionality}{
  name={conditionality},
  sort={conditionality},
  description={Policy conditions attached to loans or debt restructuring (commonly by the IMF or creditor blocs). Typical conditions include subsidy cuts, wage restraint, privatisation, deregulation, central bank “independence,” and fiscal consolidation.}
}

\newglossaryentry{structural-adjustment}{
  name={structural adjustment},
  sort={structural adjustment},
  description={A reform programme—historically associated with IMF/World Bank lending—that restructures economies toward export orientation, market pricing, privatisation, and reduced public provision. The “adjustment” is usually borne through depressed wages, weakened labour protections, and reduced social spending.}
}

\newglossaryentry{balance-of-payments}{
  name={balance of payments},
  sort={balance of payments},
  description={A country’s accounting of transactions with the rest of the world (trade in goods/services, income flows, and financial transfers). Persistent deficits often create pressure for devaluation, import compression, and external borrowing.}
}

\newglossaryentry{capital-flight}{
  name={capital flight},
  sort={capital flight},
  description={Rapid private movement of funds out of a country (or out of domestic investment into safer assets), often triggered by crisis, political conflict, or expectations of devaluation. Capital flight can force currency pressure, reserve loss, and harsher adjustment.}
}

\newglossaryentry{exchange-rate-pass-through}{
  name={exchange-rate pass-through},
  sort={exchange rate pass-through},
  description={The extent to which a currency devaluation raises domestic prices, especially for imported essentials (fuel, fertiliser, medicine, machinery). High pass-through can turn devaluation into immediate inflation and real-wage cuts.}
}

\newglossaryentry{inflation-targeting}{
  name={inflation targeting},
  sort={inflation targeting},
  description={A monetary-policy framework where the central bank prioritises hitting an inflation target, typically via interest-rate moves. Critics argue it can treat inflation as a purely monetary phenomenon while ignoring supply shocks, monopoly pricing, and import dependence.}
}

\newglossaryentry{protectionism}{
  name={protectionism},
  sort={protectionism},
  description={Using tariffs, quotas, licensing, local-content rules, or public procurement to shelter domestic producers from foreign competition. It can defend jobs and industrial capacity, but its effects depend on who controls protected firms, how prices/wages move, and whether technology/inputs are domestically available.}
}

\newglossaryentry{trade-liberalisation}{
  name={trade liberalisation},
  sort={trade liberalisation},
  description={Reducing tariffs, quotas, and other trade barriers. It is often sold as “efficiency,” but in unequal world markets it can accelerate deindustrialisation, worsen trade deficits, and deepen dependence on imported inputs and foreign currency.}
}

\newglossaryentry{import-substitution}{
  name={import-substitution industrialisation (ISI)},
  sort={import substitution industrialisation},
  description={A strategy to replace imports with domestic production through tariffs, credit allocation, industrial policy, and state procurement. ISI can build capacity, but it often hits constraints around technology, energy, foreign exchange, and class control of investment decisions.}
}

\newglossaryentry{qe-term}{
  name={Quantitative easing (QE)},
  sort={Quantitative easing},
  description={A central bank policy of purchasing government bonds and/or other financial assets to expand its balance sheet and push down longer-term interest rates. QE can stabilise financial markets, but it often inflates asset prices and does not automatically translate into productive investment or higher wages.}
}

\newglossaryentry{mmt-term}{
  name={Modern Monetary Theory (MMT)},
  sort={Modern Monetary Theory},
  description={A heterodox framework arguing that a state that issues its own currency cannot “run out” of money in the way households can; the binding constraints are real resources, productive capacity, and inflation dynamics. It stresses the role of taxation and bond issuance in managing demand, distribution, and monetary conditions rather than “funding” spending in a mechanical sense.}
}

\newglossaryentry{ubi-term}{
  name={Universal basic income (UBI)},
  sort={Universal basic income},
  description={An unconditional cash transfer to all residents or citizens. Proposals differ sharply: some are designed to replace welfare and subsidise low wages, while others are framed as an income floor that complements strong public services, labour rights, and decommodification.}
}

\newglossaryentry{ubs-term}{
  name={Universal basic services (UBS)},
  sort={Universal basic services},
  description={A model of guaranteeing key services—health, education, housing, transport, care, water/energy—through public provision or social rights rather than cash transfers. UBS centres decommodification and collective infrastructure, but requires fiscal capacity, democratic control, and organised labour to prevent deterioration or capture.}
}

\newglossaryentry{jg-term}{
  name={Job Guarantee (JG)},
  sort={Job Guarantee},
  description={A proposal that the state offers a public job at a socially defined wage to anyone willing to work. Advocates treat it as an employment floor and stabiliser; critics debate job quality, political control, and whether it can be insulated from austerity and patronage without strong democratic governance.}
}

\newglossaryentry{esg-term}{
  name={ESG},
  sort={ESG},
  description={A framework used by investors and firms to score “environmental, social, and governance” performance. ESG can pressure disclosure and some standards, but it is often criticised as compatible with continued extraction and financialisation, turning ecological crisis into a portfolio and reputational management problem.}
}

\newglossaryentry{troika}{
  name={the Troika},
  sort={Troika},
  description={A term commonly used for the European Commission (EC), the European Central Bank (ECB), and the IMF acting jointly in crisis programmes and conditional lending in the Eurozone.}
}

\newglossaryentry{eurozone}{
  name={Eurozone},
  sort={Eurozone},
  description={The group of EU member states using the euro. Eurozone membership removes independent monetary policy and exchange-rate adjustment, making fiscal policy and wage/price dynamics central sites of “internal devaluation” during crises.}
}

\newglossaryentry{syriza}{
  name={Syriza},
  sort={Syriza},
  description={A Greek left party (Coalition of the Radical Left) that came to power in 2015 on an anti-austerity mandate. Its confrontation with the Troika became a major reference point for debates on debt, monetary sovereignty, and the limits imposed by Eurozone institutions.}
}

% ----------------------------------------------------------------------
% 2) PRINTING MODE (when % ============================
%  glossary.tex
%  Acronyms + glossary entries (glossaries-extra, no-index workflow)
%  ALSO prints Appendix B when % ============================
%  glossary.tex
%  Acronyms + glossary entries (glossaries-extra, no-index workflow)
%  ALSO prints Appendix B when % ============================
%  glossary.tex
%  Acronyms + glossary entries (glossaries-extra, no-index workflow)
%  ALSO prints Appendix B when \input{glossary} is used in content.tex.
%
%  IMPORTANT (preamble.tex):
%    \usepackage[acronym]{glossaries-extra}
%    \makenoidxglossaries
%    \def\GLOSSARYENTRIESONLY{}%
%    \loadglsentries{glossary}
%    \let\GLOSSARYENTRIESONLY\undefined
% ============================

\ProvidesFile{glossary.tex}[Glossary and acronym entries + Appendix printing]

% ----------------------------------------------------------------------
% 1) ENTRIES-ONLY MODE (when loaded in the preamble)
% ----------------------------------------------------------------------
\ifdefined\GLOSSARYENTRIESONLY

% ---------- Acronyms ----------
\newacronym{ai}{AI}{artificial intelligence}
\newacronym{ltv}{LTV}{labour theory of value}
\newacronym{snlt}{SNLT}{socially necessary labour time}
\newacronym{melt}{MELT}{monetary expression of labour time}
\newacronym{occ}{OCC}{organic composition of capital}
\newacronym{trpf}{TRPF}{tendency of the rate of profit to fall}

\newacronym{ubi}{UBI}{universal basic income}
\newacronym{ubs}{UBS}{universal basic services}
\newacronym{qe}{QE}{quantitative easing}
\newacronym{mmt}{MMT}{Modern Monetary Theory}
\newacronym{jg}{JG}{Job Guarantee}
\newacronym{esg}{ESG}{environmental, social, and governance}

\newacronym{imf}{IMF}{International Monetary Fund}
\newacronym{ipcc}{IPCC}{Intergovernmental Panel on Climate Change}

% ---------- Glossary terms (existing) ----------
\newglossaryentry{keynesianism}{
  name={Keynesianism},
  sort={Keynesianism},
  description={A macroeconomic approach associated with John Maynard Keynes that emphasises stabilising output and employment via demand management (especially fiscal policy and public spending), particularly during downturns.},
  first={Keynesianism (a macroeconomic approach that emphasises demand management—especially fiscal policy and public spending—to stabilise output and employment)}
}

\newglossaryentry{monetarism}{
  name={Monetarism},
  sort={Monetarism},
  description={A macroeconomic doctrine associated with Milton Friedman that prioritises controlling inflation by managing the money supply and/or interest rates, often via tight monetary policy.},
  first={Monetarism (a macroeconomic doctrine that prioritises inflation control via tight monetary policy—money supply and/or interest rates)}
}

\newglossaryentry{financialisation}{
  name={financialisation},
  sort={financialisation},
  description={A pattern in which profits, strategies, and power shift toward finance, asset price inflation, and rent extraction, rather than expanded productive investment and wage growth.}
}

\newglossaryentry{fictitious-capital}{
  name={fictitious capital},
  sort={fictitious capital},
  description={Tradable claims on future income streams (shares, bonds, securitised claims, many derivatives) whose market valuation can expand beyond surplus value currently produced, until crises force devaluation.}
}

\newglossaryentry{decommodification}{
  name={decommodification},
  sort={decommodification},
  description={Shifting access to essentials (housing, health, care, transport, energy, water) out of the market and away from ability to pay, toward rights-based provision.}
}

\newglossaryentry{capital-controls}{
  name={capital controls},
  sort={capital controls},
  description={Regulatory restrictions on cross-border movement of capital designed to limit capital flight, currency pressure, and the ability of owners to discipline reforms through financial exit.}
}

% ---------- Glossary terms (extended, targeted additions) ----------
\newglossaryentry{austerity}{
  name={austerity},
  sort={austerity},
  description={A policy package of spending cuts, hiring freezes, welfare retrenchment, regressive taxation, and/or user fees justified as “fiscal discipline”. In practice it often shifts crisis costs onto workers and the poor while protecting creditors and asset owners.}
}

\newglossaryentry{fiscal-consolidation}{
  name={fiscal consolidation},
  sort={fiscal consolidation},
  description={Reducing government deficits through spending cuts and/or tax rises. It is frequently presented as technocratic “budget repair,” but its class content depends on who is taxed, which services are cut, and whether interest payments to creditors are ring-fenced.}
}

\newglossaryentry{primary-balance}{
  name={primary balance},
  sort={primary balance},
  description={A government’s fiscal balance excluding interest payments on existing debt. A “primary surplus” can coexist with rising total debt burdens if interest costs remain high or growth is weak.}
}

\newglossaryentry{debt-service}{
  name={debt servicing},
  sort={debt servicing},
  description={Ongoing payments of interest and principal on debt. For many states, debt service becomes a prior claim on public revenue, structurally pressuring social spending and investment even without an explicit austerity programme.}
}

\newglossaryentry{conditionality}{
  name={conditionality},
  sort={conditionality},
  description={Policy conditions attached to loans or debt restructuring (commonly by the IMF or creditor blocs). Typical conditions include subsidy cuts, wage restraint, privatisation, deregulation, central bank “independence,” and fiscal consolidation.}
}

\newglossaryentry{structural-adjustment}{
  name={structural adjustment},
  sort={structural adjustment},
  description={A reform programme—historically associated with IMF/World Bank lending—that restructures economies toward export orientation, market pricing, privatisation, and reduced public provision. The “adjustment” is usually borne through depressed wages, weakened labour protections, and reduced social spending.}
}

\newglossaryentry{balance-of-payments}{
  name={balance of payments},
  sort={balance of payments},
  description={A country’s accounting of transactions with the rest of the world (trade in goods/services, income flows, and financial transfers). Persistent deficits often create pressure for devaluation, import compression, and external borrowing.}
}

\newglossaryentry{capital-flight}{
  name={capital flight},
  sort={capital flight},
  description={Rapid private movement of funds out of a country (or out of domestic investment into safer assets), often triggered by crisis, political conflict, or expectations of devaluation. Capital flight can force currency pressure, reserve loss, and harsher adjustment.}
}

\newglossaryentry{exchange-rate-pass-through}{
  name={exchange-rate pass-through},
  sort={exchange rate pass-through},
  description={The extent to which a currency devaluation raises domestic prices, especially for imported essentials (fuel, fertiliser, medicine, machinery). High pass-through can turn devaluation into immediate inflation and real-wage cuts.}
}

\newglossaryentry{inflation-targeting}{
  name={inflation targeting},
  sort={inflation targeting},
  description={A monetary-policy framework where the central bank prioritises hitting an inflation target, typically via interest-rate moves. Critics argue it can treat inflation as a purely monetary phenomenon while ignoring supply shocks, monopoly pricing, and import dependence.}
}

\newglossaryentry{protectionism}{
  name={protectionism},
  sort={protectionism},
  description={Using tariffs, quotas, licensing, local-content rules, or public procurement to shelter domestic producers from foreign competition. It can defend jobs and industrial capacity, but its effects depend on who controls protected firms, how prices/wages move, and whether technology/inputs are domestically available.}
}

\newglossaryentry{trade-liberalisation}{
  name={trade liberalisation},
  sort={trade liberalisation},
  description={Reducing tariffs, quotas, and other trade barriers. It is often sold as “efficiency,” but in unequal world markets it can accelerate deindustrialisation, worsen trade deficits, and deepen dependence on imported inputs and foreign currency.}
}

\newglossaryentry{import-substitution}{
  name={import-substitution industrialisation (ISI)},
  sort={import substitution industrialisation},
  description={A strategy to replace imports with domestic production through tariffs, credit allocation, industrial policy, and state procurement. ISI can build capacity, but it often hits constraints around technology, energy, foreign exchange, and class control of investment decisions.}
}

\newglossaryentry{qe-term}{
  name={Quantitative easing (QE)},
  sort={Quantitative easing},
  description={A central bank policy of purchasing government bonds and/or other financial assets to expand its balance sheet and push down longer-term interest rates. QE can stabilise financial markets, but it often inflates asset prices and does not automatically translate into productive investment or higher wages.}
}

\newglossaryentry{mmt-term}{
  name={Modern Monetary Theory (MMT)},
  sort={Modern Monetary Theory},
  description={A heterodox framework arguing that a state that issues its own currency cannot “run out” of money in the way households can; the binding constraints are real resources, productive capacity, and inflation dynamics. It stresses the role of taxation and bond issuance in managing demand, distribution, and monetary conditions rather than “funding” spending in a mechanical sense.}
}

\newglossaryentry{ubi-term}{
  name={Universal basic income (UBI)},
  sort={Universal basic income},
  description={An unconditional cash transfer to all residents or citizens. Proposals differ sharply: some are designed to replace welfare and subsidise low wages, while others are framed as an income floor that complements strong public services, labour rights, and decommodification.}
}

\newglossaryentry{ubs-term}{
  name={Universal basic services (UBS)},
  sort={Universal basic services},
  description={A model of guaranteeing key services—health, education, housing, transport, care, water/energy—through public provision or social rights rather than cash transfers. UBS centres decommodification and collective infrastructure, but requires fiscal capacity, democratic control, and organised labour to prevent deterioration or capture.}
}

\newglossaryentry{jg-term}{
  name={Job Guarantee (JG)},
  sort={Job Guarantee},
  description={A proposal that the state offers a public job at a socially defined wage to anyone willing to work. Advocates treat it as an employment floor and stabiliser; critics debate job quality, political control, and whether it can be insulated from austerity and patronage without strong democratic governance.}
}

\newglossaryentry{esg-term}{
  name={ESG},
  sort={ESG},
  description={A framework used by investors and firms to score “environmental, social, and governance” performance. ESG can pressure disclosure and some standards, but it is often criticised as compatible with continued extraction and financialisation, turning ecological crisis into a portfolio and reputational management problem.}
}

\newglossaryentry{troika}{
  name={the Troika},
  sort={Troika},
  description={A term commonly used for the European Commission (EC), the European Central Bank (ECB), and the IMF acting jointly in crisis programmes and conditional lending in the Eurozone.}
}

\newglossaryentry{eurozone}{
  name={Eurozone},
  sort={Eurozone},
  description={The group of EU member states using the euro. Eurozone membership removes independent monetary policy and exchange-rate adjustment, making fiscal policy and wage/price dynamics central sites of “internal devaluation” during crises.}
}

\newglossaryentry{syriza}{
  name={Syriza},
  sort={Syriza},
  description={A Greek left party (Coalition of the Radical Left) that came to power in 2015 on an anti-austerity mandate. Its confrontation with the Troika became a major reference point for debates on debt, monetary sovereignty, and the limits imposed by Eurozone institutions.}
}

% ----------------------------------------------------------------------
% 2) PRINTING MODE (when \input{glossary} is called in content.tex)
% ----------------------------------------------------------------------
\else

\section{Glossary and acronyms}
\label{sec:glossary}

\begingroup
\setlength{\parskip}{0pt}

% Print even if entries were not referenced yet:
\glsaddallunused

% Acronyms
\printnoidxglossary[type=\acronymtype,style=compactgls,title={Acronyms}]

\vspace{0.6\baselineskip}

% Terms
\printnoidxglossary[style=termscolon,title={Glossary of terms}]

\endgroup

\fi is used in content.tex.
%
%  IMPORTANT (preamble.tex):
%    \usepackage[acronym]{glossaries-extra}
%    \makenoidxglossaries
%    \def\GLOSSARYENTRIESONLY{}%
%    \loadglsentries{glossary}
%    \let\GLOSSARYENTRIESONLY\undefined
% ============================

\ProvidesFile{glossary.tex}[Glossary and acronym entries + Appendix printing]

% ----------------------------------------------------------------------
% 1) ENTRIES-ONLY MODE (when loaded in the preamble)
% ----------------------------------------------------------------------
\ifdefined\GLOSSARYENTRIESONLY

% ---------- Acronyms ----------
\newacronym{ai}{AI}{artificial intelligence}
\newacronym{ltv}{LTV}{labour theory of value}
\newacronym{snlt}{SNLT}{socially necessary labour time}
\newacronym{melt}{MELT}{monetary expression of labour time}
\newacronym{occ}{OCC}{organic composition of capital}
\newacronym{trpf}{TRPF}{tendency of the rate of profit to fall}

\newacronym{ubi}{UBI}{universal basic income}
\newacronym{ubs}{UBS}{universal basic services}
\newacronym{qe}{QE}{quantitative easing}
\newacronym{mmt}{MMT}{Modern Monetary Theory}
\newacronym{jg}{JG}{Job Guarantee}
\newacronym{esg}{ESG}{environmental, social, and governance}

\newacronym{imf}{IMF}{International Monetary Fund}
\newacronym{ipcc}{IPCC}{Intergovernmental Panel on Climate Change}

% ---------- Glossary terms (existing) ----------
\newglossaryentry{keynesianism}{
  name={Keynesianism},
  sort={Keynesianism},
  description={A macroeconomic approach associated with John Maynard Keynes that emphasises stabilising output and employment via demand management (especially fiscal policy and public spending), particularly during downturns.},
  first={Keynesianism (a macroeconomic approach that emphasises demand management—especially fiscal policy and public spending—to stabilise output and employment)}
}

\newglossaryentry{monetarism}{
  name={Monetarism},
  sort={Monetarism},
  description={A macroeconomic doctrine associated with Milton Friedman that prioritises controlling inflation by managing the money supply and/or interest rates, often via tight monetary policy.},
  first={Monetarism (a macroeconomic doctrine that prioritises inflation control via tight monetary policy—money supply and/or interest rates)}
}

\newglossaryentry{financialisation}{
  name={financialisation},
  sort={financialisation},
  description={A pattern in which profits, strategies, and power shift toward finance, asset price inflation, and rent extraction, rather than expanded productive investment and wage growth.}
}

\newglossaryentry{fictitious-capital}{
  name={fictitious capital},
  sort={fictitious capital},
  description={Tradable claims on future income streams (shares, bonds, securitised claims, many derivatives) whose market valuation can expand beyond surplus value currently produced, until crises force devaluation.}
}

\newglossaryentry{decommodification}{
  name={decommodification},
  sort={decommodification},
  description={Shifting access to essentials (housing, health, care, transport, energy, water) out of the market and away from ability to pay, toward rights-based provision.}
}

\newglossaryentry{capital-controls}{
  name={capital controls},
  sort={capital controls},
  description={Regulatory restrictions on cross-border movement of capital designed to limit capital flight, currency pressure, and the ability of owners to discipline reforms through financial exit.}
}

% ---------- Glossary terms (extended, targeted additions) ----------
\newglossaryentry{austerity}{
  name={austerity},
  sort={austerity},
  description={A policy package of spending cuts, hiring freezes, welfare retrenchment, regressive taxation, and/or user fees justified as “fiscal discipline”. In practice it often shifts crisis costs onto workers and the poor while protecting creditors and asset owners.}
}

\newglossaryentry{fiscal-consolidation}{
  name={fiscal consolidation},
  sort={fiscal consolidation},
  description={Reducing government deficits through spending cuts and/or tax rises. It is frequently presented as technocratic “budget repair,” but its class content depends on who is taxed, which services are cut, and whether interest payments to creditors are ring-fenced.}
}

\newglossaryentry{primary-balance}{
  name={primary balance},
  sort={primary balance},
  description={A government’s fiscal balance excluding interest payments on existing debt. A “primary surplus” can coexist with rising total debt burdens if interest costs remain high or growth is weak.}
}

\newglossaryentry{debt-service}{
  name={debt servicing},
  sort={debt servicing},
  description={Ongoing payments of interest and principal on debt. For many states, debt service becomes a prior claim on public revenue, structurally pressuring social spending and investment even without an explicit austerity programme.}
}

\newglossaryentry{conditionality}{
  name={conditionality},
  sort={conditionality},
  description={Policy conditions attached to loans or debt restructuring (commonly by the IMF or creditor blocs). Typical conditions include subsidy cuts, wage restraint, privatisation, deregulation, central bank “independence,” and fiscal consolidation.}
}

\newglossaryentry{structural-adjustment}{
  name={structural adjustment},
  sort={structural adjustment},
  description={A reform programme—historically associated with IMF/World Bank lending—that restructures economies toward export orientation, market pricing, privatisation, and reduced public provision. The “adjustment” is usually borne through depressed wages, weakened labour protections, and reduced social spending.}
}

\newglossaryentry{balance-of-payments}{
  name={balance of payments},
  sort={balance of payments},
  description={A country’s accounting of transactions with the rest of the world (trade in goods/services, income flows, and financial transfers). Persistent deficits often create pressure for devaluation, import compression, and external borrowing.}
}

\newglossaryentry{capital-flight}{
  name={capital flight},
  sort={capital flight},
  description={Rapid private movement of funds out of a country (or out of domestic investment into safer assets), often triggered by crisis, political conflict, or expectations of devaluation. Capital flight can force currency pressure, reserve loss, and harsher adjustment.}
}

\newglossaryentry{exchange-rate-pass-through}{
  name={exchange-rate pass-through},
  sort={exchange rate pass-through},
  description={The extent to which a currency devaluation raises domestic prices, especially for imported essentials (fuel, fertiliser, medicine, machinery). High pass-through can turn devaluation into immediate inflation and real-wage cuts.}
}

\newglossaryentry{inflation-targeting}{
  name={inflation targeting},
  sort={inflation targeting},
  description={A monetary-policy framework where the central bank prioritises hitting an inflation target, typically via interest-rate moves. Critics argue it can treat inflation as a purely monetary phenomenon while ignoring supply shocks, monopoly pricing, and import dependence.}
}

\newglossaryentry{protectionism}{
  name={protectionism},
  sort={protectionism},
  description={Using tariffs, quotas, licensing, local-content rules, or public procurement to shelter domestic producers from foreign competition. It can defend jobs and industrial capacity, but its effects depend on who controls protected firms, how prices/wages move, and whether technology/inputs are domestically available.}
}

\newglossaryentry{trade-liberalisation}{
  name={trade liberalisation},
  sort={trade liberalisation},
  description={Reducing tariffs, quotas, and other trade barriers. It is often sold as “efficiency,” but in unequal world markets it can accelerate deindustrialisation, worsen trade deficits, and deepen dependence on imported inputs and foreign currency.}
}

\newglossaryentry{import-substitution}{
  name={import-substitution industrialisation (ISI)},
  sort={import substitution industrialisation},
  description={A strategy to replace imports with domestic production through tariffs, credit allocation, industrial policy, and state procurement. ISI can build capacity, but it often hits constraints around technology, energy, foreign exchange, and class control of investment decisions.}
}

\newglossaryentry{qe-term}{
  name={Quantitative easing (QE)},
  sort={Quantitative easing},
  description={A central bank policy of purchasing government bonds and/or other financial assets to expand its balance sheet and push down longer-term interest rates. QE can stabilise financial markets, but it often inflates asset prices and does not automatically translate into productive investment or higher wages.}
}

\newglossaryentry{mmt-term}{
  name={Modern Monetary Theory (MMT)},
  sort={Modern Monetary Theory},
  description={A heterodox framework arguing that a state that issues its own currency cannot “run out” of money in the way households can; the binding constraints are real resources, productive capacity, and inflation dynamics. It stresses the role of taxation and bond issuance in managing demand, distribution, and monetary conditions rather than “funding” spending in a mechanical sense.}
}

\newglossaryentry{ubi-term}{
  name={Universal basic income (UBI)},
  sort={Universal basic income},
  description={An unconditional cash transfer to all residents or citizens. Proposals differ sharply: some are designed to replace welfare and subsidise low wages, while others are framed as an income floor that complements strong public services, labour rights, and decommodification.}
}

\newglossaryentry{ubs-term}{
  name={Universal basic services (UBS)},
  sort={Universal basic services},
  description={A model of guaranteeing key services—health, education, housing, transport, care, water/energy—through public provision or social rights rather than cash transfers. UBS centres decommodification and collective infrastructure, but requires fiscal capacity, democratic control, and organised labour to prevent deterioration or capture.}
}

\newglossaryentry{jg-term}{
  name={Job Guarantee (JG)},
  sort={Job Guarantee},
  description={A proposal that the state offers a public job at a socially defined wage to anyone willing to work. Advocates treat it as an employment floor and stabiliser; critics debate job quality, political control, and whether it can be insulated from austerity and patronage without strong democratic governance.}
}

\newglossaryentry{esg-term}{
  name={ESG},
  sort={ESG},
  description={A framework used by investors and firms to score “environmental, social, and governance” performance. ESG can pressure disclosure and some standards, but it is often criticised as compatible with continued extraction and financialisation, turning ecological crisis into a portfolio and reputational management problem.}
}

\newglossaryentry{troika}{
  name={the Troika},
  sort={Troika},
  description={A term commonly used for the European Commission (EC), the European Central Bank (ECB), and the IMF acting jointly in crisis programmes and conditional lending in the Eurozone.}
}

\newglossaryentry{eurozone}{
  name={Eurozone},
  sort={Eurozone},
  description={The group of EU member states using the euro. Eurozone membership removes independent monetary policy and exchange-rate adjustment, making fiscal policy and wage/price dynamics central sites of “internal devaluation” during crises.}
}

\newglossaryentry{syriza}{
  name={Syriza},
  sort={Syriza},
  description={A Greek left party (Coalition of the Radical Left) that came to power in 2015 on an anti-austerity mandate. Its confrontation with the Troika became a major reference point for debates on debt, monetary sovereignty, and the limits imposed by Eurozone institutions.}
}

% ----------------------------------------------------------------------
% 2) PRINTING MODE (when % ============================
%  glossary.tex
%  Acronyms + glossary entries (glossaries-extra, no-index workflow)
%  ALSO prints Appendix B when \input{glossary} is used in content.tex.
%
%  IMPORTANT (preamble.tex):
%    \usepackage[acronym]{glossaries-extra}
%    \makenoidxglossaries
%    \def\GLOSSARYENTRIESONLY{}%
%    \loadglsentries{glossary}
%    \let\GLOSSARYENTRIESONLY\undefined
% ============================

\ProvidesFile{glossary.tex}[Glossary and acronym entries + Appendix printing]

% ----------------------------------------------------------------------
% 1) ENTRIES-ONLY MODE (when loaded in the preamble)
% ----------------------------------------------------------------------
\ifdefined\GLOSSARYENTRIESONLY

% ---------- Acronyms ----------
\newacronym{ai}{AI}{artificial intelligence}
\newacronym{ltv}{LTV}{labour theory of value}
\newacronym{snlt}{SNLT}{socially necessary labour time}
\newacronym{melt}{MELT}{monetary expression of labour time}
\newacronym{occ}{OCC}{organic composition of capital}
\newacronym{trpf}{TRPF}{tendency of the rate of profit to fall}

\newacronym{ubi}{UBI}{universal basic income}
\newacronym{ubs}{UBS}{universal basic services}
\newacronym{qe}{QE}{quantitative easing}
\newacronym{mmt}{MMT}{Modern Monetary Theory}
\newacronym{jg}{JG}{Job Guarantee}
\newacronym{esg}{ESG}{environmental, social, and governance}

\newacronym{imf}{IMF}{International Monetary Fund}
\newacronym{ipcc}{IPCC}{Intergovernmental Panel on Climate Change}

% ---------- Glossary terms (existing) ----------
\newglossaryentry{keynesianism}{
  name={Keynesianism},
  sort={Keynesianism},
  description={A macroeconomic approach associated with John Maynard Keynes that emphasises stabilising output and employment via demand management (especially fiscal policy and public spending), particularly during downturns.},
  first={Keynesianism (a macroeconomic approach that emphasises demand management—especially fiscal policy and public spending—to stabilise output and employment)}
}

\newglossaryentry{monetarism}{
  name={Monetarism},
  sort={Monetarism},
  description={A macroeconomic doctrine associated with Milton Friedman that prioritises controlling inflation by managing the money supply and/or interest rates, often via tight monetary policy.},
  first={Monetarism (a macroeconomic doctrine that prioritises inflation control via tight monetary policy—money supply and/or interest rates)}
}

\newglossaryentry{financialisation}{
  name={financialisation},
  sort={financialisation},
  description={A pattern in which profits, strategies, and power shift toward finance, asset price inflation, and rent extraction, rather than expanded productive investment and wage growth.}
}

\newglossaryentry{fictitious-capital}{
  name={fictitious capital},
  sort={fictitious capital},
  description={Tradable claims on future income streams (shares, bonds, securitised claims, many derivatives) whose market valuation can expand beyond surplus value currently produced, until crises force devaluation.}
}

\newglossaryentry{decommodification}{
  name={decommodification},
  sort={decommodification},
  description={Shifting access to essentials (housing, health, care, transport, energy, water) out of the market and away from ability to pay, toward rights-based provision.}
}

\newglossaryentry{capital-controls}{
  name={capital controls},
  sort={capital controls},
  description={Regulatory restrictions on cross-border movement of capital designed to limit capital flight, currency pressure, and the ability of owners to discipline reforms through financial exit.}
}

% ---------- Glossary terms (extended, targeted additions) ----------
\newglossaryentry{austerity}{
  name={austerity},
  sort={austerity},
  description={A policy package of spending cuts, hiring freezes, welfare retrenchment, regressive taxation, and/or user fees justified as “fiscal discipline”. In practice it often shifts crisis costs onto workers and the poor while protecting creditors and asset owners.}
}

\newglossaryentry{fiscal-consolidation}{
  name={fiscal consolidation},
  sort={fiscal consolidation},
  description={Reducing government deficits through spending cuts and/or tax rises. It is frequently presented as technocratic “budget repair,” but its class content depends on who is taxed, which services are cut, and whether interest payments to creditors are ring-fenced.}
}

\newglossaryentry{primary-balance}{
  name={primary balance},
  sort={primary balance},
  description={A government’s fiscal balance excluding interest payments on existing debt. A “primary surplus” can coexist with rising total debt burdens if interest costs remain high or growth is weak.}
}

\newglossaryentry{debt-service}{
  name={debt servicing},
  sort={debt servicing},
  description={Ongoing payments of interest and principal on debt. For many states, debt service becomes a prior claim on public revenue, structurally pressuring social spending and investment even without an explicit austerity programme.}
}

\newglossaryentry{conditionality}{
  name={conditionality},
  sort={conditionality},
  description={Policy conditions attached to loans or debt restructuring (commonly by the IMF or creditor blocs). Typical conditions include subsidy cuts, wage restraint, privatisation, deregulation, central bank “independence,” and fiscal consolidation.}
}

\newglossaryentry{structural-adjustment}{
  name={structural adjustment},
  sort={structural adjustment},
  description={A reform programme—historically associated with IMF/World Bank lending—that restructures economies toward export orientation, market pricing, privatisation, and reduced public provision. The “adjustment” is usually borne through depressed wages, weakened labour protections, and reduced social spending.}
}

\newglossaryentry{balance-of-payments}{
  name={balance of payments},
  sort={balance of payments},
  description={A country’s accounting of transactions with the rest of the world (trade in goods/services, income flows, and financial transfers). Persistent deficits often create pressure for devaluation, import compression, and external borrowing.}
}

\newglossaryentry{capital-flight}{
  name={capital flight},
  sort={capital flight},
  description={Rapid private movement of funds out of a country (or out of domestic investment into safer assets), often triggered by crisis, political conflict, or expectations of devaluation. Capital flight can force currency pressure, reserve loss, and harsher adjustment.}
}

\newglossaryentry{exchange-rate-pass-through}{
  name={exchange-rate pass-through},
  sort={exchange rate pass-through},
  description={The extent to which a currency devaluation raises domestic prices, especially for imported essentials (fuel, fertiliser, medicine, machinery). High pass-through can turn devaluation into immediate inflation and real-wage cuts.}
}

\newglossaryentry{inflation-targeting}{
  name={inflation targeting},
  sort={inflation targeting},
  description={A monetary-policy framework where the central bank prioritises hitting an inflation target, typically via interest-rate moves. Critics argue it can treat inflation as a purely monetary phenomenon while ignoring supply shocks, monopoly pricing, and import dependence.}
}

\newglossaryentry{protectionism}{
  name={protectionism},
  sort={protectionism},
  description={Using tariffs, quotas, licensing, local-content rules, or public procurement to shelter domestic producers from foreign competition. It can defend jobs and industrial capacity, but its effects depend on who controls protected firms, how prices/wages move, and whether technology/inputs are domestically available.}
}

\newglossaryentry{trade-liberalisation}{
  name={trade liberalisation},
  sort={trade liberalisation},
  description={Reducing tariffs, quotas, and other trade barriers. It is often sold as “efficiency,” but in unequal world markets it can accelerate deindustrialisation, worsen trade deficits, and deepen dependence on imported inputs and foreign currency.}
}

\newglossaryentry{import-substitution}{
  name={import-substitution industrialisation (ISI)},
  sort={import substitution industrialisation},
  description={A strategy to replace imports with domestic production through tariffs, credit allocation, industrial policy, and state procurement. ISI can build capacity, but it often hits constraints around technology, energy, foreign exchange, and class control of investment decisions.}
}

\newglossaryentry{qe-term}{
  name={Quantitative easing (QE)},
  sort={Quantitative easing},
  description={A central bank policy of purchasing government bonds and/or other financial assets to expand its balance sheet and push down longer-term interest rates. QE can stabilise financial markets, but it often inflates asset prices and does not automatically translate into productive investment or higher wages.}
}

\newglossaryentry{mmt-term}{
  name={Modern Monetary Theory (MMT)},
  sort={Modern Monetary Theory},
  description={A heterodox framework arguing that a state that issues its own currency cannot “run out” of money in the way households can; the binding constraints are real resources, productive capacity, and inflation dynamics. It stresses the role of taxation and bond issuance in managing demand, distribution, and monetary conditions rather than “funding” spending in a mechanical sense.}
}

\newglossaryentry{ubi-term}{
  name={Universal basic income (UBI)},
  sort={Universal basic income},
  description={An unconditional cash transfer to all residents or citizens. Proposals differ sharply: some are designed to replace welfare and subsidise low wages, while others are framed as an income floor that complements strong public services, labour rights, and decommodification.}
}

\newglossaryentry{ubs-term}{
  name={Universal basic services (UBS)},
  sort={Universal basic services},
  description={A model of guaranteeing key services—health, education, housing, transport, care, water/energy—through public provision or social rights rather than cash transfers. UBS centres decommodification and collective infrastructure, but requires fiscal capacity, democratic control, and organised labour to prevent deterioration or capture.}
}

\newglossaryentry{jg-term}{
  name={Job Guarantee (JG)},
  sort={Job Guarantee},
  description={A proposal that the state offers a public job at a socially defined wage to anyone willing to work. Advocates treat it as an employment floor and stabiliser; critics debate job quality, political control, and whether it can be insulated from austerity and patronage without strong democratic governance.}
}

\newglossaryentry{esg-term}{
  name={ESG},
  sort={ESG},
  description={A framework used by investors and firms to score “environmental, social, and governance” performance. ESG can pressure disclosure and some standards, but it is often criticised as compatible with continued extraction and financialisation, turning ecological crisis into a portfolio and reputational management problem.}
}

\newglossaryentry{troika}{
  name={the Troika},
  sort={Troika},
  description={A term commonly used for the European Commission (EC), the European Central Bank (ECB), and the IMF acting jointly in crisis programmes and conditional lending in the Eurozone.}
}

\newglossaryentry{eurozone}{
  name={Eurozone},
  sort={Eurozone},
  description={The group of EU member states using the euro. Eurozone membership removes independent monetary policy and exchange-rate adjustment, making fiscal policy and wage/price dynamics central sites of “internal devaluation” during crises.}
}

\newglossaryentry{syriza}{
  name={Syriza},
  sort={Syriza},
  description={A Greek left party (Coalition of the Radical Left) that came to power in 2015 on an anti-austerity mandate. Its confrontation with the Troika became a major reference point for debates on debt, monetary sovereignty, and the limits imposed by Eurozone institutions.}
}

% ----------------------------------------------------------------------
% 2) PRINTING MODE (when \input{glossary} is called in content.tex)
% ----------------------------------------------------------------------
\else

\section{Glossary and acronyms}
\label{sec:glossary}

\begingroup
\setlength{\parskip}{0pt}

% Print even if entries were not referenced yet:
\glsaddallunused

% Acronyms
\printnoidxglossary[type=\acronymtype,style=compactgls,title={Acronyms}]

\vspace{0.6\baselineskip}

% Terms
\printnoidxglossary[style=termscolon,title={Glossary of terms}]

\endgroup

\fi is called in content.tex)
% ----------------------------------------------------------------------
\else

\section{Glossary and acronyms}
\label{sec:glossary}

\begingroup
\setlength{\parskip}{0pt}

% Print even if entries were not referenced yet:
\glsaddallunused

% Acronyms
\printnoidxglossary[type=\acronymtype,style=compactgls,title={Acronyms}]

\vspace{0.6\baselineskip}

% Terms
\printnoidxglossary[style=termscolon,title={Glossary of terms}]

\endgroup

\fi is used in content.tex.
%
%  IMPORTANT (preamble.tex):
%    \usepackage[acronym]{glossaries-extra}
%    \makenoidxglossaries
%    \def\GLOSSARYENTRIESONLY{}%
%    \loadglsentries{glossary}
%    \let\GLOSSARYENTRIESONLY\undefined
% ============================

\ProvidesFile{glossary.tex}[Glossary and acronym entries + Appendix printing]

% ----------------------------------------------------------------------
% 1) ENTRIES-ONLY MODE (when loaded in the preamble)
% ----------------------------------------------------------------------
\ifdefined\GLOSSARYENTRIESONLY

% ---------- Acronyms ----------
\newacronym{ai}{AI}{artificial intelligence}
\newacronym{ltv}{LTV}{labour theory of value}
\newacronym{snlt}{SNLT}{socially necessary labour time}
\newacronym{melt}{MELT}{monetary expression of labour time}
\newacronym{occ}{OCC}{organic composition of capital}
\newacronym{trpf}{TRPF}{tendency of the rate of profit to fall}

\newacronym{ubi}{UBI}{universal basic income}
\newacronym{ubs}{UBS}{universal basic services}
\newacronym{qe}{QE}{quantitative easing}
\newacronym{mmt}{MMT}{Modern Monetary Theory}
\newacronym{jg}{JG}{Job Guarantee}
\newacronym{esg}{ESG}{environmental, social, and governance}

\newacronym{imf}{IMF}{International Monetary Fund}
\newacronym{ipcc}{IPCC}{Intergovernmental Panel on Climate Change}

% ---------- Glossary terms (existing) ----------
\newglossaryentry{keynesianism}{
  name={Keynesianism},
  sort={Keynesianism},
  description={A macroeconomic approach associated with John Maynard Keynes that emphasises stabilising output and employment via demand management (especially fiscal policy and public spending), particularly during downturns.},
  first={Keynesianism (a macroeconomic approach that emphasises demand management—especially fiscal policy and public spending—to stabilise output and employment)}
}

\newglossaryentry{monetarism}{
  name={Monetarism},
  sort={Monetarism},
  description={A macroeconomic doctrine associated with Milton Friedman that prioritises controlling inflation by managing the money supply and/or interest rates, often via tight monetary policy.},
  first={Monetarism (a macroeconomic doctrine that prioritises inflation control via tight monetary policy—money supply and/or interest rates)}
}

\newglossaryentry{financialisation}{
  name={financialisation},
  sort={financialisation},
  description={A pattern in which profits, strategies, and power shift toward finance, asset price inflation, and rent extraction, rather than expanded productive investment and wage growth.}
}

\newglossaryentry{fictitious-capital}{
  name={fictitious capital},
  sort={fictitious capital},
  description={Tradable claims on future income streams (shares, bonds, securitised claims, many derivatives) whose market valuation can expand beyond surplus value currently produced, until crises force devaluation.}
}

\newglossaryentry{decommodification}{
  name={decommodification},
  sort={decommodification},
  description={Shifting access to essentials (housing, health, care, transport, energy, water) out of the market and away from ability to pay, toward rights-based provision.}
}

\newglossaryentry{capital-controls}{
  name={capital controls},
  sort={capital controls},
  description={Regulatory restrictions on cross-border movement of capital designed to limit capital flight, currency pressure, and the ability of owners to discipline reforms through financial exit.}
}

% ---------- Glossary terms (extended, targeted additions) ----------
\newglossaryentry{austerity}{
  name={austerity},
  sort={austerity},
  description={A policy package of spending cuts, hiring freezes, welfare retrenchment, regressive taxation, and/or user fees justified as “fiscal discipline”. In practice it often shifts crisis costs onto workers and the poor while protecting creditors and asset owners.}
}

\newglossaryentry{fiscal-consolidation}{
  name={fiscal consolidation},
  sort={fiscal consolidation},
  description={Reducing government deficits through spending cuts and/or tax rises. It is frequently presented as technocratic “budget repair,” but its class content depends on who is taxed, which services are cut, and whether interest payments to creditors are ring-fenced.}
}

\newglossaryentry{primary-balance}{
  name={primary balance},
  sort={primary balance},
  description={A government’s fiscal balance excluding interest payments on existing debt. A “primary surplus” can coexist with rising total debt burdens if interest costs remain high or growth is weak.}
}

\newglossaryentry{debt-service}{
  name={debt servicing},
  sort={debt servicing},
  description={Ongoing payments of interest and principal on debt. For many states, debt service becomes a prior claim on public revenue, structurally pressuring social spending and investment even without an explicit austerity programme.}
}

\newglossaryentry{conditionality}{
  name={conditionality},
  sort={conditionality},
  description={Policy conditions attached to loans or debt restructuring (commonly by the IMF or creditor blocs). Typical conditions include subsidy cuts, wage restraint, privatisation, deregulation, central bank “independence,” and fiscal consolidation.}
}

\newglossaryentry{structural-adjustment}{
  name={structural adjustment},
  sort={structural adjustment},
  description={A reform programme—historically associated with IMF/World Bank lending—that restructures economies toward export orientation, market pricing, privatisation, and reduced public provision. The “adjustment” is usually borne through depressed wages, weakened labour protections, and reduced social spending.}
}

\newglossaryentry{balance-of-payments}{
  name={balance of payments},
  sort={balance of payments},
  description={A country’s accounting of transactions with the rest of the world (trade in goods/services, income flows, and financial transfers). Persistent deficits often create pressure for devaluation, import compression, and external borrowing.}
}

\newglossaryentry{capital-flight}{
  name={capital flight},
  sort={capital flight},
  description={Rapid private movement of funds out of a country (or out of domestic investment into safer assets), often triggered by crisis, political conflict, or expectations of devaluation. Capital flight can force currency pressure, reserve loss, and harsher adjustment.}
}

\newglossaryentry{exchange-rate-pass-through}{
  name={exchange-rate pass-through},
  sort={exchange rate pass-through},
  description={The extent to which a currency devaluation raises domestic prices, especially for imported essentials (fuel, fertiliser, medicine, machinery). High pass-through can turn devaluation into immediate inflation and real-wage cuts.}
}

\newglossaryentry{inflation-targeting}{
  name={inflation targeting},
  sort={inflation targeting},
  description={A monetary-policy framework where the central bank prioritises hitting an inflation target, typically via interest-rate moves. Critics argue it can treat inflation as a purely monetary phenomenon while ignoring supply shocks, monopoly pricing, and import dependence.}
}

\newglossaryentry{protectionism}{
  name={protectionism},
  sort={protectionism},
  description={Using tariffs, quotas, licensing, local-content rules, or public procurement to shelter domestic producers from foreign competition. It can defend jobs and industrial capacity, but its effects depend on who controls protected firms, how prices/wages move, and whether technology/inputs are domestically available.}
}

\newglossaryentry{trade-liberalisation}{
  name={trade liberalisation},
  sort={trade liberalisation},
  description={Reducing tariffs, quotas, and other trade barriers. It is often sold as “efficiency,” but in unequal world markets it can accelerate deindustrialisation, worsen trade deficits, and deepen dependence on imported inputs and foreign currency.}
}

\newglossaryentry{import-substitution}{
  name={import-substitution industrialisation (ISI)},
  sort={import substitution industrialisation},
  description={A strategy to replace imports with domestic production through tariffs, credit allocation, industrial policy, and state procurement. ISI can build capacity, but it often hits constraints around technology, energy, foreign exchange, and class control of investment decisions.}
}

\newglossaryentry{qe-term}{
  name={Quantitative easing (QE)},
  sort={Quantitative easing},
  description={A central bank policy of purchasing government bonds and/or other financial assets to expand its balance sheet and push down longer-term interest rates. QE can stabilise financial markets, but it often inflates asset prices and does not automatically translate into productive investment or higher wages.}
}

\newglossaryentry{mmt-term}{
  name={Modern Monetary Theory (MMT)},
  sort={Modern Monetary Theory},
  description={A heterodox framework arguing that a state that issues its own currency cannot “run out” of money in the way households can; the binding constraints are real resources, productive capacity, and inflation dynamics. It stresses the role of taxation and bond issuance in managing demand, distribution, and monetary conditions rather than “funding” spending in a mechanical sense.}
}

\newglossaryentry{ubi-term}{
  name={Universal basic income (UBI)},
  sort={Universal basic income},
  description={An unconditional cash transfer to all residents or citizens. Proposals differ sharply: some are designed to replace welfare and subsidise low wages, while others are framed as an income floor that complements strong public services, labour rights, and decommodification.}
}

\newglossaryentry{ubs-term}{
  name={Universal basic services (UBS)},
  sort={Universal basic services},
  description={A model of guaranteeing key services—health, education, housing, transport, care, water/energy—through public provision or social rights rather than cash transfers. UBS centres decommodification and collective infrastructure, but requires fiscal capacity, democratic control, and organised labour to prevent deterioration or capture.}
}

\newglossaryentry{jg-term}{
  name={Job Guarantee (JG)},
  sort={Job Guarantee},
  description={A proposal that the state offers a public job at a socially defined wage to anyone willing to work. Advocates treat it as an employment floor and stabiliser; critics debate job quality, political control, and whether it can be insulated from austerity and patronage without strong democratic governance.}
}

\newglossaryentry{esg-term}{
  name={ESG},
  sort={ESG},
  description={A framework used by investors and firms to score “environmental, social, and governance” performance. ESG can pressure disclosure and some standards, but it is often criticised as compatible with continued extraction and financialisation, turning ecological crisis into a portfolio and reputational management problem.}
}

\newglossaryentry{troika}{
  name={the Troika},
  sort={Troika},
  description={A term commonly used for the European Commission (EC), the European Central Bank (ECB), and the IMF acting jointly in crisis programmes and conditional lending in the Eurozone.}
}

\newglossaryentry{eurozone}{
  name={Eurozone},
  sort={Eurozone},
  description={The group of EU member states using the euro. Eurozone membership removes independent monetary policy and exchange-rate adjustment, making fiscal policy and wage/price dynamics central sites of “internal devaluation” during crises.}
}

\newglossaryentry{syriza}{
  name={Syriza},
  sort={Syriza},
  description={A Greek left party (Coalition of the Radical Left) that came to power in 2015 on an anti-austerity mandate. Its confrontation with the Troika became a major reference point for debates on debt, monetary sovereignty, and the limits imposed by Eurozone institutions.}
}

% ----------------------------------------------------------------------
% 2) PRINTING MODE (when % ============================
%  glossary.tex
%  Acronyms + glossary entries (glossaries-extra, no-index workflow)
%  ALSO prints Appendix B when % ============================
%  glossary.tex
%  Acronyms + glossary entries (glossaries-extra, no-index workflow)
%  ALSO prints Appendix B when \input{glossary} is used in content.tex.
%
%  IMPORTANT (preamble.tex):
%    \usepackage[acronym]{glossaries-extra}
%    \makenoidxglossaries
%    \def\GLOSSARYENTRIESONLY{}%
%    \loadglsentries{glossary}
%    \let\GLOSSARYENTRIESONLY\undefined
% ============================

\ProvidesFile{glossary.tex}[Glossary and acronym entries + Appendix printing]

% ----------------------------------------------------------------------
% 1) ENTRIES-ONLY MODE (when loaded in the preamble)
% ----------------------------------------------------------------------
\ifdefined\GLOSSARYENTRIESONLY

% ---------- Acronyms ----------
\newacronym{ai}{AI}{artificial intelligence}
\newacronym{ltv}{LTV}{labour theory of value}
\newacronym{snlt}{SNLT}{socially necessary labour time}
\newacronym{melt}{MELT}{monetary expression of labour time}
\newacronym{occ}{OCC}{organic composition of capital}
\newacronym{trpf}{TRPF}{tendency of the rate of profit to fall}

\newacronym{ubi}{UBI}{universal basic income}
\newacronym{ubs}{UBS}{universal basic services}
\newacronym{qe}{QE}{quantitative easing}
\newacronym{mmt}{MMT}{Modern Monetary Theory}
\newacronym{jg}{JG}{Job Guarantee}
\newacronym{esg}{ESG}{environmental, social, and governance}

\newacronym{imf}{IMF}{International Monetary Fund}
\newacronym{ipcc}{IPCC}{Intergovernmental Panel on Climate Change}

% ---------- Glossary terms (existing) ----------
\newglossaryentry{keynesianism}{
  name={Keynesianism},
  sort={Keynesianism},
  description={A macroeconomic approach associated with John Maynard Keynes that emphasises stabilising output and employment via demand management (especially fiscal policy and public spending), particularly during downturns.},
  first={Keynesianism (a macroeconomic approach that emphasises demand management—especially fiscal policy and public spending—to stabilise output and employment)}
}

\newglossaryentry{monetarism}{
  name={Monetarism},
  sort={Monetarism},
  description={A macroeconomic doctrine associated with Milton Friedman that prioritises controlling inflation by managing the money supply and/or interest rates, often via tight monetary policy.},
  first={Monetarism (a macroeconomic doctrine that prioritises inflation control via tight monetary policy—money supply and/or interest rates)}
}

\newglossaryentry{financialisation}{
  name={financialisation},
  sort={financialisation},
  description={A pattern in which profits, strategies, and power shift toward finance, asset price inflation, and rent extraction, rather than expanded productive investment and wage growth.}
}

\newglossaryentry{fictitious-capital}{
  name={fictitious capital},
  sort={fictitious capital},
  description={Tradable claims on future income streams (shares, bonds, securitised claims, many derivatives) whose market valuation can expand beyond surplus value currently produced, until crises force devaluation.}
}

\newglossaryentry{decommodification}{
  name={decommodification},
  sort={decommodification},
  description={Shifting access to essentials (housing, health, care, transport, energy, water) out of the market and away from ability to pay, toward rights-based provision.}
}

\newglossaryentry{capital-controls}{
  name={capital controls},
  sort={capital controls},
  description={Regulatory restrictions on cross-border movement of capital designed to limit capital flight, currency pressure, and the ability of owners to discipline reforms through financial exit.}
}

% ---------- Glossary terms (extended, targeted additions) ----------
\newglossaryentry{austerity}{
  name={austerity},
  sort={austerity},
  description={A policy package of spending cuts, hiring freezes, welfare retrenchment, regressive taxation, and/or user fees justified as “fiscal discipline”. In practice it often shifts crisis costs onto workers and the poor while protecting creditors and asset owners.}
}

\newglossaryentry{fiscal-consolidation}{
  name={fiscal consolidation},
  sort={fiscal consolidation},
  description={Reducing government deficits through spending cuts and/or tax rises. It is frequently presented as technocratic “budget repair,” but its class content depends on who is taxed, which services are cut, and whether interest payments to creditors are ring-fenced.}
}

\newglossaryentry{primary-balance}{
  name={primary balance},
  sort={primary balance},
  description={A government’s fiscal balance excluding interest payments on existing debt. A “primary surplus” can coexist with rising total debt burdens if interest costs remain high or growth is weak.}
}

\newglossaryentry{debt-service}{
  name={debt servicing},
  sort={debt servicing},
  description={Ongoing payments of interest and principal on debt. For many states, debt service becomes a prior claim on public revenue, structurally pressuring social spending and investment even without an explicit austerity programme.}
}

\newglossaryentry{conditionality}{
  name={conditionality},
  sort={conditionality},
  description={Policy conditions attached to loans or debt restructuring (commonly by the IMF or creditor blocs). Typical conditions include subsidy cuts, wage restraint, privatisation, deregulation, central bank “independence,” and fiscal consolidation.}
}

\newglossaryentry{structural-adjustment}{
  name={structural adjustment},
  sort={structural adjustment},
  description={A reform programme—historically associated with IMF/World Bank lending—that restructures economies toward export orientation, market pricing, privatisation, and reduced public provision. The “adjustment” is usually borne through depressed wages, weakened labour protections, and reduced social spending.}
}

\newglossaryentry{balance-of-payments}{
  name={balance of payments},
  sort={balance of payments},
  description={A country’s accounting of transactions with the rest of the world (trade in goods/services, income flows, and financial transfers). Persistent deficits often create pressure for devaluation, import compression, and external borrowing.}
}

\newglossaryentry{capital-flight}{
  name={capital flight},
  sort={capital flight},
  description={Rapid private movement of funds out of a country (or out of domestic investment into safer assets), often triggered by crisis, political conflict, or expectations of devaluation. Capital flight can force currency pressure, reserve loss, and harsher adjustment.}
}

\newglossaryentry{exchange-rate-pass-through}{
  name={exchange-rate pass-through},
  sort={exchange rate pass-through},
  description={The extent to which a currency devaluation raises domestic prices, especially for imported essentials (fuel, fertiliser, medicine, machinery). High pass-through can turn devaluation into immediate inflation and real-wage cuts.}
}

\newglossaryentry{inflation-targeting}{
  name={inflation targeting},
  sort={inflation targeting},
  description={A monetary-policy framework where the central bank prioritises hitting an inflation target, typically via interest-rate moves. Critics argue it can treat inflation as a purely monetary phenomenon while ignoring supply shocks, monopoly pricing, and import dependence.}
}

\newglossaryentry{protectionism}{
  name={protectionism},
  sort={protectionism},
  description={Using tariffs, quotas, licensing, local-content rules, or public procurement to shelter domestic producers from foreign competition. It can defend jobs and industrial capacity, but its effects depend on who controls protected firms, how prices/wages move, and whether technology/inputs are domestically available.}
}

\newglossaryentry{trade-liberalisation}{
  name={trade liberalisation},
  sort={trade liberalisation},
  description={Reducing tariffs, quotas, and other trade barriers. It is often sold as “efficiency,” but in unequal world markets it can accelerate deindustrialisation, worsen trade deficits, and deepen dependence on imported inputs and foreign currency.}
}

\newglossaryentry{import-substitution}{
  name={import-substitution industrialisation (ISI)},
  sort={import substitution industrialisation},
  description={A strategy to replace imports with domestic production through tariffs, credit allocation, industrial policy, and state procurement. ISI can build capacity, but it often hits constraints around technology, energy, foreign exchange, and class control of investment decisions.}
}

\newglossaryentry{qe-term}{
  name={Quantitative easing (QE)},
  sort={Quantitative easing},
  description={A central bank policy of purchasing government bonds and/or other financial assets to expand its balance sheet and push down longer-term interest rates. QE can stabilise financial markets, but it often inflates asset prices and does not automatically translate into productive investment or higher wages.}
}

\newglossaryentry{mmt-term}{
  name={Modern Monetary Theory (MMT)},
  sort={Modern Monetary Theory},
  description={A heterodox framework arguing that a state that issues its own currency cannot “run out” of money in the way households can; the binding constraints are real resources, productive capacity, and inflation dynamics. It stresses the role of taxation and bond issuance in managing demand, distribution, and monetary conditions rather than “funding” spending in a mechanical sense.}
}

\newglossaryentry{ubi-term}{
  name={Universal basic income (UBI)},
  sort={Universal basic income},
  description={An unconditional cash transfer to all residents or citizens. Proposals differ sharply: some are designed to replace welfare and subsidise low wages, while others are framed as an income floor that complements strong public services, labour rights, and decommodification.}
}

\newglossaryentry{ubs-term}{
  name={Universal basic services (UBS)},
  sort={Universal basic services},
  description={A model of guaranteeing key services—health, education, housing, transport, care, water/energy—through public provision or social rights rather than cash transfers. UBS centres decommodification and collective infrastructure, but requires fiscal capacity, democratic control, and organised labour to prevent deterioration or capture.}
}

\newglossaryentry{jg-term}{
  name={Job Guarantee (JG)},
  sort={Job Guarantee},
  description={A proposal that the state offers a public job at a socially defined wage to anyone willing to work. Advocates treat it as an employment floor and stabiliser; critics debate job quality, political control, and whether it can be insulated from austerity and patronage without strong democratic governance.}
}

\newglossaryentry{esg-term}{
  name={ESG},
  sort={ESG},
  description={A framework used by investors and firms to score “environmental, social, and governance” performance. ESG can pressure disclosure and some standards, but it is often criticised as compatible with continued extraction and financialisation, turning ecological crisis into a portfolio and reputational management problem.}
}

\newglossaryentry{troika}{
  name={the Troika},
  sort={Troika},
  description={A term commonly used for the European Commission (EC), the European Central Bank (ECB), and the IMF acting jointly in crisis programmes and conditional lending in the Eurozone.}
}

\newglossaryentry{eurozone}{
  name={Eurozone},
  sort={Eurozone},
  description={The group of EU member states using the euro. Eurozone membership removes independent monetary policy and exchange-rate adjustment, making fiscal policy and wage/price dynamics central sites of “internal devaluation” during crises.}
}

\newglossaryentry{syriza}{
  name={Syriza},
  sort={Syriza},
  description={A Greek left party (Coalition of the Radical Left) that came to power in 2015 on an anti-austerity mandate. Its confrontation with the Troika became a major reference point for debates on debt, monetary sovereignty, and the limits imposed by Eurozone institutions.}
}

% ----------------------------------------------------------------------
% 2) PRINTING MODE (when \input{glossary} is called in content.tex)
% ----------------------------------------------------------------------
\else

\section{Glossary and acronyms}
\label{sec:glossary}

\begingroup
\setlength{\parskip}{0pt}

% Print even if entries were not referenced yet:
\glsaddallunused

% Acronyms
\printnoidxglossary[type=\acronymtype,style=compactgls,title={Acronyms}]

\vspace{0.6\baselineskip}

% Terms
\printnoidxglossary[style=termscolon,title={Glossary of terms}]

\endgroup

\fi is used in content.tex.
%
%  IMPORTANT (preamble.tex):
%    \usepackage[acronym]{glossaries-extra}
%    \makenoidxglossaries
%    \def\GLOSSARYENTRIESONLY{}%
%    \loadglsentries{glossary}
%    \let\GLOSSARYENTRIESONLY\undefined
% ============================

\ProvidesFile{glossary.tex}[Glossary and acronym entries + Appendix printing]

% ----------------------------------------------------------------------
% 1) ENTRIES-ONLY MODE (when loaded in the preamble)
% ----------------------------------------------------------------------
\ifdefined\GLOSSARYENTRIESONLY

% ---------- Acronyms ----------
\newacronym{ai}{AI}{artificial intelligence}
\newacronym{ltv}{LTV}{labour theory of value}
\newacronym{snlt}{SNLT}{socially necessary labour time}
\newacronym{melt}{MELT}{monetary expression of labour time}
\newacronym{occ}{OCC}{organic composition of capital}
\newacronym{trpf}{TRPF}{tendency of the rate of profit to fall}

\newacronym{ubi}{UBI}{universal basic income}
\newacronym{ubs}{UBS}{universal basic services}
\newacronym{qe}{QE}{quantitative easing}
\newacronym{mmt}{MMT}{Modern Monetary Theory}
\newacronym{jg}{JG}{Job Guarantee}
\newacronym{esg}{ESG}{environmental, social, and governance}

\newacronym{imf}{IMF}{International Monetary Fund}
\newacronym{ipcc}{IPCC}{Intergovernmental Panel on Climate Change}

% ---------- Glossary terms (existing) ----------
\newglossaryentry{keynesianism}{
  name={Keynesianism},
  sort={Keynesianism},
  description={A macroeconomic approach associated with John Maynard Keynes that emphasises stabilising output and employment via demand management (especially fiscal policy and public spending), particularly during downturns.},
  first={Keynesianism (a macroeconomic approach that emphasises demand management—especially fiscal policy and public spending—to stabilise output and employment)}
}

\newglossaryentry{monetarism}{
  name={Monetarism},
  sort={Monetarism},
  description={A macroeconomic doctrine associated with Milton Friedman that prioritises controlling inflation by managing the money supply and/or interest rates, often via tight monetary policy.},
  first={Monetarism (a macroeconomic doctrine that prioritises inflation control via tight monetary policy—money supply and/or interest rates)}
}

\newglossaryentry{financialisation}{
  name={financialisation},
  sort={financialisation},
  description={A pattern in which profits, strategies, and power shift toward finance, asset price inflation, and rent extraction, rather than expanded productive investment and wage growth.}
}

\newglossaryentry{fictitious-capital}{
  name={fictitious capital},
  sort={fictitious capital},
  description={Tradable claims on future income streams (shares, bonds, securitised claims, many derivatives) whose market valuation can expand beyond surplus value currently produced, until crises force devaluation.}
}

\newglossaryentry{decommodification}{
  name={decommodification},
  sort={decommodification},
  description={Shifting access to essentials (housing, health, care, transport, energy, water) out of the market and away from ability to pay, toward rights-based provision.}
}

\newglossaryentry{capital-controls}{
  name={capital controls},
  sort={capital controls},
  description={Regulatory restrictions on cross-border movement of capital designed to limit capital flight, currency pressure, and the ability of owners to discipline reforms through financial exit.}
}

% ---------- Glossary terms (extended, targeted additions) ----------
\newglossaryentry{austerity}{
  name={austerity},
  sort={austerity},
  description={A policy package of spending cuts, hiring freezes, welfare retrenchment, regressive taxation, and/or user fees justified as “fiscal discipline”. In practice it often shifts crisis costs onto workers and the poor while protecting creditors and asset owners.}
}

\newglossaryentry{fiscal-consolidation}{
  name={fiscal consolidation},
  sort={fiscal consolidation},
  description={Reducing government deficits through spending cuts and/or tax rises. It is frequently presented as technocratic “budget repair,” but its class content depends on who is taxed, which services are cut, and whether interest payments to creditors are ring-fenced.}
}

\newglossaryentry{primary-balance}{
  name={primary balance},
  sort={primary balance},
  description={A government’s fiscal balance excluding interest payments on existing debt. A “primary surplus” can coexist with rising total debt burdens if interest costs remain high or growth is weak.}
}

\newglossaryentry{debt-service}{
  name={debt servicing},
  sort={debt servicing},
  description={Ongoing payments of interest and principal on debt. For many states, debt service becomes a prior claim on public revenue, structurally pressuring social spending and investment even without an explicit austerity programme.}
}

\newglossaryentry{conditionality}{
  name={conditionality},
  sort={conditionality},
  description={Policy conditions attached to loans or debt restructuring (commonly by the IMF or creditor blocs). Typical conditions include subsidy cuts, wage restraint, privatisation, deregulation, central bank “independence,” and fiscal consolidation.}
}

\newglossaryentry{structural-adjustment}{
  name={structural adjustment},
  sort={structural adjustment},
  description={A reform programme—historically associated with IMF/World Bank lending—that restructures economies toward export orientation, market pricing, privatisation, and reduced public provision. The “adjustment” is usually borne through depressed wages, weakened labour protections, and reduced social spending.}
}

\newglossaryentry{balance-of-payments}{
  name={balance of payments},
  sort={balance of payments},
  description={A country’s accounting of transactions with the rest of the world (trade in goods/services, income flows, and financial transfers). Persistent deficits often create pressure for devaluation, import compression, and external borrowing.}
}

\newglossaryentry{capital-flight}{
  name={capital flight},
  sort={capital flight},
  description={Rapid private movement of funds out of a country (or out of domestic investment into safer assets), often triggered by crisis, political conflict, or expectations of devaluation. Capital flight can force currency pressure, reserve loss, and harsher adjustment.}
}

\newglossaryentry{exchange-rate-pass-through}{
  name={exchange-rate pass-through},
  sort={exchange rate pass-through},
  description={The extent to which a currency devaluation raises domestic prices, especially for imported essentials (fuel, fertiliser, medicine, machinery). High pass-through can turn devaluation into immediate inflation and real-wage cuts.}
}

\newglossaryentry{inflation-targeting}{
  name={inflation targeting},
  sort={inflation targeting},
  description={A monetary-policy framework where the central bank prioritises hitting an inflation target, typically via interest-rate moves. Critics argue it can treat inflation as a purely monetary phenomenon while ignoring supply shocks, monopoly pricing, and import dependence.}
}

\newglossaryentry{protectionism}{
  name={protectionism},
  sort={protectionism},
  description={Using tariffs, quotas, licensing, local-content rules, or public procurement to shelter domestic producers from foreign competition. It can defend jobs and industrial capacity, but its effects depend on who controls protected firms, how prices/wages move, and whether technology/inputs are domestically available.}
}

\newglossaryentry{trade-liberalisation}{
  name={trade liberalisation},
  sort={trade liberalisation},
  description={Reducing tariffs, quotas, and other trade barriers. It is often sold as “efficiency,” but in unequal world markets it can accelerate deindustrialisation, worsen trade deficits, and deepen dependence on imported inputs and foreign currency.}
}

\newglossaryentry{import-substitution}{
  name={import-substitution industrialisation (ISI)},
  sort={import substitution industrialisation},
  description={A strategy to replace imports with domestic production through tariffs, credit allocation, industrial policy, and state procurement. ISI can build capacity, but it often hits constraints around technology, energy, foreign exchange, and class control of investment decisions.}
}

\newglossaryentry{qe-term}{
  name={Quantitative easing (QE)},
  sort={Quantitative easing},
  description={A central bank policy of purchasing government bonds and/or other financial assets to expand its balance sheet and push down longer-term interest rates. QE can stabilise financial markets, but it often inflates asset prices and does not automatically translate into productive investment or higher wages.}
}

\newglossaryentry{mmt-term}{
  name={Modern Monetary Theory (MMT)},
  sort={Modern Monetary Theory},
  description={A heterodox framework arguing that a state that issues its own currency cannot “run out” of money in the way households can; the binding constraints are real resources, productive capacity, and inflation dynamics. It stresses the role of taxation and bond issuance in managing demand, distribution, and monetary conditions rather than “funding” spending in a mechanical sense.}
}

\newglossaryentry{ubi-term}{
  name={Universal basic income (UBI)},
  sort={Universal basic income},
  description={An unconditional cash transfer to all residents or citizens. Proposals differ sharply: some are designed to replace welfare and subsidise low wages, while others are framed as an income floor that complements strong public services, labour rights, and decommodification.}
}

\newglossaryentry{ubs-term}{
  name={Universal basic services (UBS)},
  sort={Universal basic services},
  description={A model of guaranteeing key services—health, education, housing, transport, care, water/energy—through public provision or social rights rather than cash transfers. UBS centres decommodification and collective infrastructure, but requires fiscal capacity, democratic control, and organised labour to prevent deterioration or capture.}
}

\newglossaryentry{jg-term}{
  name={Job Guarantee (JG)},
  sort={Job Guarantee},
  description={A proposal that the state offers a public job at a socially defined wage to anyone willing to work. Advocates treat it as an employment floor and stabiliser; critics debate job quality, political control, and whether it can be insulated from austerity and patronage without strong democratic governance.}
}

\newglossaryentry{esg-term}{
  name={ESG},
  sort={ESG},
  description={A framework used by investors and firms to score “environmental, social, and governance” performance. ESG can pressure disclosure and some standards, but it is often criticised as compatible with continued extraction and financialisation, turning ecological crisis into a portfolio and reputational management problem.}
}

\newglossaryentry{troika}{
  name={the Troika},
  sort={Troika},
  description={A term commonly used for the European Commission (EC), the European Central Bank (ECB), and the IMF acting jointly in crisis programmes and conditional lending in the Eurozone.}
}

\newglossaryentry{eurozone}{
  name={Eurozone},
  sort={Eurozone},
  description={The group of EU member states using the euro. Eurozone membership removes independent monetary policy and exchange-rate adjustment, making fiscal policy and wage/price dynamics central sites of “internal devaluation” during crises.}
}

\newglossaryentry{syriza}{
  name={Syriza},
  sort={Syriza},
  description={A Greek left party (Coalition of the Radical Left) that came to power in 2015 on an anti-austerity mandate. Its confrontation with the Troika became a major reference point for debates on debt, monetary sovereignty, and the limits imposed by Eurozone institutions.}
}

% ----------------------------------------------------------------------
% 2) PRINTING MODE (when % ============================
%  glossary.tex
%  Acronyms + glossary entries (glossaries-extra, no-index workflow)
%  ALSO prints Appendix B when \input{glossary} is used in content.tex.
%
%  IMPORTANT (preamble.tex):
%    \usepackage[acronym]{glossaries-extra}
%    \makenoidxglossaries
%    \def\GLOSSARYENTRIESONLY{}%
%    \loadglsentries{glossary}
%    \let\GLOSSARYENTRIESONLY\undefined
% ============================

\ProvidesFile{glossary.tex}[Glossary and acronym entries + Appendix printing]

% ----------------------------------------------------------------------
% 1) ENTRIES-ONLY MODE (when loaded in the preamble)
% ----------------------------------------------------------------------
\ifdefined\GLOSSARYENTRIESONLY

% ---------- Acronyms ----------
\newacronym{ai}{AI}{artificial intelligence}
\newacronym{ltv}{LTV}{labour theory of value}
\newacronym{snlt}{SNLT}{socially necessary labour time}
\newacronym{melt}{MELT}{monetary expression of labour time}
\newacronym{occ}{OCC}{organic composition of capital}
\newacronym{trpf}{TRPF}{tendency of the rate of profit to fall}

\newacronym{ubi}{UBI}{universal basic income}
\newacronym{ubs}{UBS}{universal basic services}
\newacronym{qe}{QE}{quantitative easing}
\newacronym{mmt}{MMT}{Modern Monetary Theory}
\newacronym{jg}{JG}{Job Guarantee}
\newacronym{esg}{ESG}{environmental, social, and governance}

\newacronym{imf}{IMF}{International Monetary Fund}
\newacronym{ipcc}{IPCC}{Intergovernmental Panel on Climate Change}

% ---------- Glossary terms (existing) ----------
\newglossaryentry{keynesianism}{
  name={Keynesianism},
  sort={Keynesianism},
  description={A macroeconomic approach associated with John Maynard Keynes that emphasises stabilising output and employment via demand management (especially fiscal policy and public spending), particularly during downturns.},
  first={Keynesianism (a macroeconomic approach that emphasises demand management—especially fiscal policy and public spending—to stabilise output and employment)}
}

\newglossaryentry{monetarism}{
  name={Monetarism},
  sort={Monetarism},
  description={A macroeconomic doctrine associated with Milton Friedman that prioritises controlling inflation by managing the money supply and/or interest rates, often via tight monetary policy.},
  first={Monetarism (a macroeconomic doctrine that prioritises inflation control via tight monetary policy—money supply and/or interest rates)}
}

\newglossaryentry{financialisation}{
  name={financialisation},
  sort={financialisation},
  description={A pattern in which profits, strategies, and power shift toward finance, asset price inflation, and rent extraction, rather than expanded productive investment and wage growth.}
}

\newglossaryentry{fictitious-capital}{
  name={fictitious capital},
  sort={fictitious capital},
  description={Tradable claims on future income streams (shares, bonds, securitised claims, many derivatives) whose market valuation can expand beyond surplus value currently produced, until crises force devaluation.}
}

\newglossaryentry{decommodification}{
  name={decommodification},
  sort={decommodification},
  description={Shifting access to essentials (housing, health, care, transport, energy, water) out of the market and away from ability to pay, toward rights-based provision.}
}

\newglossaryentry{capital-controls}{
  name={capital controls},
  sort={capital controls},
  description={Regulatory restrictions on cross-border movement of capital designed to limit capital flight, currency pressure, and the ability of owners to discipline reforms through financial exit.}
}

% ---------- Glossary terms (extended, targeted additions) ----------
\newglossaryentry{austerity}{
  name={austerity},
  sort={austerity},
  description={A policy package of spending cuts, hiring freezes, welfare retrenchment, regressive taxation, and/or user fees justified as “fiscal discipline”. In practice it often shifts crisis costs onto workers and the poor while protecting creditors and asset owners.}
}

\newglossaryentry{fiscal-consolidation}{
  name={fiscal consolidation},
  sort={fiscal consolidation},
  description={Reducing government deficits through spending cuts and/or tax rises. It is frequently presented as technocratic “budget repair,” but its class content depends on who is taxed, which services are cut, and whether interest payments to creditors are ring-fenced.}
}

\newglossaryentry{primary-balance}{
  name={primary balance},
  sort={primary balance},
  description={A government’s fiscal balance excluding interest payments on existing debt. A “primary surplus” can coexist with rising total debt burdens if interest costs remain high or growth is weak.}
}

\newglossaryentry{debt-service}{
  name={debt servicing},
  sort={debt servicing},
  description={Ongoing payments of interest and principal on debt. For many states, debt service becomes a prior claim on public revenue, structurally pressuring social spending and investment even without an explicit austerity programme.}
}

\newglossaryentry{conditionality}{
  name={conditionality},
  sort={conditionality},
  description={Policy conditions attached to loans or debt restructuring (commonly by the IMF or creditor blocs). Typical conditions include subsidy cuts, wage restraint, privatisation, deregulation, central bank “independence,” and fiscal consolidation.}
}

\newglossaryentry{structural-adjustment}{
  name={structural adjustment},
  sort={structural adjustment},
  description={A reform programme—historically associated with IMF/World Bank lending—that restructures economies toward export orientation, market pricing, privatisation, and reduced public provision. The “adjustment” is usually borne through depressed wages, weakened labour protections, and reduced social spending.}
}

\newglossaryentry{balance-of-payments}{
  name={balance of payments},
  sort={balance of payments},
  description={A country’s accounting of transactions with the rest of the world (trade in goods/services, income flows, and financial transfers). Persistent deficits often create pressure for devaluation, import compression, and external borrowing.}
}

\newglossaryentry{capital-flight}{
  name={capital flight},
  sort={capital flight},
  description={Rapid private movement of funds out of a country (or out of domestic investment into safer assets), often triggered by crisis, political conflict, or expectations of devaluation. Capital flight can force currency pressure, reserve loss, and harsher adjustment.}
}

\newglossaryentry{exchange-rate-pass-through}{
  name={exchange-rate pass-through},
  sort={exchange rate pass-through},
  description={The extent to which a currency devaluation raises domestic prices, especially for imported essentials (fuel, fertiliser, medicine, machinery). High pass-through can turn devaluation into immediate inflation and real-wage cuts.}
}

\newglossaryentry{inflation-targeting}{
  name={inflation targeting},
  sort={inflation targeting},
  description={A monetary-policy framework where the central bank prioritises hitting an inflation target, typically via interest-rate moves. Critics argue it can treat inflation as a purely monetary phenomenon while ignoring supply shocks, monopoly pricing, and import dependence.}
}

\newglossaryentry{protectionism}{
  name={protectionism},
  sort={protectionism},
  description={Using tariffs, quotas, licensing, local-content rules, or public procurement to shelter domestic producers from foreign competition. It can defend jobs and industrial capacity, but its effects depend on who controls protected firms, how prices/wages move, and whether technology/inputs are domestically available.}
}

\newglossaryentry{trade-liberalisation}{
  name={trade liberalisation},
  sort={trade liberalisation},
  description={Reducing tariffs, quotas, and other trade barriers. It is often sold as “efficiency,” but in unequal world markets it can accelerate deindustrialisation, worsen trade deficits, and deepen dependence on imported inputs and foreign currency.}
}

\newglossaryentry{import-substitution}{
  name={import-substitution industrialisation (ISI)},
  sort={import substitution industrialisation},
  description={A strategy to replace imports with domestic production through tariffs, credit allocation, industrial policy, and state procurement. ISI can build capacity, but it often hits constraints around technology, energy, foreign exchange, and class control of investment decisions.}
}

\newglossaryentry{qe-term}{
  name={Quantitative easing (QE)},
  sort={Quantitative easing},
  description={A central bank policy of purchasing government bonds and/or other financial assets to expand its balance sheet and push down longer-term interest rates. QE can stabilise financial markets, but it often inflates asset prices and does not automatically translate into productive investment or higher wages.}
}

\newglossaryentry{mmt-term}{
  name={Modern Monetary Theory (MMT)},
  sort={Modern Monetary Theory},
  description={A heterodox framework arguing that a state that issues its own currency cannot “run out” of money in the way households can; the binding constraints are real resources, productive capacity, and inflation dynamics. It stresses the role of taxation and bond issuance in managing demand, distribution, and monetary conditions rather than “funding” spending in a mechanical sense.}
}

\newglossaryentry{ubi-term}{
  name={Universal basic income (UBI)},
  sort={Universal basic income},
  description={An unconditional cash transfer to all residents or citizens. Proposals differ sharply: some are designed to replace welfare and subsidise low wages, while others are framed as an income floor that complements strong public services, labour rights, and decommodification.}
}

\newglossaryentry{ubs-term}{
  name={Universal basic services (UBS)},
  sort={Universal basic services},
  description={A model of guaranteeing key services—health, education, housing, transport, care, water/energy—through public provision or social rights rather than cash transfers. UBS centres decommodification and collective infrastructure, but requires fiscal capacity, democratic control, and organised labour to prevent deterioration or capture.}
}

\newglossaryentry{jg-term}{
  name={Job Guarantee (JG)},
  sort={Job Guarantee},
  description={A proposal that the state offers a public job at a socially defined wage to anyone willing to work. Advocates treat it as an employment floor and stabiliser; critics debate job quality, political control, and whether it can be insulated from austerity and patronage without strong democratic governance.}
}

\newglossaryentry{esg-term}{
  name={ESG},
  sort={ESG},
  description={A framework used by investors and firms to score “environmental, social, and governance” performance. ESG can pressure disclosure and some standards, but it is often criticised as compatible with continued extraction and financialisation, turning ecological crisis into a portfolio and reputational management problem.}
}

\newglossaryentry{troika}{
  name={the Troika},
  sort={Troika},
  description={A term commonly used for the European Commission (EC), the European Central Bank (ECB), and the IMF acting jointly in crisis programmes and conditional lending in the Eurozone.}
}

\newglossaryentry{eurozone}{
  name={Eurozone},
  sort={Eurozone},
  description={The group of EU member states using the euro. Eurozone membership removes independent monetary policy and exchange-rate adjustment, making fiscal policy and wage/price dynamics central sites of “internal devaluation” during crises.}
}

\newglossaryentry{syriza}{
  name={Syriza},
  sort={Syriza},
  description={A Greek left party (Coalition of the Radical Left) that came to power in 2015 on an anti-austerity mandate. Its confrontation with the Troika became a major reference point for debates on debt, monetary sovereignty, and the limits imposed by Eurozone institutions.}
}

% ----------------------------------------------------------------------
% 2) PRINTING MODE (when \input{glossary} is called in content.tex)
% ----------------------------------------------------------------------
\else

\section{Glossary and acronyms}
\label{sec:glossary}

\begingroup
\setlength{\parskip}{0pt}

% Print even if entries were not referenced yet:
\glsaddallunused

% Acronyms
\printnoidxglossary[type=\acronymtype,style=compactgls,title={Acronyms}]

\vspace{0.6\baselineskip}

% Terms
\printnoidxglossary[style=termscolon,title={Glossary of terms}]

\endgroup

\fi is called in content.tex)
% ----------------------------------------------------------------------
\else

\section{Glossary and acronyms}
\label{sec:glossary}

\begingroup
\setlength{\parskip}{0pt}

% Print even if entries were not referenced yet:
\glsaddallunused

% Acronyms
\printnoidxglossary[type=\acronymtype,style=compactgls,title={Acronyms}]

\vspace{0.6\baselineskip}

% Terms
\printnoidxglossary[style=termscolon,title={Glossary of terms}]

\endgroup

\fi is called in content.tex)
% ----------------------------------------------------------------------
\else

\section{Glossary and acronyms}
\label{sec:glossary}

\begingroup
\setlength{\parskip}{0pt}

% Print even if entries were not referenced yet:
\glsaddallunused

% Acronyms
\printnoidxglossary[type=\acronymtype,style=compactgls,title={Acronyms}]

\vspace{0.6\baselineskip}

% Terms
\printnoidxglossary[style=termscolon,title={Glossary of terms}]

\endgroup

\fi is called in content.tex)
% ----------------------------------------------------------------------
\else

\section{Glossary and acronyms}
\label{sec:glossary}

\begingroup
\setlength{\parskip}{0pt}

% Print even if entries were not referenced yet:
\glsaddallunused

% Acronyms
\printnoidxglossary[type=\acronymtype,style=compactgls,title={Acronyms}]

\vspace{0.6\baselineskip}

% Terms
\printnoidxglossary[style=termscolon,title={Glossary of terms}]

\endgroup

\fi