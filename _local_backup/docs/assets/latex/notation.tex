\section{Marxian notation at a glance}
\label{sec:notation}

\noindent This pamphlet uses a compact set of standard Marxian symbols. Where a symbol can mean different things in different contexts (for example, $k$ as ``cost price'' versus $k$ as a conversion factor), we state explicitly which meaning we use locally. We also use indices ($i,j,\dots$) as generic labels whose meaning is always stated in context (e.g.\ commodity $i$, firm $j$, industry $i$). We reserve $d$ for the discount (interest) rate in present-value expressions to avoid confusion with indices. Where we refer to prices or money magnitudes in numerical examples, we flag explicitly when we are using a simplifying proportionality between money magnitudes and values for expository clarity.

\begingroup
\footnotesize
\setlength{\LTpre}{6pt}      % space between paragraph and caption
\setlength{\LTpost}{0pt}
\setlength{\LTcapwidth}{\textwidth}
\setlength{\tabcolsep}{4pt}
\renewcommand{\arraystretch}{1.18} % a touch more breathing room

\begin{longtable}{@{}>{\centering\arraybackslash}p{0.12\textwidth}@{\hspace{8pt}}%
                  >{\raggedright\arraybackslash}p{\dimexpr\textwidth-0.12\textwidth-8pt\relax}@{}}

\caption{Key symbols and definitions used throughout the pamphlet.}\label{tab:notation}%
\tabularnewline[-0.6\baselineskip]   % pulls the top rule up toward the caption

% ---------- First page header ----------
\toprule
\textbf{Symbol} & \textbf{Meaning (as used in this pamphlet)} \\
\midrule
\endfirsthead

% ---------- Continuation page header ----------
\multicolumn{2}{@{}l@{}}{\footnotesize\itshape \tablename~\thetable\ (continued from previous page)}\\[-0.2\baselineskip]
\toprule
\textbf{Symbol} & \textbf{Meaning (as used in this pamphlet)} \\
\midrule
\endhead

% ---------- Continuation page footer (all but last page) ----------
\midrule
\multicolumn{2}{r@{}}{\footnotesize\itshape Continued on next page}\\[-0.2\baselineskip]
\bottomrule
\endfoot

% ---------- Last page footer ----------
\bottomrule
\endlastfoot

% --- rows start here ---

$i$ &
generic index used for a particular object (commodity $i$, industry $i$, etc.); the meaning is always stated locally at the point of use. \\

$j$ &
index for a producer/firm (firm $j$) when comparing multiple producers; in other contexts it may serve as a second commodity index, and the meaning is always stated locally. \\

$n$ &
number of firms/producers included in a simple multi-producer illustration (where needed). \\

$d$ &
discount (interest) rate used for present-value calculations (e.g.\ in the fictitious-capital formula). \\

$C$ &
a commodity (in circuit notation), or commodity-capital as the output of production. \\

$M$ &
money-capital advanced; in the circuit $M \rightarrow C \rightarrow M'$. \\

$M'$ &
money realised after sale, $M' = M + \Delta M$ (surplus-value realised in money-form). \\

$\Delta M$ &
increment of money in the circuit of capital: $\Delta M = M' - M$ (surplus-value realised in money-form). \\

$W$ &
total value of the commodity product of a given production period: $W = c + v + s$. \\

$c$ &
\textbf{constant capital}: value of used-up means of production (materials, energy, depreciation of machinery/buildings) transferred into the product; it does \emph{not} create new value. \\

$v$ &
\textbf{variable capital}: value advanced to purchase labour-power (the wage bill in value terms); ``variable'' because living labour can create more value than it costs. \\

$s$ &
\textbf{surplus value}: unpaid value created by living labour beyond $v$, appropriated by capital; in value terms $s = W - (c+v)$. \\

$e$ &
\textbf{rate of exploitation} (rate of surplus value): $e = \dfrac{s}{v}$. \\

$r$ &
\textbf{rate of profit}: $r = \dfrac{s}{c+v}$. \\

$\mathrm{OCC}$ &
\textbf{organic composition of capital}: $\mathrm{OCC} = \dfrac{c}{v}$. \\

$\mathrm{TRPF}$ &
\textbf{tendency of the rate of profit to fall}: \emph{ceteris paribus}, rising $\mathrm{OCC}$ tends to press $r$ downward over long periods (with counter-tendencies). \\

$T$ &
length of the working day (hours). \\

$T_{\mathrm{N}}$ &
\textbf{necessary labour time}: part of the working day that reproduces the value of labour-power (corresponding to $v$). \\

$T_{\mathrm{S}}$ &
\textbf{surplus labour time}: part of the working day beyond $T_{\mathrm{N}}$ that produces surplus value (corresponding to $s$); $T_{\mathrm{S}} = T - T_{\mathrm{N}}$. \\

$w_h$ &
money wage per hour under time-wages. \\

$w_p$ &
piece rate: money paid per unit under piece-wages. \\

$q$ &
output per hour (units/hour) in a simple piece-wage illustration. \\

$W_w$ &
money wage bill (total wages paid in money terms), introduced where needed to avoid clashing with $W$; for time-wages, $W_w = w_h\,T$. \\

$\mathrm{SNLT}$ &
\textbf{socially necessary labour time}: labour time required to produce a commodity under average conditions of production, with average skill and intensity, using prevailing technology. \\

$L^{\mathrm{SN}}_i$ &
SNLT for commodity $i$: socially necessary labour time required to produce one unit of commodity $i$ under average, prevailing conditions. \\

$L_{ij}$ &
labour time per unit of commodity $i$ at firm $j$ (as produced by $j$; in practice this stands in for both direct and indirect labour time embodied in the unit). \\

$q_{ij}$ &
output of commodity $i$ produced by firm $j$ in the period (used as a weight in output-weighted averages). \\

$k$ &
\textbf{cost price}: the money-capital laid out by the capitalist to purchase means of production and labour-power; in value terms, $k = c+v$. \\

$\MELT$ &
\textbf{monetary expression of labour time (MELT)}: conversion factor from labour-time to money magnitudes, used here as money per hour of \textbf{SNLT}. In this pamphlet we denote this conversion factor by $\MELT$ throughout. If $L^{\mathrm{SN}}_i$ is the socially necessary labour time embodied in one unit of commodity $i$, then its value in money terms is $V_i = \MELT\,L^{\mathrm{SN}}_i$. \\

$V_i$ &
value of one unit of commodity $i$ in money terms: $V_i = \MELT\,L^{\mathrm{SN}}_i$. \\

$K_f$ &
\textbf{fictitious capital}: market valuation (price) of a tradable financial claim on future income, typically modelled as the discounted present value of expected receipts. \\

$R_t$ &
expected net receipts (cash flows) at time $t$ for a financial claim (dividends, interest, rents, or other contractual payoffs), used in present-value expressions. \\

\end{longtable}
\endgroup

\subsection*{Reminder: two circuits of circulation}

We use standard circuit shorthand (with arrows indicating exchange, and with the dashed circuit notation \CMC{} and \MCM{} used elsewhere as compact labels):

\begin{itemize}[leftmargin=1.5em]
  \item $C \rightarrow M \rightarrow C$ (\CMC): selling in order to buy; the goal is use-value.
  \item $M \rightarrow C \rightarrow M'$ (\MCM): buying in order to sell; the goal is $\Delta M$.
\end{itemize}